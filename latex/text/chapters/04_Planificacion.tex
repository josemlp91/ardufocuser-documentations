\chapter{Planificación}

\section{Fases}

Dado que el proyecto cuenta con módulos que pueden considerarse subsistemas de un sistema de magnitud mayor. 
Una vez identificados los módulos que conforman el sistema, podemos hacer una separación en fases que nos ayude a evaluar las tareas y
hacer estimaciones. Adicionalmente debemos contemplar un coste extra de integración de los módulos.

\begin{itemize}
	
	\item \textbf{ Módulo Hardware}
	\begin{itemize}
		\item \textbf{Fase 0:} Planteamiento del problema.
		\item \textbf{Fase 1:} Investigar tecnologías implicadas.
		\item \textbf{Fase 2:} Análisis y diseño.
		\item \textbf{Fase 3:} Implementación.
		\item \textbf{Fase 4:} Pruebas.
	\end{itemize}
	
	\item \textbf{Módulo Firmware}
	\begin{itemize}
		\item \textbf{Fase 0:} Planteamiento del problema.
		\item \textbf{Fase 1:} Investigar librerías y lenguaje programación.
		\item \textbf{Fase 2:} Análisis y diseño.
		\item \textbf{Fase 3:} Implementación.
		\item \textbf{Fase 4:} Pruebas.
	\end{itemize}
	
	\item \textbf{Módulo Driver Indi}
	\begin{itemize}
		\item \textbf{Fase 0:} Planteamiento del problema.
		\item \textbf{Fase 1:} Familiarizarme con arquitectura INDI.
		\item \textbf{Fase 2:} Análisis y diseño.
		\item \textbf{Fase 3:} Implementación.
		\item \textbf{Fase 4:} Pruebas.
	\end{itemize}
	
	\item \textbf{Módulo Software} 
	\begin{itemize}
		\item \textbf{Fase 0:} Planteamiento del problema.
		\item \textbf{Fase 1:} Investigar entorno de procesamiento de imágenes.
		\item \textbf{Fase 2:} Análisis y diseño.
		\item \textbf{Fase 3:} Implementación.
		\item \textbf{Fase 4:} Pruebas.
	\end{itemize}
		\item \textbf{Integración} 
		\item \textbf{Pruebas de integración}
		\item \textbf{Documentación}
\end{itemize}



\begin{figure}[h]
\centering
\includegraphics[width=0.7\linewidth]{../images/Fases}
\caption{Desarrollo de los módulos}
\label{fig:Fases}
\end{figure}



\section{Estimación de tiempos}

Dicho desglose lo realizo en el capitulo destinado a cada uno de los módulos, entrado en las tareas realizadas en cada una de las fases.


\section{Recursos humanos}

Dado que el objetivo del proyecto es formarme tanto en la gestión, como en el desarrollo de un proyecto, he sido el único integrante del equipo técnico, a excepción de tareas que requerían manejo de maquinaria avanzada, como cortadora laser, CNC o torno. 

\section{Presupuesto}

Para el presente proyecto se tendrán en cuenta los siguientes costes:

\begin{itemize}
	
	\item \textbf{Costes componentes y hardware necesario}, hacemos un desglose pormenorizado en el capitulo hardware.
	
	\item \textbf{Costes por hora de equipo humano}, a 25\euro la hora.
	
	\item \textbf{Costes asociados a licencias necesarias para publicar o desarrollar
	el software} dado que trabajamos con GNU3 \cite{GNU3}, no tenemos que invertir dinero en este aspecto.

	\item \textbf{Máquina de desarrollo}, portátil valorado en 400\euro con sistema operativo Ubuntu 15.10
	
	\item \textbf{Servidor INDI}, Raspberry Pi, B+ que tiene un coste asociado de 35\euro. Para realizar una instalación de Servidor Indi con el Driver del Ardufocuser.
 
\end{itemize}

\section{Metodología}

\bigskip
En este punto explicamos la metodologia seguida para conseguir  solucionar el problema propuesto, cumpliendo cada uno de los objetivos. 
Podemos diferenciar grandes bloques o subsistemas, que dada la diferente naturaleza debemos analizar, diseñar, implementar y testear por separado, para posteriormente dedicarnos a la integración de unos módulos con los otros.

\bigskip
Es importante mencionar que dado la novedad del proyecto y la gran labor de investigación que necesitaba, tanto a nivel conceptual como a nivel técnico, no se ha tratado con un diseño inicial totalmente estanco y cerrado, se deja abierto en muchos aspectos, y a medida que han surgido nuevas ideas o se han descubierto tecnologías, no se tenga limitación en incorporarlas, siempre con el objetivo de  optimizar la solución mediante refinamiento.

\bigskip
La metodología seguida en el proyecto podemos enmarcarla como \textit{"Iterativa basada en prototipos"}, para cada módulo se crean versiones 
cada vez más optimas y aproximadas a la solución optima.

Esto permite:

\begin{itemize}
	\item Una rápida retroalimentación por parte del cliente y validación de los requisitos. 
	\item Optimizacion de la solución según se tiene un conocimiento más fuerte del dominio, la tecnología manejada.
	\item Los prototipos son rápidos y economicos de construir.
	\item Los prototipos garantizan que el producto se adapta a las necesidades del cliente, dado que el cliente es participe del dasarrollo mediante revisiones continuas. 
	\item Los prototipos constituyen un medio de comunicación mejor que los modelos formales, ya sean diagramas o listados de requisitos. 
	\item Permiten refinar la estimación de tiemposde desarrollo, observando de forma más realista las dificultades que se van dando.  
	
\end{itemize}

\begin{figure}[h]
	\centering
	\includegraphics[width=0.7\linewidth]{../images/analisis1}
	\caption[Iterativa basada en prototipos modular]{Iterativa basada en prototipos modular}
	\label{fig:}
\end{figure}
\newpage
