\chapter{Análisis}

\bigskip
En este punto pasamos a analizar la metodologia seguida para conseguir solucionar el problema propuesto, cumpliendo cada uno de los objetivos. 
Podemos diferenciar grandes bloques o subsistemas, que ademas de analizar, diseñar, implementar y testear, tenemos que  hacer una integración de los unos con los otros.

\bigskip
Es importante mencionar que dado la novedad del proyecto y la gran labor de investigación que necesitaba, tanto a nivel conceptual como a nivel técnico, no se ha tratado con un diseño inicial totalmente estanco y cerrado, se deja abierto en muchos aspectos, y a medida que han surgido nuevas ideas o se han descubierto tecnologías, no se tenga limitación en incorporarlas, siempre con el objetivo de  optimizar la solución, consiguiendo una solución lo más elegante y eficiente posible. Además según se ha avanzado sobre capas superiores del del proyecto hemos detectado deficiencias en requisitos iniciales en capas iniciales, teniendo que hacer una nueva iteración en la capa inferior y incorporar tal requisito. 

\bigskip
La filosofía seguida en el proyecto se denomina \textit{"Iterativa basada en prototipos"}, las metodologías basadas en prototipos tienen como finalidad proyectos donde el cliente o usuario final, no tiene claro en primera instancia la totalidad de los requisitos del proyecto, en este proyecto aunque se tienen claros los requisitos, no tanto la manera de conseguir solucionarlos.


  

