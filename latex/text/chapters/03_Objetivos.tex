\chapter{Objetivos}

El objetivo global de este proyecto \textbf{OBJ-G} es desarrollar un sistema automático de \textbf{autoenfoque}.
Dicho sistema debe permitir un modo de funcionamiento remoto utilizando el protocolo \textbf{INDI}.


\bigskip
Este objetivo global se desglosa en los siguientes objetivos principales:

\begin{itemize}
  \item \textbf{OBJ-P1.} Diseño dispositivo hardware, usando plataforma libre Arduino  \textbf{Arduino},
  que interactúa directamente con los mandos de enfoque del \textbf{telescopio}, mediante motor y distintos sensores.
  \item \textbf{OBJ-P2.}-Implementar un firmware, que permita realizar el control de todos los periféricos y electrónica.
  \item \textbf{OBJ-P3.} Implementar  \textbf{Driver INDI} que provea de la funcionalidad básica para controlar
   todas las características o propiedades del dispositivo enfocador.
  \item \textbf{OBJ-P4.} Implementar algoritmo de autofocus, basándonos en la medida del foco y
   coordinada con las funciones del dispositivo y las imágenes capturadas por el CCD.

\item \textbf{OBJ-P5.} Documentarlo extensivamente para que sea reproducible por todos los interesados. (subobjetivos, disfusión, publicitaicón.)

\end{itemize}


\bigskip
Dado los objetivos centrales del proyecto, podemos hacer un desglose en objetivos menores, separándolos a su vez en objetivos  \textbf{Obligatorios},
que deben cumplirse para llegar a completar su correspondiente objetivo principal,
y objetivos  \textbf{Secundarios}, objetivos igualmente interesantes para el proyecto, pero que carecen de importancia vital
y su incumplimiento no evita que se pueda completar su correspondiente objetivo central.

\begin{itemize}
  \item \textbf{OBJ-P1.O1.}  Investigación de los periféricos y sensores existentes para la plataforma Arduino,
  \item \textbf{OBJ-P1.O2.}  Realizar una estimación de precios y gastos en componentes, sopesando distintas alternativas.
  \item \textbf{OBJ-P1.O3.}  Investigar sobre los distintos métodos de mecanización para los materiales empleados, así como posibles alternativas para crear la PCB.
  \item \textbf{OBJ-P1.O4.}  Implementar un prototipo, completo.
  \item \textbf{OBJ-P1.S5.}  Implementación de distintas versiones, completa o solo funcionamiento remoto.
\end{itemize}



\begin{itemize}
  \item \textbf{OBJ-P2.O1.} Investigar librerías de control de los periféricos, buscando alternativas y realizando pruebas y ejemplos.
  \item \textbf{OBJ-P2.O2.} Implementar \textbf{Firmware}, basándonos en un diseño modular, permitiendo así fáciles ampliaciones.
  \item \textbf{OBJ-P2.O3.} Definir protocolo de comunicación, mensajes y parámetros, entre ordenador y el dispositivo.

  \item \textbf{OBJ-P2.O4.} Realizar las pruebas pertinentes para comprobar el buen comportamiento de la lógica, sobre la electrónica previamente diseñada, así como la integración de los distintos módulos.
  \item \textbf{OBJ-P2.S4.} Implementar protocolo de intercambio de mensajes \textbf{Robofocus}, (enfocador comercial de características similares).
\end{itemize}

\begin{itemize}
\item \textbf{OBJ-P2.O2.M1} Módulo de control de motores paso a paso.
\item \textbf{OBJ-P2.O2.M2} Módulo visualización de datos en pantalla LCD.
\item \textbf{OBJ-P2.O2.M3} Módulo control manual.
\item \textbf{OBJ-P2.O2.M4} Módulo control remoto y comunicación con host.
\item \textbf{OBJ-P2.O2.M4} Módulo de sensores externos.
\end{itemize}


\begin{itemize}
  \item \textbf{OBJ-3.O1.} Que cumpla el entandar INDI para enfoca dores
  \item \textbf{OBJ-3.S2.} Que el driver proporcione información sobre los sensores.
  \item \textbf{OBJ-3.S3.} Realizar script de despliegue automatizado, para instalar y correr el Servidor INDI con su correspondiente Driver Ardufocuser.

\end{itemize}


\begin{itemize}
  \item \textbf{OBJ-4.S1} Implementar framework de visualización y procesamiento de imagenes, que permita evaluar y parametrizar, de forma simple los algoritmos que aplicamos sobre la imagen.
  \item \textbf{OBJ-4.O2} Diseño y implementación de algoritmos de detección de objetos celestes, con propiedades estelares, basándonos en la curva de luz característica.
  \item \textbf{OBJ-4.O3} Realizar cálculo de distintas medidas de enfoque, calculo del FWHM.
  \item \textbf{OBJ-4.O4} Dada la medida anterior, implementar algoritmo de búsqueda de máximo enfoque, coordinado con el movimiento del enfocador y la adquisición de las imagenes por parte de la CCD.

\end{itemize}

\bigskip
Además de los objetivos anteriores, se persigue alcanzar los siguientes objetivos secundarios globales:

\begin{itemize}
  \item \textbf{OBJ-5-O1.} Completar una buena documentación técnicas, así como manuales para desarrolladores, a fin de que cualquier interesado sea capaz de reproducir fácilmente y a bajo costo el proyecto.
  \item \textbf{OBJ-5-O2.} Publicar código fuente, diseños y documentación bajo una licencia libre, que permita que la comunidad de desarrolladores interesados pueda realizar modificaciones y personalizaciones, siempre que se mantenga la misma licencia.
  \item \textbf{OBJ-5-S3.} Difusión en la comunidad astronómica y desarrollo de una web propia del proyecto.

\end{itemize}

\bigskip
Para la realización de los objetivos se pondrán en practica los conocimientos alcanzados en:

\begin{itemize}
  \item \textbf{Ingeniería del software} para el análisis y diseño del proyecto, así como modelar el sistema.
  \item \textbf{Programación orientada a objetos} para la estructura y la organización del código \textbf{Java}.
  \item \textbf{Programación de sistemas multimedia} para poder implementar las interfaces de usuario en \textbf{Java Swing}, así como visualizar y tratar las imagenes.
  \item \textbf{Infraestructura virtual} para poder gestionar los sistemas, teniendo habilidad para realizar instalaciones y aprovisionamiento del servidor.
  \item \textbf{Transmisión de datos y redes de computadores} para comprender el comportamiento del protocolo \textbf{INDI} y configurar correctamente las redes para las pruebas.
  \item \textbf{Algorítmica} para optimizar las rutinas de tratamiento de imagenes.
  \item \textbf{Calculo matemático} para modelar los objetos celestes como una función gaussiana que representa la luminosidad, así como el calculo del FWHM.
  \item \textbf{Estructura de datos} dado que contamos con representaciones de las imagenes, que debemos conocer y saber manejar.


\end{itemize}

\bigskip
Por otro lado, han sido necesarios alcanzar conocimientos en otras áreas:

\begin{itemize}

  \item \textbf{Astronomía básica y equipos astronómicos} para entender a los usuarios potenciales y poder acomodar la aplicación a sus necesidades.
  \item \textbf{Raspberry Pi}\footnote{Ordenador de placa reducida y única de bajo coste.} para montar un servidor permanente de pruebas o acceso público para probar la aplicación
  \item \textbf{Latex}\footnote{Sistema de composición de textos.} para la realización del presente documento y la ampliación de conocimientos para futuros textos científicos.
  \item \textbf{Git} para la gestión de versiones y la publicación de código abierto que permita a otros desarrolladores participar.
\end{itemize}
