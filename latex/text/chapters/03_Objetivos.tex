\chapter{Objetivos}

El objetivo global de este proyecto es desarrollar un sistema automático de \textbf{autoenfoque}, dicho sistema debe permitir un modo de funcionamiento remoto utilizando el protocolo \textbf{INDI},
adicionalmente se implementa una interfaz  \textbf{JAVA} que permita visualizar el funcionamiento de las distintas rutinas de procesamiento de imagenes,
sirviéndonos como framework de trabajo, por último se implementa una interfaz que permite ejecutar directamente la rutina de enfoque automático,
así como interactuar con el dispositivo enfocador de forma remota.


\bigskip
Este objetivo se desglosa en los siguientes objetivos principales:

\begin{itemize}
  \item \textbf{OBJ-1.} Diseño dispositivo usando plataforma libre Arduino  \textbf{Arduino}, que interactúa directamente con controles de enfoque de \textbf{telescopios}, mediante motor y distintos sensores, Ardufocuser.
  \item \textbf{OBJ-2.} Rutinas de procesamiento de imágenes, encargadas de detectar y filtrar objetos celestes con características estelares.
  \item \textbf{OBJ-3.} Extraer distintas medidas y heurísticas para evaluar el nivel de enfoque alcanzado en las distintas imágenes estelares, basándonos en el calculo FWHM.
  \item \textbf{OBJ-4.} Implementar Driver Indi \textbf{Driver INDI} que provea de la funcionalidad básica para controlar todas las características del dispositivo enfocador.
  \item \textbf{OBJ-5.} Implementar algoritmo de autofocus, basándonos en la medida del foco calculada y coordinada con las funciones del dispositivo Ardufocuser y las imagenes capturadas por el CCD.
\end{itemize}

\bigskip
Además de los objetivos principales, se persigue alcanzar los siguientes objetivos secundarios:

\begin{itemize}
  \item \textbf{OBJ-S-1.} Completar una buena documentación técnicas, así como manuales para desarrolladores, a fin de que cualquier interesado sea capaz de reproducir fácilmente y a bajo costo el proyecto.
  \item \textbf{OBJ-S-2.} Publicar código fuente, diseños y documentación bajo una licencia libre, que permita que la comunidad de desarrolladores interesados pueda realizar modificaciones y personalizaciones, siempre que se mantenga la misma licencia.
  \item \textbf{OBJ-S-3.} Difusión en la comunidad astronómica y desarrollo de una web propia del proyecto.

\end{itemize}

\bigskip
Para la realización de los objetivos se pondrán en practica los conocimientos alcanzados en:

\begin{itemize}
  \item \textbf{Ingeniería del software} para el análisis y diseño del proyecto, así como modelar el sistema.
  \item \textbf{Programación orientada a objetos} para la estructura y la organización del código \textbf{Java}.
  \item \textbf{Programación de sistemas multimedia} para poder implementar las interfaces de usuario en \textbf{Java Swing}, así como visualizar y tratar las imagenes.
  \item \textbf{Infraestructura virtual} para poder gestionar los sistemas, teniendo habilidad para realizar instalaciones y aprovisionamiento del servidor.
  \item \textbf{Transmisión de datos y redes de computadores} para comprender el comportamiento del protocolo \textbf{INDI} y configurar correctamente las redes para las pruebas.
  \item \textbf{Algorítmica} para optimizar las rutinas de tratamiento de imagenes.
  \item \textbf{Calculo matemático} para modelar los objetos celestes como una función gaussiana que representa la luminosidad, así como el calculo del FWHM.
  \item \textbf{Estructura de datos} dado que contamos con representaciones de las imagenes, que debemos conocer y saber manejar.


\end{itemize}

\bigskip
Por otro lado, han sido necesarios alcanzar conocimientos en otras áreas:

\begin{itemize}

  \item \textbf{Astronomía básica y equipos astronómicos} para entender a los usuarios potenciales y poder acomodar la aplicación a sus necesidades.
  \item \textbf{Raspberry Pi}\footnote{Ordenador de placa reducida y única de bajo coste.} para montar un servidor permanente de pruebas o acceso público para probar la aplicación
  \item \textbf{Latex}\footnote{Sistema de composición de textos.} para la realización del presente documento y la ampliación de conocimientos para futuros textos científicos.
  \item \textbf{Git} para la gestión de versiones y la publicación de código abierto que permita a otros desarrolladores participar.
\end{itemize}
