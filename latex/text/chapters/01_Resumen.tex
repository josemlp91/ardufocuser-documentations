\chapter{Resumen}

\section{Breve resumen y palabras clave}
\noindent{\textbf{Palabras clave}: \textit{arduino}, \textit{hardware libre}, \textit{astronomía}, \textit{procesamiento imagen}, \textit{}, \textit{INDI}, \textit{software libre}.\\

\bigskip
El objetivo principal de este proyecto es diseñar e implementar un sistema completo de enfoque automático, que se pueda acoplable directamente
 a un telescopio comercial y consiga de forma óptima realizar una configuración de las lentes para conseguir el mejor enfoque posible de las imágenes estelares.

\bigskip
Dado la gran historia y recorrido de la astronomía, podemos apreciar una gran evolución en las herramientas astronómicas,
que han ido apareciendo sirviéndose de los últimos avances en las demás ciencias como puede ser la óptica, las matemáticas, la mecánica, la electrónica y la informática.

\bigskip
Hoy día ya contamos con numerosas plataformas de control y automatización de dispositivos astronómicos, aunque en muchos casos se relacionan con grandes compañías, que no ofrecen detalles técnicos.

\bigskip
El reto al que nos enfrentamos en este proyecto es acercar el hardware y el software libre a la astronomía para crear herramientas avanzadas con un bajo presupuesto y alcance libre a todas sus interioridades y detalles técnicos.


\newpage
\begin{center}
{\LARGE\bfseries\tituloEng}\\
\end{center}
\begin{center}
\autor\
\end{center}

\section{Extended abstract and key words}

\noindent{\textbf{Palabras clave}: \textit{arduino}, \textit{hardware libre}, \textit{astronomía}, \textit{procesamiento imagen}, \textit{}, \textit{INDI}, \textit{software libre}.\\


\bigskip
The main objective of this project is to design and implement a complete autofocus system that can be coupled directly to a commercial telescope and get optimally perform a configuration of lenses, to observe stellar images.
