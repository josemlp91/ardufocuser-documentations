\documentclass[a4paper,11pt]{book}
\usepackage{listings}
\usepackage[utf8]{inputenc}
\usepackage{titlesec}
\usepackage{fancyhdr}
\usepackage[spanish,es-tabla,es-nodecimaldot]{babel}
\usepackage[hidelinks]{hyperref}
\usepackage{xcolor}
\usepackage{pdfpages}
\usepackage{eurosym}
\usepackage{graphicx}
\usepackage{caption}
\usepackage{subcaption}
\usepackage{parskip}
\usepackage{gensymb}

%\usepackage{natbib}
\usepackage[toc,style=altlistgroup,hyperfirst=false]{glossaries}

% Información reutilizable
\newcommand{\asunto}{Trabajo de Fin de Grado}
\newcommand{\titulo}{Ardufocuser: Una Solución Libre para Enfoque Astronómico}
\newcommand{\tituloEng}{Ardufocuser: An Automatic Astronomical Focusing Solution}
\newcommand{\grado}{Grado en Ingeniería Informática}
\newcommand{\autor}{José Miguel López Pérez}
\newcommand{\dni}{15516308W}
\newcommand{\email}{josmilope@gmail.com}
\newcommand{\tutor}{Dr. Sergio Alonso Burgos}
\newcommand{\escuela}{Escuela Técnica Superior de Ingenierías Informática y de Telecomunicación}
\newcommand{\universidad}{Universidad de Granada}
\newcommand{\ciudad}{Granada}
\newcommand{\vers}{Versión 1.0}

%Comandos personalizados
\newcommand{\grad}{$^{\circ}$}

% Información archivo
\hypersetup{
	pdfauthor = {\autor\ (\email)},
	pdftitle = {\titulo},
	pdfsubject = {\asunto},
	pdfkeywords = {Arduino, hardware libre, astronomía, robotización de telescopios, INDI, software libre, driver, internet of things, procesamiento imagen},
	pdfcreator = {LaTeX con el paquete texlive},
	pdfproducer = {pdflatex}
}



% Estilo de cabeceras
\pagestyle{fancy}
\fancyhf{}
\fancyhead[LO]{\leftmark}
\fancyhead[RE]{\rightmark}
\fancyhead[RO,LE]{\textbf{\thepage}}
\setlength{\headheight}{1.5\headheight}

% Redefinición de comandos
\renewcommand{\lstlistingname}{Fragmento de Código}
\renewcommand{\lstlistlistingname}{Índice de Fragmentos de Código}
\renewcommand{\chaptermark}[1]{\markboth{\textbf{#1}}{}}
\renewcommand{\sectionmark}[1]{\markright{\textbf{\thesection. #1}}}

% Definición de colores
\definecolor{gray97}{gray}{.97}
\definecolor{gray75}{gray}{.75}
\definecolor{gray45}{gray}{.45}
\definecolor{gray30}{gray}{.94}
\definecolor{lightgray}{rgb}{.9,.9,.9}
\definecolor{darkgray}{rgb}{.4,.4,.4}
\definecolor{purple}{rgb}{0.65, 0.12, 0.82}
\definecolor{background}{HTML}{EEEEEE}
\definecolor{delim}{RGB}{20,105,176}
\colorlet{punct}{red!60!black}
\colorlet{numb}{magenta!60!black}

\definecolor{dkgreen}{rgb}{0,0.6,0}
\definecolor{dred}{rgb}{0.545,0,0}
\definecolor{dblue}{rgb}{0,0,0.545}
\definecolor{lgrey}{rgb}{0.9,0.9,0.9}
\definecolor{gray}{rgb}{0.4,0.4,0.4}
\definecolor{darkblue}{rgb}{0.0,0.0,0.6}

% Listados
\lstset{
	aboveskip=0.5cm,
	backgroundcolor=\color{gray97},
	basicstyle=\scriptsize\ttfamily,
	breaklines=true,
	commentstyle=\color{gray45},
	frame=Ltb,
	framerule=0.5pt,
	framesep=0pt,
	framexbottommargin=0.1pt,
	framexleftmargin=0.1cm,
	framextopmargin=0.5pt,
	keywordstyle=\bfseries,
	numberfirstline = false,
	numbers=left,
	numbersep=5pt,
	numberstyle=\tiny,
	rulesep=.4pt,
	rulesepcolor=\color{black},
	showstringspaces = false,
	stringstyle=\ttfamily,
	literate={á}{{\'a}}1
	         {é}{{\'e}}1
	         {í}{{\'i}}1
	         {ó}{{\'o}}1
	         {ú}{{\'u}}1
	         {ñ}{{\~n}}1
}

% Minimizar fragmentado de listados
\lstnewenvironment{listing}[1][]
	{\lstset{#1}\pagebreak[0]}{\pagebreak[0]}



	
	

\usepackage{xcolor}
\definecolor{dkgreen}{rgb}{0,0.6,0}
\definecolor{dred}{rgb}{0.545,0,0}
\definecolor{dblue}{rgb}{0,0,0.545}
\definecolor{lgrey}{rgb}{0.9,0.9,0.9}
\definecolor{gray}{rgb}{0.4,0.4,0.4}
\definecolor{darkblue}{rgb}{0.0,0.0,0.6}

% Listado definido para JavaScript
% http://tex.stackexchange.com/questions/89574/language-option-supported-in-listings/89576#89576
\lstdefinelanguage{javascript}{
	backgroundcolor=\color{background},
	basicstyle=\footnotesize,
	breaklines=true,
	captionpos=b,
	comment=[l]{//},
	commentstyle=\color{purple}\ttfamily,
	frame=lines,
	identifierstyle=\color{black},
	keywordstyle=\color{blue}\bfseries,
	morecomment=[s]{/*}{*/},
	morestring=[b]',
	morestring=[b]",
	ndkeywordstyle=\color{darkgray}\bfseries,
	numbers=left,
	numbersep=8pt,
	numberstyle=\scriptsize,
	sensitive=false,
	showstringspaces=false,
	stepnumber=1,
	stringstyle=\color{red}\ttfamily,
	keywords={
		break,
		case,
		catch,
		catch,
		do,
		else,
		false,
		function,
		if,
		in,
		new,
		null,
		return,
		switch,
		true,
		typeof,
		var,
		while},
	ndkeywords={
		boolean,
		class,
		export,
		implements,
		import,
		this,
		throw}
}



\lstdefinelanguage{cpp}{
	backgroundcolor=\color{lgrey},  
	basicstyle=\footnotesize \ttfamily \color{black} \bfseries,   
	breakatwhitespace=false,       
	breaklines=true,               
	captionpos=b,                   
	commentstyle=\color{dkgreen},   
	deletekeywords={...},          
	escapeinside={\%*}{*)},                  
	frame=single,                  
	language=C++,                
	keywordstyle=\color{purple},  
	morekeywords={BRIEFDescriptorConfig,string,TiXmlNode,DetectorDescriptorConfigContainer,istringstream,cerr,exit}, 
	identifierstyle=\color{black},
	stringstyle=\color{blue},      
	numbers=right,                 
	numbersep=5pt,                  
	numberstyle=\tiny\color{black}, 
	rulecolor=\color{black},        
	showspaces=false,               
	showstringspaces=false,        
	showtabs=false,                
	stepnumber=1,                   
	tabsize=5,                     
	title=\lstname,                 
}


% Para que las páginas en blanco no tengan cabecera
\makeatletter
\def\clearpage{%
  \ifvmode
    \ifnum \@dbltopnum =\m@ne
      \ifdim \pagetotal <\topskip
        \hbox{}
      \fi
    \fi
  \fi
  \newpage
  \thispagestyle{empty}
  \write\m@ne{}
  \vbox{}
  \penalty -\@Mi
}
\makeatother

\begin{document}
 \input{front/front}
 \frontmatter
 \begin{center}
{\LARGE\bfseries\titulo}\\
\end{center}
\begin{center}
\autor\
\end{center}

\noindent\rule[-1ex]{\textwidth}{2pt}\\[4.5ex]
\noindent{\textbf{Declaración de Originalidad del TFG}}\\

D. \textbf{\autor}, con DNI \dni, declara que el presente Trabajo de Fin de Grado es original, no habiéndose utilizado fuentes sin ser citadas debidamente. De no cumplir con este compromiso, soy consciente de que, de acuerdo con la  \href{http://secretariageneral.ugr.es/bougr/pages/bougr71/ncg712/}{Normativa de Evaluación y de Calificación de los estudiantes de la Universidad de Granada de 20 de mayo de 2013}, esto conllevará automáticamente la calificación numérica de cero independientemente del resto de las calificaciones que el estudiante hubiera obtenido. Esta consecuencia debe entenderse sin perjuicio de las responsabilidades disciplinarias en las que pudieran incurrir los estudiantes que plagie.

\bigskip
\noindent Para que así conste lo firmo el  \today\\

\vspace{1cm}
\begin{center}
\includegraphics[width=0.45\textwidth]{../images/firmaJM.jpg}\\

Firma del alumno
\end{center}


\newpage
\begin{center}
{\LARGE\bfseries\titulo}\\
\end{center}
\begin{center}
\autor\
\end{center}
\noindent\rule[-1ex]{\textwidth}{2pt}\\[4.5ex]

Yo, \textbf{\autor}, alumno de \grado ~ de la \textbf{Escuela Técnica Superior de Ingenierías Informática y de Telecomunicación de la Universidad de Granada}, con DNI \dni, autorizo la ubicación de la siguiente copia de mi Trabajo Fin de Grado en la biblioteca del centro para que pueda ser consultada por las personas que lo deseen.

\bigskip
Además, este mismo trabajo es realizado bajo licencia \href{https://creativecommons.org/licenses/by-sa/4.0/}{\textbf{Creative Commons Attribution-ShareAlike 4.0}}, dando permiso para copiarlo y redistribuirlo en cualquier medio o formato, también de adaptarlo de la forma que se quiera, pero todo esto siempre y cuando se reconozca la autoría y se distribuya con la misma licencia que el trabajo original. 


\vspace{3cm}

\begin{center}
\includegraphics[width=0.45\textwidth]{../images/firmaJM.jpg}\\

Firma del alumno\\

\ciudad ~ a \today\\
\end{center}





\newpage
\begin{center}
{\LARGE\bfseries\titulo}\\
\end{center}
\begin{center}
\autor\
\end{center}
\noindent\rule[-1ex]{\textwidth}{2pt}\\[4.5ex]


El \textbf{\tutor}, profesor del \textbf{Departamento de Lenguajes y Sistemas Informáticos} de la \textbf{\universidad}.

\vspace{0.5cm}

\vspace{0.5cm}

\textbf{Informa:}

\vspace{0.5cm}

Que el presente trabajo, titulado \textit{\textbf{\titulo}},
ha sido realizado bajo su supervisión por D. \textbf{\autor}, y autoriza la defensa de dicho trabajo ante el tribunal
que corresponda.

\vspace{0.5cm}

Y para que conste, expido y firmo el presente informe en \ciudad ~ a \today\\.

\vspace{2cm}

\vspace{1cm}
\begin{center}
\includegraphics[width=0.35\textwidth]{../images/firmaZerjillo3.png}\\

Firma del tutor
\end{center}


\chapter*{Agradecimientos}
\thispagestyle{empty}



A mis padres por la cantidad de oportunidades que me han brindado con su esfuerzo, así como su cariño, en los buenos y malos momentos. Sin olvidarme de mi hermana siempre sacándome una sonrisa y mis abuelos siempre presentes. 

\bigskip
Agradecer a mi tutor \tutor ~ todo lo que he aprendido gracias a su labor en el presente proyecto. Transmitiendo su pasión y teniendo mucha paciencia.

\bigskip
Así como agradecer a mis compañeros de piso Javier, Marta, Antonio, David y Jorge por hacerme sentir en familia durante tanto tiempo. Sin olvidarme de mis compañero de clase Jaime, Oscar, Roge, Mario y Migue por hacer tantas horas de estudio más llevaderas. 

\bigskip
Por último y no menos importante, dar las gracias a mis compañeros de trabajo, en especial a Diego, Francisco y Brian, por saber comprenderme y ayudar en todo lo posible. 


\vspace{6cm}
\begin{center}
	\begin{minipage}{0.9\linewidth}
		\vspace{5pt}%margen superior de minipage
		{\textit{
			Somos polvo de estrellas que piensa acerca de las estrellas.
		}}
		\begin{flushright}
			  \small - Carl Edward Sagan
		\end{flushright}
		\vspace{5pt}%margen inferior de la minipage
	\end{minipage}
\end{center}








 \begingroup
 \let\cleardoublepage\clearpage
  \tableofcontents
  \listoffigures
  \listoftables
  \lstlistoflistings
 \endgroup


 \thispagestyle{empty}
 \
 \mainmatter
\chapter{Resumen}

\section{Breve resumen y palabras clave}
\noindent{\textbf{Palabras clave}: \textit{arduino}, \textit{hardware libre}, \textit{astronomía}, \textit{procesamiento imagen}, \textit{}, \textit{INDI}, \textit{software libre}.\\

\bigskip
El objetivo principal de este proyecto es diseñar e implementar un sistema completo de enfoque automático, que se pueda acoplable directamente
 a un telescopio comercial y consiga de forma óptima realizar una configuración de las lentes para conseguir el mejor enfoque posible de las imágenes estelares.

\bigskip
Dado la gran historia y recorrido de la astronomía, podemos apreciar una gran evolución en las herramientas astronómicas,
que han ido apareciendo sirviéndose de los últimos avances en las demás ciencias como puede ser la óptica, las matemáticas, la mecánica, la electrónica y la informática.

\bigskip
Hoy día ya contamos con numerosas plataformas de control y automatización de dispositivos astronómicos, aunque en muchos casos se relacionan con grandes compañías, que no ofrecen detalles técnicos.

\bigskip
El reto al que nos enfrentamos en este proyecto es acercar el hardware y el software libre a la astronomía para crear herramientas avanzadas con un bajo presupuesto y alcance libre a todas sus interioridades y detalles técnicos.


\newpage
\begin{center}
{\LARGE\bfseries\tituloEng}\\
\end{center}
\begin{center}
\autor\
\end{center}

\section{Extended abstract and key words}

\noindent{\textbf{Palabras clave}: \textit{arduino}, \textit{hardware libre}, \textit{astronomía}, \textit{procesamiento imagen}, \textit{}, \textit{INDI}, \textit{software libre}.\\


\bigskip
The main objective of this project is to design and implement a complete autofocus system that can be coupled directly to a commercial telescope and get optimally perform a configuration of lenses, to observe stellar images.

\chapter{Introducción y motivación}

<<<<<<< HEAD

Toda observación de cualquier fenómeno astronómico, requiere de múltiples trabajos laboriosos y operaciones de control, para cada componente del observatorio. Ya sea la montura del telescopio (sección~\ref{montura}), la cámara CCD (sección~\ref{ccd}), el enfocador (sección~\ref{enfocadores}), la rueda de filtros (sección~\ref{filtros}) etc., asumiendo una pérdida de tiempo útil y el riesgo que se puedan producir errores humanos o accidentes, que pongan en peligro la noche de observación.


Muchas de las operaciones siguen patrones claros y tienen relación con el estado de un sensor o periférico, permitiendo definir rutinas automáticas que agilicen los trabajos inherentes a la observación. 

Un ejemplo claro puede ser la decisión de abrir o cerrar la cúpula en un momento dado en función de los datos proporcionados por la estación meteorológica (sección~\ref{estacion_meteorologica}).


En el mercado existen ya gran cantidad de recursos que se encargan de automatizar estas tareas, sin embargo, el acceso a estos recursos es caro y no todo aficionado a la astronomía puede permitírselo.  

Otro punto en contra de estas soluciones, es que suelen ser muy cerradas. Las compañías ofrecen muy pocos detalles técnicos (los justos para poder instalarlos e integrarlos), no permiten ninguna modificación y deben funcionar exactamente bajo la plataformas definidas, en la mayoría de los casos con el sistema operativo Windows. 

El presente proyecto se puede englobar dentro de las soluciones ``Internet of Things'' que hoy día están en una de las fases de expectativas más alta, como podemos observar en el \textit{Hype Cycle for Emerging Technologies} de Gartner \cite{hypeGartner} (figura~\ref{fig:gartner}).


\begin{figure}[h]
\includegraphics[width=1\linewidth]{../images/emergingGartner}
\caption[Hype Cycle for Emerging Technologies]{Hype Cycle for Emerging Technologies, 2015. \textbf{Fuente:}~\cite{hypeGartner}}
\label{fig:gartner}
\end{figure}


Para poner en contexto el proyecto, en las siguientes secciones se propone un recorrido por algunas de las ciencias, tecnologías y disciplinas que están involucradas en el desarrollo del presente TFG:

=======
\bigskip
Toda observación de cualquier fenómeno astronómico requiere de múltiples trabajos laboriosos y operaciones de control para cada componente del observatorio, ya sea la montura del telescopio, la cámara CCD, el enfocador, la rueda de filtros etc., asumiendo una pérdida de tiempo útil y un riesgo que se puedan producir errores humanos o accidentes, que pongan en peligro la noche de observación.

\bigskip
Muchas de las operaciones siguen patrones claros, y tienen relación con el estado de un sensor o periférico, por tanto podemos definir rutinas automáticas, 
que agilicen los trabajos inherentes a la observación. 

\bigskip
Un ejemplo claro puede ser la decisión de abrir o cerrar la cúpula en un momento dado en función de los datos proporcionados por la estación meteorológica, o la decisión de corregir el foco según el perfil de brillo obtenido para las estrellas presentes en las imágenes tomadas por la cámara CCD.

\bigskip
En el mercado existen ya gran cantidad de recursos que se encargan de automatizar las tareas antes citadas, sin embargo, el acceso a estos recursos es caro y no todo aficionado a la astronomía puede permitírselo.  Además el nivel de complejidad de algunos modelos complican su manejo y por tanto el disfrute en la observación.
Otro punto en contra de estas soluciones es que suelen ser muy cerradas, las compañías ofrecen muy pocos detalles tecnicos (los justos para poder instalarlos e integrarlos), no permiten ninguna modificación y deben funcionar exactamente bajo la plataformas definidas, en la mayoría de los cosas Windows. 

\newpage
\bigskip
Para poner en contexto el proyecto, se  propone un recorrido por algunos de las ciencias, tecnologías y disciplinas que están involucradas en el desarrollo del presente TFG.
>>>>>>> c9f08dfe66521d4f0dba18e652f93a6a37a333aa

\begin{itemize}

  \item {Astronomía}
<<<<<<< HEAD
%    \begin{itemize}
%    	\setlength\itemsep{0em}
% 	   \item{Clásica}
% 	   \item{Moderna}
% 	   \item{Actual}	 
% 	\end{itemize}

  \item {Instrumental astronómico}
%   \begin{itemize}
%   	\setlength\itemsep{0em}	
%     \item{Telescopios}
%     \item{Monturas robotizadas}
%     \item{Enfocadores}
%     \item{Cámara y CCD}
%     \item{Rueda portafiltros}
%     \item{Cúpulas y estaciones meteorológicas}
%       
%  \end{itemize}
  \item {Software astronómico}
  \item {Formato imágenes astronómicas}
  
  \item {Control remoto de observatorios}
%   \begin{itemize}
%   	\setlength\itemsep{0em}
%    	\item{ASCOM}
%    	\item{INDI}
%    	\item{Clientes remotos}
%   \end{itemize}
  
  \item {Hardware libre}
%    \begin{itemize}
%    	\setlength\itemsep{0em}
%      \item{Arduino}
%      \item{Raspberry Pi}
%    \end{itemize}
   
   \item {Enfocadores astronómicos: estado del arte}
=======
   \begin{itemize}
	   \item{Clásica}
	   \item{Moderan}
	   \item{Actual}	 
	\end{itemize}

  \item {Instrumental astronómico}
  \begin{itemize}	
    \item{Telescopios}
    \item{Monturas Robotizadas}
    \item{Enfocadores}
    \item{Cámara y CCD}
    \item{Rueda Portafiltros}
    \item{Cúpulas y estaciones meteorológicas}
      
  \end{itemize}
  
  \item {Hardware libre}
   \begin{itemize}
     \item{Arduino}
     \item{Raspberry Pi}
   \end{itemize}
   
  \item {Observatorios remotos}
   \begin{itemize}
     \item{ASCOM}
     \item{INDI}
     \item{Clientes remotos}
  \end{itemize}
  
  \item{Formato Imágenes}
  \item {Software procesamiento imagenes astronómicas}
>>>>>>> c9f08dfe66521d4f0dba18e652f93a6a37a333aa

\end{itemize}


\begin{figure}[h]
<<<<<<< HEAD
\includegraphics[width=1\linewidth]{../images/observatorio_amateur}
\caption[Observatorio Amateur]{\textbf{Astrónomo amateur} en plena observación, utilizando telescopio y prismáticos. \textbf{Fuente:} \cite{astrociencia}}
\label{fig:observatorio_amateur}
\end{figure}

=======
\centering
\includegraphics[width=0.7\linewidth]{../images/observatorio_amateur}
\caption[Observatorio Amateur]{\href{http://astrocienciasecu.blogspot.com.es/}{Observatorio Amateur}}
\label{fig:observatorio_amateur}
\end{figure}
\newpage
>>>>>>> c9f08dfe66521d4f0dba18e652f93a6a37a333aa

\section{La Astronomía}


Significa literalmente el estudio de las \textbf{leyes} que rigen los \textbf{astros} o cuerpos celestes, según su etimología griega y latina.

<<<<<<< HEAD

Se define más formalmente como \cite{astronomia}:

\begin{quote}``\textit{Ciencia que se ocupa del estudio de los cuerpos celestes del universo, incluidos los planetas y sus satélites, los cometas y meteoritos, las estrellas y la materia interestelar,
  los sistemas de materia oscura, estrellas, gas y polvo llamados galaxias y los cúmulos de galaxias; por lo que estudia sus movimientos y fenómenos ligados a ellos.}''
\end{quote}


Es una de las ciencia más remota por su impacto visual, emocional y su gran utilidad en la agricultura. Captó la atención de nuestros antepasados, motivándolos al estudio de los objetos del firmamento y su movimiento, fenómeno que muchas veces no conseguían explicar por completo, y por eso los llegaban a divinizar en múltiples culturas.


Por tanto la  astronomía es una ciencia antigua y moderna a la vez: Antigua porque se remonta prácticamente al origen de la humanidad; Moderna por proporcionar uno de los campos de estudio e investigación más avanzados.


Se trata de una ciencia donde aún perduran muchos interrogantes como por ejemplo, si existe vida fuera de la Tierra o si somos la única civilización inteligente \cite{astrociencia}.
=======
\bigskip
Se define más formalmente como:

\begin{quote}``\textit{Ciencia que se ocupa del estudio de los cuerpos celestes del universo, incluidos los planetas y sus satélites, los cometas y meteoritos, las estrellas y la materia interestelar,
  los sistemas de materia oscura, estrellas, gas y polvo llamados galaxias y los cúmulos de galaxias; por lo que estudia sus movimientos y fenómenos ligados a ellos.}''
\newline(\href{https://es.wikipedia.org/wiki/Astronom%C3%ADa}{Astronomia})
\end{quote}

\bigskip
Se puede afirmar que fue la ciencia más remota por su impacto visual, emocional y gran utilidad en la agricultura, por ello captó la atención de nuestros antepasados, motivandolos al estudio de los objetos suspendidos en el firmamento, fenómenos que muchas veces no conseguían explicar por completo, y por eso los llegaban a divinizar en múltiples culturas.

\bigskip
Por tanto la  astronomía es una ciencia antigua y moderna a la vez. \newline
Antigua porque se remonta prácticamente al origen de la humanidad.\newline
Moderna por proporcionar uno de los campos de estudio e investigación más avanzados.

\bigskip
Se trata por tanto de una ciencia cargada de múltiples misterios que a día de hoy se siguen manteniéndose, con muchos interrogante, como por ejemplo,  si existe vida fuera de la Tierra, o si somos la única civilización
inteligente.


\begin{figure}[b]
\centering
\includegraphics[width=0.9\linewidth]{../images/astrofooter}
\caption[Galaxia M88]{\href{https://commons.wikimedia.org/wiki/File:Messier_88_galaxy.jpg}{Galaxia M88}}
\label{fig:astrofooter}
\end{figure}
>>>>>>> c9f08dfe66521d4f0dba18e652f93a6a37a333aa


\subsection{Astronomía Antigua}

<<<<<<< HEAD
=======
\bigskip
Civilizaciones tan antiguas como la asiria, babilónica o  sumeria, ya comenzaron a transmitirnos los primeros conocimientos sobre el universo que conocemos gracias a la difusión realizada por la cultura griega.
>>>>>>> c9f08dfe66521d4f0dba18e652f93a6a37a333aa

Civilizaciones tan antiguas como la asiria, babilónica o sumeria, ya comenzaron a transmitirnos los primeros conocimientos sobre el universo que conocemos gracias a la difusión realizada por la cultura griega \cite{historiaAstronomia}.

<<<<<<< HEAD

Tienen especial importancia los conocimientos astronómicos de los egipcios, dado que para ellos el estudio del cielo y elaborar su \textbf{calendario egipcio} \cite{calendario_egipcio} les era de vital importancia, porque les permitía controlar los ciclos de la agricultura y prever la gran inundación del río Nilo, que sigue ciclos anuales.


En el Nuevo Mundo, los mayas llegaron a alcanzar importantes conocimientos de los cuerpos celestes y elaborando un calendario bastante preciso. 


Los incas se consideraban a sí mismos descendientes del Sol y los aztecas adoraban al dios \textbf{Huitzilopochtli}, símbolo del Sol que amanecía cada mañana para hacer la lucha con sus hermanas, las estrellas y así imponer su reinado diurno.


En el  siglo III a.C, el astrónomo griego \textbf{Aristarco de Samos}~\cite{Arist}, puso en duda todo el modelo geocéntrico griego y postuló que la Tierra gira en 24 horas y se traslada en torno al Sol en un año. Realizando también dibujos de las órbitas planetarias en el orden que ahora las conocemos.


\textbf{Pitágoras}~\cite{Pitagoras} en el siglo VI  a.C. ya tenia ideas sobre los movimientos de rotación  Terrestre y de traslación en torno al Sol, así como conocimiento de la esfericidad de la Tierra, Luna y Sol~\cite{AstroAnti}.

=======
\bigskip
Teniendo especial importancia los conocimientos astronómicos de los egipcios, dado para ellos el estudio del cielo y elaborar su 
(\href{https://es.wikipedia.org/wiki/Calendario_egipcio}{\textbf{calendario egipcio}} que les era de vital importancia, dado que permitía controlar los ciclos de la agricultura y prever la gran inundación del río Nilo, que sigue ciclos anuales.

\bigskip
En el Nuevo Mundo, los mayas llegaron a alcanzar importantes conocimientos de los cuerpos celestes y elaborando un calendario bastante preciso. 
\newline

\bigskip
Los incas se consideraban a sí mismos descendientes del Sol y los aztecas adoraban al dios \textbf{Huitzilopochtli}, símbolo del Sol
que amanecía cada mañana para hacer la lucha con sus hermanas, las estrellas y así imponer su reinado diurno.
\newline

\bigskip
Se presume que ya en el  siglo III a,C, el astrónomo griego \textbf{Aristarco de Samos} ~\cite{Arist} puso en duda todo el modelo geocéntrico griego y postuló que la Tierra gira en 24 horas y se traslada en torno al Sol en un año. Realizando también dibujos de las órbitas planetarias en el orden que ahora las conocemos.

\bigskip
\textbf{Pitágoras} ~\cite{Pitagoras} en el siglo VI  a.C. ya tenia ideas sobre los movimientos de rotación  Terrestre en torno a su eje, de traslación en torno al Sol y conocimiento de la esfericidad de la Tierra, Luna y So.  ~\cite{AstroAnti}.
>>>>>>> c9f08dfe66521d4f0dba18e652f93a6a37a333aa


\subsection{Astronomía Moderna}

<<<<<<< HEAD
=======
\bigskip
Se dice que la Astronomía moderna\cite{AstMod} inicia su desarrollo con Nicolás Copérnico (1473-1543)  quien el año de su muerte publica un trabajo de importancia capital, \textbf{De revolutionibus orbium caelestium}\cite{Copernico}.
>>>>>>> c9f08dfe66521d4f0dba18e652f93a6a37a333aa

La Astronomía moderna \cite{AstMod} inicia su desarrollo con \textbf{Nicolás Copérnico (1473-1543)} \cite{Copernico}, quien el año de su muerte publica su trabajo más importante \textbf{De revolutionibus orbium caelestium}.

<<<<<<< HEAD

Se mantiene que la Tierra tiene un doble movimiento: de rotación sobre ella misma, en 24 horas, y de revolución alrededor del Sol, en un año. También establece movimientos similares para los planetas y satélites.


Una figura muy importante es \textbf{Tycho Brahe (1546-1601)} \cite{Tycho} con  sus observaciones, facilitó a su discípulo \textbf{Johannes Kepler (1571-1630)} \cite{Kepler}, el descubrimiento de las famosas leyes que rigen el movimiento de los planetas, el abandono de las órbitas circulares y la ruptura definitiva con unos conceptos tradicionales que estaban profundamente arraigados. 



\textbf{Galileo Galilei (1564, 1642)} \cite{galileo} fue un astrónomo, filósofo, ingeniero, matemático y físico italiano. En 1609 construye su primer telescopio, motivado por el rumor de la existencia de un telescopio fabricado en Holanda. Su telescopio no deforma los objetos y es capaz de aumentarlos 6 veces. Tras muchas versiones consigue aumentar los objetos hasta 20 veces. Con su instrumento realiza numerosos descubrimientos de la morfología de la Luna y del universo en general.  
 


=======
\bigskip
Una aportación fundamental en el desarrollo de la nueva es debida a Tycho Brahe (1546-1601)\cite{Tycho} teniendo gran importancia sus observaciones, y  sentando las bases que facilitarían a su discípulo Johannes Kepler (1571-1630)\cite{Kepler}, el descubrimiento de las famosas leyes que rigen el movimiento, el abandono de las órbitas circulares y la ruptura definitiva con unos conceptos tradicionales que estaban profundamente arraigados. 

\bigskip
La publicación de los Principia en 1685 por Isaac Newton (1643-1727)\cite{Newton} marca uno de los puntos culminantes de la ciencia moderna, las leyes de Kepler quedan incluidas en un sistema físico que explica una serie de fenómenos naturales como las estaciones del año, las mareas, los movimientos de los astros, mediante un conjunto consistente de leyes de carácter general que podían ser probadas en un laboratorio.

\bigskip
En este punto Astronomía y Astrología inician caminos diferentes y desde entonces no tienen ningún punto común.

\bigskip
Mientras que la primera busca una explicación mecanicista de los fenómenos naturales aplicando leyes formuladas consistentemente y controladas en laboratorio, la Astrología tiene como objetivos la realización de predicciones sobre la personalidad de los individuos y de los sucesos, basándose en las posiciones relativas de los astros.\cite{astrologia}

\bigskip
Durante el siglo XVIII tienen lugar aportaciones importantes en el campo de la astronomía observacional que constituyeron la base observacional para el estudio del Universo a gran escala. Ch. Messier, presentó en la Academia de Ciencias de Francia en 1771, el primer catálogo de y asociaciones de cúmulos estelares, descubiertas u observadas por él\cite{messier}.

\bigskip
El descubrimiento de la fotografía y el progreso en la elaboración de emulsiones fotográficas, produjo un rápido avance en la aplicación de la astronomía. En 1863 Huggins obtiene los primeros espectros estelares ~\cite{analisispectal} abriendo una nueva era en la Astronomía\cite{huggins}.

\bigskip
Desde finales del siglo XIX y principios del XX la Física pasa a desempeñar un papel decisivo en la interpretación de los fenómenos astronómicos. La Astrofísica\cite{astrofisica} adquiere una progresiva importancia sobre la astronomía clásica, utilizando en la actualidad ambos términos de forma sinónima.
>>>>>>> c9f08dfe66521d4f0dba18e652f93a6a37a333aa

\textbf{Isaac Newton (1643-1727)} \cite{Newton} publica \textbf{los Principia} en 1685 \cite{los_principia} y explica una serie de fenómenos naturales como las estaciones del año, las mareas, los movimientos de los astros, mediante un conjunto de leyes que podían ser probadas en un laboratorio. En este punto la \textbf{Astronomía} \cite{astronomia} y \textbf{Astrología} \cite{astrologia} inician caminos diferentes y desde entonces no tienen ningún punto en común.


<<<<<<< HEAD
Durante el siglo XVIII tienen lugar aportaciones importantes en el campo de la astronomía observacional para el estudio del Universo. \textbf{Charles Messier (1730-1817)} \cite{messier}, presentó en la Academia de Ciencias de Francia en 1771 el primer catálogo de estrellas y asociaciones de cúmulos estelares, descubiertas u observadas por él.


El descubrimiento de la \textbf{fotografía} produjo un rápido avance en la aplicación de la astronomía. En 1863 Huggins \cite{huggins}, obtiene los primeros espectros estelares ~\cite{analisispectal} abriendo una nueva era en la Astronomía.


Desde finales del siglo XIX y principios del XX la Física pasa a desempeñar un papel decisivo en la interpretación de los fenómenos astronómicos. La Astrofísica~\cite{astrofisica}, adquiere una progresiva importancia sobre la astronomía clásica, utilizando en la actualidad ambos términos de forma casi sinónima.


\begin{figure}[b]
	\centering
	\includegraphics[width=1\linewidth]{../images/astrofooter}
	\caption[Galaxia M88]{Galaxia en espiral \textbf{M88}, a 49 millones de años luz, descubierta por \textbf{Charles Messier} en 1781. \textbf{Fuente:} \cite{M88}}
	\label{fig:astrofooter}
\end{figure}


\subsection{Astronomía en la actualidad}

Viendo el recorrido de esta ciencia en el pasado, nos preguntamos por su repercusión en la sociedad presente. Poniendo en valor los beneficios que tiene su estudio y los interrogantes que a día de hoy científicos de todo el mundo trabajan por dar respuesta \cite{beneficiosastro}.

Los beneficios que obtiene nuestra sociedad, los podemos catalogar según su alcance:

\begin{itemize}
	\item \textbf{Puramente científico}, el impulso por el conocimiento. Responder interrogantes existenciales como: ``¿De dónde venimos?. ¿A dónde vamos?”.


	\item \textbf{La expansión del Universo}, el \textbf{Big Bang}, \textbf{Agujeros Negros}. Conceptos cosmológicos complejos que poco a poco llegan a la sociedad con el esfuerzo de los investigadores a lo largo de décadas, junto con el desarrollo de instrumentos cada vez más sofisticados, para llegar a nuevas conclusiones. 

	\item \textbf{Desarrollo tecnológico}, que posteriormente han tenido aplicaciones directas en la sociedad. Por ejemplo:   \cite{beneficiosastro2}
	
	
	\begin{itemize}
		\item \textbf{Detectores CCD} (\ref{CCD}), que usan nuestras cámaras de fotos fueron inventados en 1969, pero lograron desarrollarse rápidamente gracias a sus aplicaciones en instrumentos astronómicos \cite{ccd}.
		
		\item \textbf{Detectores de rayos X}, que existen en los aeropuertos fueron desarrollados para aplicar los  conocimientos  adquiridos en los detectores de rayos X para instrumentos astronómicos \cite{detectores_rayos_x}.
		
		\item \textbf{Televisión satélite}, con la que estamos familiarizados y usamos a diario en nuestra vida cotidiana \cite{tv_satelite}. 
=======
\bigskip
Los beneficios que obtiene nuestra sociedad los podemos catalogar según su alcance en:
\begin{itemize}
	\item Beneficio puramente científico, el impulso por el conocimiento que ha caracterizado desde siempre al ser humano. Responder interrogantes existenciales como “¿De dónde venimos?, ¿a dónde vamos?”.


	\item Buscar la expansión del Universo, el Big Bang, Agujeros Negros.. son conceptos cosmológicos complejos, que poco a poco  llegan a la sociedad. Para ello ha sido necesario  muchos esfuerzos por parte de investigadores, a lo largo de décadas, junto con el desarrollo de instrumentos cada vez más sofisticados, para llegar a estas conclusiones. 

	\item Desarrollo de numerosas tecnologías que posteriormente han tenido aplicaciones en la sociedad, tenemos varios ejemplos:   \cite{beneficiosastro2}
	
	
	\begin{itemize}
		\item Detectores CCD que usan nuestras cámaras de fotos fueron inventados en 1969, pero lograron desarrollarse rápidamente gracias a sus aplicaciones en instrumentos astronómicos.
		
		\item Detectores de  rayos X que existen en los aeropuertos fueron desarrollados  para aplicar los  conocimientos  adquiridos en los detectores de rayos X para instrumentos astronómicos. 
		
		\item Televisión satélite, con la que estamos familiarizados y usamos a diario en nuestra vida cotidiana. 
		
>>>>>>> c9f08dfe66521d4f0dba18e652f93a6a37a333aa
	\end{itemize}
\end{itemize}


\subsection{Astronomía amateur}

<<<<<<< HEAD
La astronomía \textbf{amateur} es la realizada por astrónomos no profesionales, normalmente sin formación reglada en la materia, y que su interés primordial esta en aprender, conocer esta ciencia, asistir a charlas o quedadas astronómicas y compartir esta afición con otras personas.


La labor de este conjunto es muy valorada, dado que suelen compartir sus trabajos, colaborando con asociaciones y nutriendo de numeroso material muy variopinto: observaciones desde múltiples ubicaciones, momentos en el tiempo o equipo de observación diferente, ampliando así la cantidad de muestras. No puede en este caso llevar más razón el refranero español: \textit{``Más ven cuatro ojos que dos''}.


Todo este trabajo posteriormente puede ser organizado y estudiado por profesionales y para sacar conclusiones valiosas.


Además, muchos astrónomos aficionados, dedican tiempo y energías impartiendo conferencias divulgativas que acercan esta ciencia, sus métodos y sus conclusiones al gran público.


Otro de los perfiles más visuales en el mundo de la astronomía amateur es el de la \textbf{astrofotografía} ~\cite{AstroFoto} (figura~\ref{fig:nightsky}), que consiste en la captación fotográfica de las imágenes de cuerpos celestes, teniendo gran valor artístico en muchos casos.


=======
La astronomía \textbf{amateur} es la realizada por astrónomos no profesionales, normalmente sin realizar formación reglada, y que su interés primordial esta en aprender, conocer esta ciencia, asistir a charlas o quedadas astronómicas y compartir esta afición con otras personas.

\bigskip
La labor de este conjunto es muy valorada, dado que suelen compartir sus trabajos, colaborando con asociaciones  y nutriendo de numeroso material muy variopinto: observaciones desde múltiples ubicaciones, momentos en el tiempo o equipo de observación diferente, ampliando así la cantidad de muestras. No pudiendo llevar más razón el refranero español, \textit``Más ven cuatro ojos que dos"

\bigskip
Todo este trabajo posteriormente puede ser organizado por profesionales y sacar conclusiones.

\bigskip
Tampoco podemos olvidar la labor divulgativa dado que muchos astrónomos aficionados dedican tiempo y energías impartiendo conferencias tanto en agrupaciones, como en actos culturales promovidos por municipios, escuelas o universidades. 

\bigskip
Uno de los perfiles más visuales, es  la \textbf{astrofotografía} ~\cite{AstroFoto}, que consiste  en la captación fotográfica de las imágenes de  cuerpos celestes, teniendo gran valor artístico en muchos casos.

\bigskip
>>>>>>> c9f08dfe66521d4f0dba18e652f93a6a37a333aa
En muchas ocasiones, la frontera entre astrónomos profesionales y amateur es muy tenue porque ambos contribuyen de manera destacada al conocimiento del cielo.


\begin{figure}
\centering
<<<<<<< HEAD
\includegraphics[width=1\linewidth]{../images/nightsky}
\caption[Sierra Nevada Night Sky Time Lapse]{\textbf{Isidro Villo - Sierra Nevada Night Sky Time Lapse}, fotograma del vídeo tomado entre junio y agosto de 2011 en Sierra Nevada en memoria de su compañero Iker Canales Onaindia, muestra el lado más creativo y artístico de la astronomía. \textbf{Fuente:} \cite{sierra_nevada_sky}}
=======
\includegraphics[width=0.7\linewidth]{../images/nightsky}
\caption{\href{https://vimeo.com/28399458}{Isidro Villo - Sierra Nevada Night Sky Time Lapse}}
>>>>>>> c9f08dfe66521d4f0dba18e652f93a6a37a333aa
\label{fig:nightsky}
\end{figure}



\section{Instrumental Astronómico}

<<<<<<< HEAD
La astronomía está intimamente relacionada con una serie de herramientas e instrumentos que permiten expandir los sentidos del observador, permitiendo llegar a ver objetos más lejanos y tenues y apreciar más características de ellos. A continuación se describen algunos de estos instrumentos. 

\subsection{Telescopios} \label{telescopio}
=======
Como ya he enunciado anteriormente la astronomía también está relacionada  con una serie de herramientas e instrumentos básicos que permiten expandir los sentidos del observador, permitiendo llegar a ver objetos más lejanos y apreciar más características de ellos.  
>>>>>>> c9f08dfe66521d4f0dba18e652f93a6a37a333aa

El telescopio es un instrumento que básicamente recoge la mayor cantidad posible de luz emitida por un objeto situado fuera de la atmósfera y la concentrar para así permitir la detección de imágenes que a simple vista son inapreciables \cite{telescopio} (figura~\ref{fig:telescopio}).

<<<<<<< HEAD
=======
El telescopio es un instrumento que básicamente recoge la mayor cantidad posible de luz emitida por un objeto situado fuera de la atmósfera y concentrarla, para así permitir la detección de imágenes que a simple vista son inapreciables\cite{Telescopio}.
>>>>>>> c9f08dfe66521d4f0dba18e652f93a6a37a333aa


\begin{figure}[!ht]
	\begin{center}
		\includegraphics[width=0.6\textwidth]{../images/telescopio2.jpg}
<<<<<<< HEAD
			\caption[Telescopio]{Vista de un telescopio refractor \textbf{Fuente:} \cite{telescopio}}	
		\label{fig:telescopio}
	\end{center}
\end{figure}


 Es una herramienta fundamental en astronomía, y cada mejora de este instrumento ha permitido avances en la comprensión del Universo.

=======
			\caption[Telescopio]{Telescopio \href{http://cienciaaaoa.blogspot.com.es/2014/11/instrumentos-cientificos_20.html}{Telescopio}.}
		\label{fig:telescop}
	\end{center}
\end{figure}

\bigskip
 Es una herramienta fundamental en astronomía, y cada desarrollo o perfeccionamiento de este instrumento ha permitido avances en la comprensión del Universo.
>>>>>>> c9f08dfe66521d4f0dba18e652f93a6a37a333aa

Debemos agradecer este instrumento en gran parte a \textbf{Galileo} \cite{galileo}, cuyos avances permitieron usar el aparato como instrumento astronómico. 

Existen dos grandes tipologías entre los telescopios, según el tipo de sistema óptico que utilizan: los \textbf{reflectores} y los \textbf{refractores}. 


\textbf{Los reflectores} (figura~\ref{fig:reflector}) se constituyen de un espejo principal (espejo primario u objetivo), el cual no es plano como los espejos convencionales, sino que es provisto de cierta curvatura que le permite concentrar la luz en un punto.

\begin{figure}[h]
	\centering
	\includegraphics[width=1\linewidth]{../images/refrector}
	\caption[Diagrama telescopio reflector]{Diagrama funcionamiento telescopio reflector \\ \textbf{Fuente:} \cite{tipos_telescopios}}
	\label{fig:reflector}
\end{figure}

<<<<<<< HEAD

\textbf{Los refractores} (figura~\ref{fig:refractor}) poseen como objetivo una lente (o serie de lentes, la cantidad varía según el diseño y calidad) que de forma análoga al funcionamiento de una lupa, concentran la luz en el plano focal. 
=======
Existen varios diseños para este tipo de telescopios. Los mas conocidos entre los aficionados son el \textbf{reflector Newtoniano} y el \textbf{reflector Schmidt-Cassegrain}. 


\bigskip
Los refractores poseen como objetivo una lente (o serie de lentes, la cantidad varía según el diseño y calidad) que de forma análoga al funcionamiento de una lupa, concentran la luz en el plano focal. 
>>>>>>> c9f08dfe66521d4f0dba18e652f93a6a37a333aa


\begin{figure}[h]
	\centering
	\includegraphics[width=1\linewidth]{../images/refractor}
		\caption[Diagrama telescopio refractor]{Diagrama funcionamiento telescopio Refractor \\ \textbf{Fuente:}\cite{tipos_telescopios} }
	\label{fig:refractor}
\end{figure}



\subsection{Cámaras CCD} \label{ccd}

Un \textit{dispositivo de carga acoplada}, conocido también como \textbf{CCD} (figura~\ref{fig:ccd}) es un circuito integrado que contiene un número determinado de condensadores enlazados bajo el control de un circuito interno. Se encarga de convertir la señal luminosa en una señal eléctrica, componiendo una imagen digital. 


La capacidad de resolución de la imagen, depende del número de células fotoeléctricas del CCD. A mayor número de píxeles, mayor nitidez en relación con el tamaño. Actualmente las cámaras fotográficas digitales incorporan CCD con capacidades de hasta 160 megapíxeles \cite{ccd}.

<<<<<<< HEAD
=======
Dispositivo de carga acoplada (en inglés \textbf{Charge-Coupled Device}, conocido también como \textbf{CCD}), es un circuito integrado que contiene un número determinado de condensadores enlazados bajo el control de un circuito interno. Se encarga de la conversión de una señal luminosa en una señal eléctrica.
>>>>>>> c9f08dfe66521d4f0dba18e652f93a6a37a333aa

\begin{figure}[!ht]
	\begin{center}
		\includegraphics[width=0.6\textwidth]{../images/ccd.jpg}
		\caption[Cámara CCD]{Cámara CCD modelo  Starlight Xpress. \textbf{Fuente:}\cite{ccd_ellunatico}}
		\label{fig:ccd}
	\end{center}
\end{figure}


<<<<<<< HEAD
\subsection{Monturas} \label{montura}

La montura de un telescopio (figura~\ref{fig:montura}) es la parte mecánica que une el trípode o base al instrumento óptico o telescopio. Existen varios tipos de monturas, algunas muy simples, otras mas complejas. Hoy en día son comunes las que tienen correctores electrónicos y dispositivos de localización y seguimiento sofisticados (sistemas \textbf{GOTO}).


\begin{figure}[!ht]
	\begin{center}
		\includegraphics[width=0.8\textwidth]{../images/montura.jpg}
		\caption[Montura]{Montura robotizada modelo SkyWatcher EQ-8  \textbf{Fuente:} \cite{montura_ellunatico}}
		\label{fig:montura}
	\end{center}
\end{figure}

La montura tiene como objetivo proveer de movimiento controlado al telescopio \cite{montura}.


La más simple es la montura \textbf{altacimutal}, que realiza movimientos horizontales y verticales. Este tipo de diseño lo traen incorporados los telescopios pequeños, por lo general \textbf{telescopios refractores} de uso terrestre, dado que su uso es simple, y también varios modelos de equipos automatizados.


Le sigue la \textbf{montura ecuatorial}, que utiliza como plano fundamental el ecuador celeste (proyección del ecuador terrestre). Este diseño usa las coordenadas ecuatoriales, ascensión recta (A.R. o R.A.) y declinación (Dec.), que son proyecciones de las coordenadas terrestres longitud y latitud, respectivamente, sobre la esfera celeste. 


\subsection{Rueda portafiltros} \label{filtros}

La rueda porta-filtros (figura~\ref{fig:portafiltros}), consiste en un cuerpo generalmente de aluminio, que en su interior puede alojar varios \textbf{filtros}. El tamaño de los filtros así como el número de los mismos que puede alojar una rueda portafiltros varía segun los distintos modelos existentes.

Unos filtros comunes son los de colores, utilizados para resaltar las características de los objetos observados, sobre todo la atmosferas y superficies de los planetas.
=======
\bigskip
Físicamente, un \textbf{CCD} es una malla muy empaquetada de electrodos de polisilicio colocados sobre la superficie de un chip. Al impactar los fotones sobre el silicio se generan electrones que pueden guardarse temporalmente. Periódicamente se lee el contenido de cada píxel haciendo que los electrones se desplacen físicamente desde la posición donde se originaron (en la superficie del chip), hacia el amplificador de señal con lo que se genera una corriente eléctrica que será proporcional al número de fotones que llegaron al píxel. Para coordinar los periodos de almacenamiento (tiempo de exposición) y vaciado del píxel (lectura del píxel) debe existir una fuente eléctrica externa que marque el ritmo de almacenamiento-lectura: el reloj del sistema. La forma y amplitud de reloj son críticas en la operación de lectura del contenido de los píxeles.

>>>>>>> c9f08dfe66521d4f0dba18e652f93a6a37a333aa


Gracias a los filtros, podemos fragmentar el espectro de luz dejando pasar luz de una determinada longitud de onda infiriendo a partir de estas mediciones características de los objetos observados.


\begin{figure}[!ht]
	\begin{center}
		\includegraphics[width=0.6\textwidth]{../images/portafiltros.jpg}
		\caption[Rueda portafiltros]{Rueda portafiltros \textbf{Fuente:} \cite{rueda_portafiltros}).}
		\label{fig:portafiltros}
	\end{center}
\end{figure}


\subsection{Cúpulas}
Las \textbf{cúpulas} (figura~\ref{fig:cupula}) son recintos cerrados mas o menos grandes que nos permiten albergar y proteger el instrumental astronómico. De esta forma, las \textbf{cúpulas} pueden ser abiertas o cerradas para exponer los instrumentos en el momento de las observaciones.


\begin{figure}[!ht]
	\begin{center}
		\includegraphics[width=0.65\textwidth]{../images/cupula.jpg}
		\caption[Cúpula]{Cúpula - \textbf{Fuente:} \cite{cupula_ellunatico}.}
		\label{fig:cupula}
	\end{center}
\end{figure}


\subsection{Estaciones meteorológicas} \label{estacion_meteorologica}

Las \textbf{estaciones meteorológicas} son sistemas compuestos por un  un conjunto de sensores que nos proporcionan datos de las distintas magnitudes meteorológicas, tales como la temperatura, humedad, presión barométrica, presencia de nubes, viento, etc... permitiéndonos generar modelos a partir de los cuales conocer la situación climática y su posible evolución. 


Gracias a los datos aportados por las \textbf{estaciones meteorológicas}, podemos conocer la climatología en el momento de realizar observaciones astronómicas. De esta forma podemos decidir si las condiciones son óptimas o si debemos cerrar la cúpula y abortar una observación para evitar daños en los instrumentos por lluvias o similar. 


\subsection{Enfocadores} \label{enfocadores}

El \textbf{enfocador} (figura~\ref{fig:enfocador_tecnosky}), es una pieza fundamental del telescopio que nos permitirá ver con nitidez las imágenes formadas tras la reflexión de la luz en el espejo primario y su desviación por el espejo secundario.


El enfoque viene determinado por la convergencia de la mayor parte de los rayos justo en el plano focal, que es donde colocamos el ojo o una cámara. 


El enfocador, es una pieza adaptada al telescopio, usualmente con una rueda dentada sobre una cremallera que podemos manipular para desplazar el portaocular y variar la distancia focal para encontrar el punto de foco óptimo \cite{enfocador}.


Como en observación astronómica se suele trabajar a muchos aumentos, un pequeño error de enfoque se magnifica en una imagen poco nítida o desenfocada.


Para ayudar en esta operación los astrónomos han perfeccionado algunas técnicas, una de las más extendidas es hacer uso de una \textbf{máscara de enfoque}, \cite{FocusMascara} que consiste en unas rendijas por las que hacemos pasar la luz de un objeto luminoso, difractando los rayos y observando la dirección que toman.


Otra característica que incorporan muchos enfocadores comerciales es la \textbf{compensación por temperatura}, para ello incorpora un sensor térmico en la óptica que informa de oscilaciones en la temperatura (que puedan producir dilatación en la lente o tubo). Cuando se detecta una oscilación considerable en la temperatura, se ejecuta un ciclo de autoenfoque o se compensa según alguna regla establecida.


\begin{figure}[h!]
	\begin{center}
		\includegraphics[width=0.6\textwidth]{../images/enfocador.jpg}
		\caption[Enfocador modelo Tecnosky]{Enfocador modelo Tecnosky- \textbf{Fuente:} \cite{enfocador_ellunatico}}
		\label{fig:enfocador_tecnosky}
	\end{center}
\end{figure}


Otra técnica más reciente pero muy extendida entre los astrónomos, es el uso de \textbf{software} especializado, que mediante procesamiento de imagenes es capaz de obtener una medida de la nitidez del objeto observado con el fin de establecer algoritmos automáticos de enfoque.

<<<<<<< HEAD

\section{Software astronómico}

Según su utilidad podemos distinguir y catalogar el software astronómico en los siguientes conjuntos básicos:

\begin{itemize}
	\item \textbf{Planetarios y cartas celestes}: Permiten orientarnos en el cielo, permitiendo así reconocer estrellas, planetas, galaxias, constelaciones. Tiene conexión con bases de datos de todos estos objetos, teniendo información en tiempo real, de su posición. Un ejemplo es \href{http://www.stellarium.org/es/}{\textbf{Stellarium}} \cite{stellarium} o \href{https://www.google.com/intl/es_es/sky/}{\textbf{Google Sky}} \cite{gsky}.
	
	\item \textbf{Alineación y Guiado}: Su función es calibrar y sincronizar los elementos del observatorio (entre ellos y con el movimiento del cielo), para ello también pueden incorporar un ``planetario''. Ejemplos de este tipo de software son \href{https://edu.kde.org/kstars/}{\textbf{KStars}} \cite{kstars} (solución libre) y \href{http://www.cyanogen.com/maxim_main.php}{\textbf{MaxIm DL}} \cite{maximdl}.
	
	\item \textbf{Captura y procesado de imágenes}: se encargan especialmente de capturar imagenes y procesarlas de forma automática. Podemos destacar \href{http://www.mlunsold.com/}{\textbf{ImagesPlus}} (privativo), \href{http://deepskystacker.free.fr/spanish/index.html}{\textbf{Deep Sky Stacker}} y \href{https://github.com/ejeschke/ginga}{\textbf{Ginga}} (software libre) \cite{ginga}. 
	
	
\end{itemize}

\section{Imágenes Astronómicas}

En las imágenes astronómicas interesa almacenar la máxima cantidad de información sobre la imagen, es por ello que siempre se usan formatos \textbf{sin compresión o con compresión sin pérdidas}, como puede ser RAW \cite{Raw} o FITS (figura~\ref{fig:fit}) \cite{FITS}.


\textbf{FITS} es un formato de imágenes especialmente concebido para el mundo de la astronomía, por permitir almacenar información más allá de la visible, así como espectros electromagnéticos.

=======
\bigskip
Para ayudar en esta operación los astrónomos han perfeccionado algunas técnicas, una de las más extendidas es hacer uso de una máscara de enfoque, \cite{FocusMascara} que no es más que unas rendijas por las que hacemos pasar la luz de un objeto luminoso, difractando los rayos y observando la dirección que toman.

\bigskip
Otra característica que incorporan muchos enfocadores comerciales es la \textbf{compensación por temperatura}, para ello incorpora un sensor en la óptica, que informa de  oscilaciones en la temperatura, (que puedan producir dilatación en la lente o tubo), 
cuando se detecta una oscilación considerable en la temperatura, se ejecuta un ciclo de autoenfoque o se compensa según alguna regla establecida.


Una técnica muy extendida entre los astrónomos es el uso de \textbf{software} especializado, que mediante procesamiento de imagenes es capaz de obtener una medida de la nitidez del objeto y compararlo con el valor óptimo.

\bigskip
Dado que es una de los puntos clave del ámbito del proyecto, no paramos a profundizar dado que más adelante enterremos en detalle en la explicación de las medidas, algoritmos, así como las implementaciones de los mismo.


\section{Software astronomico}

\begin{itemize}
	\item \textbf{Planetarios y cartas del cielo}: Permiten orientarnos en el cielo, permitiendo así reconocer estrellas, planetas, galaxias, constelaciones. Tiene conexión con bases de datos de todos estos objetos, teniendo información en tiempo real, de su posición. Un ejemplo es \href{http://www.stellarium.org/es/}{Stellarium} o \href{https://www.google.com/intl/es_es/sky/}{Google Sky}.
	
	\item \textbf{Alineación y Guiado}: Su función es calibrar y sincronizar los elementos del observatorio (entre ellos y con el movimiento del cielo), para ello también pueden incorporar un "planetario". Un ejemplo de tal software es \href{https://edu.kde.org/kstars/}{KStars}, (solución libre) y por \href{http://www.cyanogen.com/maxim_main.php}{MaxIm DL}
	
	\item \textbf{Captura y procesado de imágenes}, se encargan especialmente de capturar imagenes y procesarlas de forma automática, señalar \href{http://www.mlunsold.com/}{ImagesPlus Camera Control} (privativo), \href{http://deepskystacker.free.fr/spanish/index.html}{Deep Sky Stacker} y \href{https://github.com/ejeschke/ginga}{Ginga} (software libre). 
	
	
\end{itemize}

\subsection{Formato imágenes}

En las imagenes astronómicas nos interesa almacenar la máxima cantidad de información sobre la imagen que tomamos, es por ello que siempre se usan formatos sin compresión, como puede ser RAW \cite{Raw} o FITS "Flexible Image Transport System"  \cite{FITS}.

\bigskip
\textbf{FITS} es un formato de imagenes especialmente concebido para el mundo de la astronomía, por permitir almacenar información más allá de la visible, así como espectros electromagnéticos.

\bigskip
Una característica muy interesante para los astrónomos es la incorporación de cabeceras en texto plano y legibles sin software adicional, en esta cabeceras se introducen \textbf{metadatos}, acerca del la observación,  posición geográfica, marcas de tiempo,  características de la cámara, filtros empleados entre otros.

\bigskip
FITS está soportado mediante bibliotecas disponibles en los lenguajes más utilizados en el ámbito científico, incluyendo C, FORTRAN, Java, Perl, PDL, Python, e IDL. 

\begin{figure}[h]
	\centering
	\includegraphics[width=1.0\linewidth]{../images/fit}
	\caption[Cabecera FITS]{Cabecera FITS}
	\label{fig:fit}
	\end{figure}
	
	\bigskip
	Además existen numerosos entornos de procesamiento, que permiten manipular este tipo de imagenes, por nombrar uno de los más conocidos \textbf{ImageJ} \cite{Imagej}.
	
>>>>>>> c9f08dfe66521d4f0dba18e652f93a6a37a333aa

Una característica muy interesante para los astrónomos, es la incorporación de cabeceras en texto plano y legibles sin software adicional: en esta cabecera se introducen \textbf{metadatos} sobre la observación realizada, posición geográfica, marcas de tiempo, características de la cámara, filtros empleados, etc.


<<<<<<< HEAD
FITS está soportado por bibliotecas disponibles en los lenguajes más utilizados en el ámbito científico, incluyendo C, FORTRAN, Java, Perl, PHP, Python.

\begin{figure}[h]
	\centering
	\includegraphics[width=1\linewidth]{../images/fit}
	\caption[Cabecera FITS]{\textbf{Cabecera FITS}. Observamos como se puede leer la cabecera de una imagen en formato FITS usando un editor de texto.}
	\label{fig:fit}
=======
En la actualidad se están implantando diversos protocolos y estándares de manejo remoto al campo de la astronomía, para agilizar y facilitar la observación, librar al astrónomo de tareas tediosas, soportar condiciones climatologías adversas y permitirle centrarse en la propia observación así como multiplicar el numero de puestos, dado que un equipo de astrónomos puede controlar varios observatorios en remoto, mientras que de forma física se ven limitados a uno solo. 

\begin{figure}[h]
\centering
\includegraphics[width=0.7\linewidth]{../images/robotizacion}
\caption{\href{http://www.lost-infinity.com/equipment/}{Observatorio Carsten Schmitt - (Lost Infinity)}}
\label{fig:robotizacion}
>>>>>>> c9f08dfe66521d4f0dba18e652f93a6a37a333aa
\end{figure}
	

Además existen numerosos entornos de procesamiento de imágenes que permiten manipular este tipo de imagenes, como por ejemplo \textbf{ImageJ} \cite{Imagej}.
	

\section{Control remoto de dispositivos astronómicos}

En la actualidad se están implantando diversos protocolos y estándares de control remoto al campo de la astronomía para agilizar y facilitar las observaciones. Estos estándares pretenden librar al astrónomo de tareas tediosas controlar la seguridad en la operación de los instrumentos (por ejemplo condiciones climatologías adversas) y permitirle centrarse en la propia observación, que puede realizarse desde cualquier parte del mundo y a cualquier hora.

También permite multiplicar el número de puestos, dado que un equipo de astrónomos puede controlar varios observatorios en remoto, mientras que de forma física se ven limitados a uno solo. 

Existen diversas formas de controlar los dispositivos astronómicos pero la mayoría presenta los mismos inconvenientes:

\begin{itemize}
	\item Normalmente se controlan los dispositivos diréctamente: conectando cada dispositivo a un PC y se trabaja desde dicho ordenador.
	\item En ocasiones se utilizan herramientas para el control remoto del PCcomo el \textbf{escritorio remoto} \cite{escritorio_remoto}, lo que puede ocasionar algunos inconvenienes como lag excesivo o un consumo de ancho de banda alto.
\end{itemize}

La evolución de las plataformas se puede resumir como sigue:

\begin{enumerate}
<<<<<<< HEAD
\item \textbf{Esquema monolítico:} Cada dispositivo, funciona con \textbf{su propio cliente} y usa un \textbf{protocolo particular}.

\item \textbf{Esquema extensible:} Donde algunos \textbf{fabricantes comparten el código de control}, pero nadie de forma independiente puede implementar nuevos plugins.

\item \textbf{ASCOM}, \ref{ASCOM} intenta crear una \textbf{capa intermedia entre los programas cliente y los dispositivos astronómicos}, dado que es un estándar abierto, cualquiera puede implementar nuevos driver para sus dispositivos. Como inconveniente encontramos que solo puede utilizarse en sistemas \textbf{Microsoft Windows}. Su diseño tiene una relación bastante profunda con el sistema operativo lo cual dificulta el desarrollo basado en red. \cite{ascom}
=======
\item Esquema monolítico:  Cada dispositivo, funciona con us propio cliente y usa su protocolo particular.

\item Esquema extensible: Donde algunos vendedores comparten el código de control, nadie de forma independiente puede implementar nuevos plugins.

\item ASCOM, intenta crear una capa entre los programas para controlar dispositivos astronómicos y los propios dispositivos, dado que es un estándar abierto, cualquiera puede implementar nuevos driver para sus dispositivos, como inconveniente solo puede utilizarse en sistemas \textit{Microsoft Windows}. Su diseño tiene una relación bastante profunda con el sistema operativo lo cual dificulta el desarrollo basado en red.

\item INDI, es más abierto que el anterior y dispone de múltiples implementaciones en C y  Java \cite{indiforjava}, se puede desplegar en en cualquier sistema operativo, ya sea Windows, Linux o Mac. 
>>>>>>> c9f08dfe66521d4f0dba18e652f93a6a37a333aa

\item \textbf{INDI}, es un protocolo \textbf{más abierto que el anterior y dispone de múltiples implementaciones} en C y  Java \cite{indiforjava}, se puede desplegar en cualquier sistema operativo, ya sea \textbf{Windows}, \textbf{Linux} o \textbf{Mac} \cite{indi}. 
\end{enumerate}


\subsection{INDI}

\begin{quote}``\textit{The Instrument Neutral Distributed Interface (INDI) Library is a cross-platform software designed for automation  control of astronomical instruments. It supports a wide variety of telescopes, CCDs, focusers , filter wheels, etc., and it has the capability to support virtually any device. INDI is small, flexible, easy to parse, and scalable. It supports common DCS functions such as remote control, data acquisition, monitoring, and a lot more. With INDI, you have a total transparent control over your instruments so you can get more science with less time.}''
\newline
\\
\textbf{Fuente:} \cite{about_indi}
\end{quote}


El protocolo \textbf{INDI} es una plataforma software diseñada para el control de instrumental astronómico, aunque podría usarse con cualquier dispositivo, incluidos ``virtuales''. La biblioteca \textbf{INDI} permite controlar cualquier dispositivo para el que se haya desarrollado un driver \textbf{INDI}. Funciona mediante el paso de paso de información en formato XML (\ref{XML}). 


Sus principales ventajas frente a otras soluciones para el control de dispositivos son:


\begin{itemize}
	\setlength\itemsep{0.2em}
	\item Es una biblioteca \textbf{ligera}, \textbf{flexible} y \textbf{escalable}.
	\item Es de código abierto por lo que cualquiera puede ver su código y mejorarlo o crear drivers para cualquier dispositivo.
	\item El intercambio de información entre clientes, servidores y drivers es mínimo.
	\item Es \textbf{multiplataforma}.
	\item Separa clara y totalmente el cliente del servidor.
	\item Los fabricantes comienzan a desarrollar drivers para sus dispositivos o liberan las especificaciones para que la comunidad pueda desarrollarlos.
	\item Existen \textbf{numerosos clientes INDI} como \href{https://edu.kde.org/kstars/}{kstars}, \textbf{Cartes Du Ciel}, \textbf{Xephem}, \textbf{Observatorio Remoto} (Android) \cite{obsremoto}, \textbf{Stellarium} \cite{stellarium}.
	
\end{itemize}

<<<<<<< HEAD
\begin{figure}[h]
	\centering
	\includegraphics[width=0.9\linewidth]{../images/robotizacion}
	\caption[Observatorio Carsten Schmitt]{Observatorio Carsten Schmitt - (Lost Infinity), haciendo uso de control remoto y software astronómico especializado. \textbf{Fuente:} \cite{lost_infinity}}
	\label{fig:robotizacion}
=======

\bigskip

\subsection{Breve introducción a INDI}

INDI consiste a su nivel más básico en un protocolo que permite el control, automatización, obtención de datos e intercambio de los mismos entre distintos dispositivos hardware y programas cliente. La idea subyacente en el protocolo INDI es desacoplar aspectos específicos del hardware que se controla de tal manera que cambios en el hardware no impliquen necesariamente cambios en el software (cosa que ocurre en sistemas más habituales donde el el frontend software está fuertemente acoplado con el backend hardware.

\bigskip
\begin{figure}[!ht]
	\begin{center}
		\includegraphics[width=0.21\textwidth]{../images/indi.png}
		\caption[INDI Logo]{INDI Lib (\href{http://indilib.org/}{http://indilib.org/})}
		\label{fig:ascom}
	\end{center}
>>>>>>> c9f08dfe66521d4f0dba18e652f93a6a37a333aa
\end{figure}


\section{Hardware Libre}

Se llama hardware libre a aquellos dispositivos cuyas especificaciones y esquemáticos son de acceso público, ya sea bajo algún tipo de pago, o de forma gratuita. La filosofía del software libre es aplicable a la del hardware libre, y por eso forma parte de la \textbf{cultura libre} \cite{HWLIBRE}.


Los problemas que trata de solventar el hardware libre son los siguientes:

<<<<<<< HEAD
\begin{enumerate}
	\setlength\itemsep{0.2em}
	\item \textbf{Conocimiento restringido}, el conocimiento lo poseen las empresas. El hardware libre trata de hacer pública toda la documentación, diagramas, data shield, fichas, etc.

	\item \textbf{Falta de materiales o herramientas para la fabricación.} 	Tratan de utilizar componentes estándar, que se puedan encontrar fácilmente así como hacer recomendaciones de algunas tiendas online donde puedes comprar tales componentes.  

	\item \textbf{Altos costes de producción.} Al ser de diseño libre, diferentes fábricas pueden ocuparse de proveer los componentes, preocupándose por optimizar y agilizar el proceso, llegando a bajar los costes de producción.

	\item \textbf{Gran inversión en realizar trabajos redundantes.}	No hay que preocuparse por solucionar una y otra vez los mismo problemas: para nuestros diseños podemos partir de otros proyectos consolidados.
\end{enumerate}

=======
Los problemas que trata de solventar el hardware libre:

\bigskip
\begin{enumerate}
	\item Conocimiento lo poseen las empresas.
	Haciendo público todos la documentación, diagramas, data shield, fichas etc.

	\item Falta de materiales o herramientas para la fabricación.
	Tratan de utilizar componentes estándar, que se puedan encontrar fácilmente. 
	Así como hacer recomendaciones de algunas tiendas online donde puedes comprar tales componetes.  

	\item Altos costes de producción.
	Al ser de diseño libre, diferentes factorías pueden ocuparse de fabricar los componentes, preocupandondose por optimizar y agilizar el proceso, llegando a bajar los costes de producción.

	\item Gran inversión en realizar trabajos redundantes. 
	No hay que preocuparse por solucionar una y otra vez los mismo problemas, para nuestros diseños podemos partir de otros proyectos consolidados.
		  
\end{enumerate}
>>>>>>> c9f08dfe66521d4f0dba18e652f93a6a37a333aa

Al auspicio de este movimiento han surgido muchísimas plataformas y tecnologías, pero en especial cabe remarcar dos de las más importantes, \textbf{Arduino} \cite{ARDUINO} y \textbf{Raspberry Pi} \cite{raspberry}.

<<<<<<< HEAD
=======
Dado este movimiento han surgido muchísimas plataformas, pero en especial cabe remarcar dos de las más importantes, Arduino y Raspberry Pi.
>>>>>>> c9f08dfe66521d4f0dba18e652f93a6a37a333aa

\subsection{Arduino}


\begin{figure}
\centering
\includegraphics[width=0.75\linewidth]{../images/arduino}
\caption[Arduino]{Logo oficial de Arduino - \textbf{Fuente:} \cite{ARDUINO}}
\label{fig:arduino}
\end{figure}


<<<<<<< HEAD
\paragraph{Arduino} es una compañía de hardware libre, la cual desarrolla placas de desarrollo (figura~\ref{fig:arduinocasero}) que integran un \textbf{microcontrolador} y un entorno de desarrollo \cite{IDE}.
=======
\bigskip
\paragraph{Arduino} es una compañía de hardware libre, la cual desarrolla placas de desarrollo que integran un \textbf{microcontrolador} y un entorno de desarrollo  \cite{IDE}.

\bigskip
Está diseñado para facilitar el uso de la electrónica en proyectos multidisciplinarios\cite{ARDUINO}.

\bigskip
La primer placa Arduino fue introducida en el 2005, ofreciendo un bajo costo y facilidad de uso para novatos y profesionales buscando desarrollar proyectos interactivos con su entorno mediante actuadores y sensores.
>>>>>>> c9f08dfe66521d4f0dba18e652f93a6a37a333aa

Está diseñado para facilitar el uso de la electrónica en proyectos multidisciplinares \cite{ARDUINO}.

Tal como marcan los principios del hardware libre, los esquemáticos de diseño del Hardware, están disponibles bajo licencia Libre, permitiendo a cualquier persona crear su propia placa Arduino sin necesidad de comprar una prefabricada. 


\begin{figure}[h]
	\centering
	\includegraphics[width=1\linewidth]{../images/caracteristicas_arduino}
	\caption[Diagrama Arduino]{Vista principal placa Arduino, podemos diferencias los diferentes componentes. \textbf{Fuente:} \cite{ARDUINO}}
	\label{fig:arduinocasero}
\end{figure}


\paragraph{Modelos de Arduino}  

Existen multitud de ediciones o modelos de placa, cada una pensada para un público concreto o para una serie de tareas o proyectos específicos. También han surgido muchos modelos no oficiales. 


Por nombrar algunas placas de las más famosas:
\begin{itemize}
\setlength\itemsep{0.2em}	
	
\item \textbf{Arduino UNO}: es la plataforma más extendida y la primera que salió al mercado.

\item \textbf{Arduino Yun}: Combina un chip ATmega32u4 y en un chip Atheros AR9331, que se comunican mediante un puente. El chip Atheros soporta distribuciones ligeras de Linux. 

\item \textbf{Arduino Mega}: Incorpora un chip ATmega2560, dando más rendimiento que el ATmega320 del Arduino UNO

\item \textbf{Arduino Ethernet}: Idéntico al Arduino UNO, pero con conexión a red.

\item \textbf{Arduino Nano y Micro y TinyDuino}: De muy pequeño tamaño y optimizadas para mejorar el consumo. 

\item \textbf{Arduino LilyPad}: está pensado para insertarse en prendas y textiles, es lavable. 

\end{itemize}


\paragraph{Shields para Arduino} 

Son placas de circuitos modulares que se montan unas encima de otras para dar funcionalidad extra a las placas de Arduino.

Las shields, se comunican con Arduino usando algunos de los pines digitales/analógicos, por algún bus como el SPI \ref{SPI}, I2C \ref{I2C} o bien por puerto serie. Además estas shields se alimentan generalmente a través del Arduino mediante los pines de 5V y GND \cite{shield}.

<<<<<<< HEAD
Podemos destacar algunos de los shield más comunes y recomendables para multitud de proyectos:
\begin{itemize}
 \item \textbf{Arduino Wifi Shield}, que añade comunicación wireless WIFI, 
 \item \textbf{Arduino GSM Shield}, comunicación GPRS (usando una tarjeta SIM),
 \item \textbf{Arduino Motor Shield}, permite manejar motores DC,
 \item \textbf{GPS Shield}, añade localización GPS,
 \item \textbf{Xbee Shield}, comunicación inalámbrica XBee \cite{XBee}, entre muchos otros. 
\end{itemize}
   
=======
\bigskip
Las shields se pueden comunicar con el arduino bien por algunos de los pines digitales o analógicos o bien por algún bus como el SPI, I2C o puerto serie, así como usar algunos pines como interrupción. Además estas shields se alimenta generalmente a través del Arduino mediante los pines de 5V y GND\cite{shield}.
>>>>>>> c9f08dfe66521d4f0dba18e652f93a6a37a333aa



\paragraph{Framework desarrollo Arduino}

El lenguaje de programación de Arduino (basado en Wiring) está implementado en C++. Posee su propio entorno de programación (figura~\ref{fig:arduinoide}), con opciones para compilar y cargar el código fuente en la placa que tenemos conectada. Los programas de Arduino se llaman ``sketch'' \cite{arduino_sketch}.

\begin{figure}
\centering
\includegraphics[width=0.9\linewidth]{../images/arduinoide}
\caption[Interfaz de programación de Arduino]{Interfaz de programación de Arduino, con un \textbf{sketch} básico \\ ``Hola Mundo" }
\label{fig:arduinoide}
\end{figure}


También podemos encontrar multitud de bibliotecas y módulos implementados por terceros \cite{arduino_libraries}, que se pueden incorporar directamente a nuestros proyectos, realizando algunas abstracciones sobre el hardware, comunicación o algún aspecto de más bajo nivel.




\subsection{Raspberry Pi}


Raspberry Pi \cite{raspberry} fue lanzado en 2006 por la Fundación Raspberry Pi con el objeto de facilitar la formación de informática en las escuelas de todo el mundo, especialmente en los lugares más desfavorecidos.

Esta placa (figura~\ref{fig:raspberry_pi}), a diferencia que Arduino, es capaz de ejecutar un sistema operativo completo, incluyendo el correspondiente servidor gráfico y servicios básicos como SSH o FTP. 

Actualmente contamos con la versión 3, que incorpora un procesador ARM de 4 núcleos con una velocidad de 1.2 GHz. El otro punto destacable es su GPU, capaz de llegar a resoluciones Full HD sin problema alguno. Junto a las características anteriores, cabe destacar 1 GB de memoria RAM, que permite ejecutar varias aplicaciones al mismo tiempo.

Otro punto a destacar de la Raspberry Pi es su conectividad, ya que incorpora en la misma placa el hardware necesario para dotar al sistema tanto de \textbf{Bluetooth}, de \textbf{WiFi} o de \textbf{Ethernet}.

También puede usarse para prototipado, por lo que incorpora 40 pines \textbf{GPIO} a los que se les puede conectar cualquier dispositivo sensor o actuador.

Recientemente ha aparecido una placa de la misma serie pero de reducidas dimensiones, solo 6x3 cm, a diferencia de la anterior tiene una potencia inferior y no cuenta con módulo de conexión a red.

Por su reducido coste y consumo estas plataformas son ideales para ejecutar servicios básicos, convertirse en concentradores de los datos que proporciona una red de sensores y dar acceso a ellos mediante Internet (función de puerta de acces o gateway).



\begin{figure}[h]
\centering
\includegraphics[width=0.8\linewidth]{../images/raspberry_pi}
\caption[Raspbery Pi]{Raspbery Pi \textbf{Fuente:} \cite{imagen_raspberrypi}}
\label{fig:raspberry_pi}
\end{figure}




\section{Enfocadores astronómicos: estado del arte}

Entendemos estado del arte como aquellos desarrollos de última tecnología realizados sobre un producto, que han sido probados en la industria y han sido acogidos y aceptados por diferentes fabricantes.

Dado el alcance e interés de este proyecto es de obligado cumplimiento el hacer una investigación entre los diferentes fabricantes de la soluciones existentes para el enfoque automático en astronomía. Una búsqueda en Internet nos ofrece algunas alternativas:

\begin{itemize}
	\item \textbf{Orion AccuFocus} \cite{orion_focuser}: Muy básico: ni control por ordendor, ni pantalla, ni botones de control preciso. Tiene un precio en el mercado de 99\$. 
	\item \textbf{FocusMaster} \cite{focusmate}, cuenta con las funciones básicas de control, incluido ajuste de velocidad y ajuste de movimiento fino. Su precio oscila en torno a los 200\$.  
	\item \textbf{Opetec} es uno de los fabricantes más especializados y dispone de varios modelos, el más económico tiene un precio de 725\$.
	\item \textbf{Robofocus}, una de las soluciones más avanzadas y extendidas, tiene un precio de 495\$.
\end{itemize}

A continuación detallamos un poco más las soluciones existentes más populares y completas.

\subsection{Opetec}

\textbf{Opetec} es uno de los fabricantes más importantes en el mundo de los productos electro-ópticos usados en astronomía. Su sede central se localiza en Lowell (Michigan, EEUU).

<<<<<<< HEAD

Entre los productos más económicos para enfoque astronómico contamos con el modelo, \textbf{17670 - Original TCF-S} (figura~\ref{fig:optecinc_focuser}), cuyas características estrellas son la \textbf{compensación térmica} y un \textbf{backlash pequeño}.


=======
>>>>>>> c9f08dfe66521d4f0dba18e652f93a6a37a333aa

\begin{figure}[h]
\centering
\includegraphics[width=0.9\linewidth]{../images/optecinc_focuser}
\caption[Opetec Original TCF-S]{Mando de control Opetec Original TCF-S. \textbf{Fuente:} \cite{optec}}
\label{fig:optecinc_focuser}
\end{figure}

<<<<<<< HEAD

La misma compañía cuenta con un software especifico para corregir el foco en tiempo real,\textbf{FocusLock Focusing Software} (figura~\ref{fig:focus_lock}).

\begin{figure}[h]
\centering
\includegraphics[width=1\linewidth]{../images/focus_lock}
\caption[Opetec Focus Lock]{Software Opetec Focus Lock en un ciclo de funcionamiento \textbf{Fuente:} \cite{optec} }
\label{fig:focus_lock}
\end{figure}

Se trata de un producto de alta calidad, pero su precio es elevado y el software requiere de la plataforma \textbf{ASCOM} (\ref{ASCOM}) lo que limita su uso. 



\subsection{Robofocus}
\label{sec:robofocus}

Robofocus (figura~\ref{fig:robofocus2}) es uno de los enfocadores astronómicos más populares que existen en el mercado es \textbf{Robofocus} \cite{robofocus}.

\begin{figure}
\centering
\includegraphics[width=1\linewidth]{../images/robofocus2}
\caption[Kit Robofocus]{Componentes de un kit Robofocus. \textbf{Fuente:} \cite{robofocus}}
\label{fig:robofocus2}
\end{figure}

Además una de contar con todas las características anteriores de la solución ofrecida por Opetec, incluida la componesación por cambio de temperatura, Robofocus es compatible con la plataforma INDI \cite{robofocus_indi}. 

Investigando el funcionamiento del dispositivo se puede encontrar el protocolo serie (tabla~\ref{robofocus_command}) que utiliza, que utiliza el siguiente formato:

\begin{itemize}
	\item \textbf{F} : Indica que es un comando del enfocador.
	\item \textbf{X} : Caracter alpha, selecciona el comando.
	\item \textbf{?} : Separador
	\item \textbf{NNNNNN} : Seis caracteres decimales (completando con ceros sí es necesario).
	\item \textbf{Z} : Checksum, suma de verificación del comando. 
\end{itemize}





\begin{table}[h]
	\centering
	\caption[Comandos Robofocus]{Descripción de los comandos de Robofocus}
	\label{robofocus_command}
	\begin{tabular}{|l|l|l|}
		\hline
		Comando   & Parámetro             & Descripción                                                                                   \\ \hline\hline
		FV      & -         & Consulta versión del firmware.                                                                              \\ \hline
		FG      & XXXXX     & Mueve a posición.                                                                                           \\ \hline
		FI      & XXXXX     & Mueve X pasos hacia el interior.                                                                            \\ \hline
		FO      & XXXXX     & Mueve X pasos hacia el exterior.                                                                            \\ \hline
		FS      & XXXXX     & Asigna posición actual como posición X.                                                                     \\ \hline
		FL      & XXXXX     & Asigna máximo recorrido.                                                                                    \\ \hline
		FP      & XXXXX     & \begin{tabular}[c]{@{}l@{}}Activa salidas de corriente electrica adicional, \\ para alimentar dispositivo adicional.\end{tabular} \\ \hline
		FB      & NXXXXXZ   & Cambia la compensación de backlash.                                                                         \\ \hline
		FT      & -         & Responde con la temperatura.   \\ \hline     
	\end{tabular}
\end{table}


A la vista de lo presentado podemos concluir que pese a que existen diversas soluciones para control del foco en astronomía, dichas soluciones:

\begin{itemize}
 \item Son privativas (lo que dificulta su mejora, estudio, modificación o arreglo).
 \item Solo funcionan bajo la plataforma ASCOM (Windows).
 \item Son caras (para lo que ofrecen).
\end{itemize}

Estos motivos son los que llevan a pensar que el desarrollo de una solución completa de enfoque que sea libre es algo interesante y que puede tener una cierta apreciación por los astrónomos (sobretodo los amateur).


=======
>>>>>>> c9f08dfe66521d4f0dba18e652f93a6a37a333aa

\chapter{Objetivos}

El objetivo global de este proyecto es desarrollar un sistema automático de \textbf{autoenfoque}, dicho sistema debe permitir un modo de funcionamiento remoto utilizando el protocolo \textbf{INDI},
adicionalmente se implementa una interfaz  \textbf{JAVA} que permita visualizar el funcionamiento de las distintas rutinas de procesamiento de imagenes,
sirviéndonos como framework de trabajo, por último se implementa una interfaz que permite ejecutar directamente la rutina de enfoque automático,
así como interactuar con el dispositivo enfocador de forma remota.


\bigskip
Este objetivo se desglosa en los siguientes objetivos principales:

\begin{itemize}
  \item \textbf{OBJ-1.} Diseño dispositivo usando plataforma libre Arduino  \textbf{Arduino}, que interactúa directamente con controles de enfoque de \textbf{telescopios}, mediante motor y distintos sensores, Ardufocuser.
  \item \textbf{OBJ-2.} Rutinas de procesamiento de imágenes, encargadas de detectar y filtrar objetos celestes con características estelares.
  \item \textbf{OBJ-3.} Extraer distintas medidas y heurísticas para evaluar el nivel de enfoque alcanzado en las distintas imágenes estelares, basándonos en el calculo FWHM.
  \item \textbf{OBJ-4.} Implementar Driver Indi \textbf{Driver INDI} que provea de la funcionalidad básica para controlar todas las características del dispositivo enfocador.
  \item \textbf{OBJ-5.} Implementar algoritmo de autofocus, basándonos en la medida del foco calculada y coordinada con las funciones del dispositivo Ardufocuser y las imagenes capturadas por el CCD.
\end{itemize}

\bigskip
Además de los objetivos principales, se persigue alcanzar los siguientes objetivos secundarios:

\begin{itemize}
  \item \textbf{OBJ-S-1.} Completar una buena documentación técnicas, así como manuales para desarrolladores, a fin de que cualquier interesado sea capaz de reproducir fácilmente y a bajo costo el proyecto.
  \item \textbf{OBJ-S-2.} Publicar código fuente, diseños y documentación bajo una licencia libre, que permita que la comunidad de desarrolladores interesados pueda realizar modificaciones y personalizaciones, siempre que se mantenga la misma licencia.
  \item \textbf{OBJ-S-3.} Difusión en la comunidad astronómica y desarrollo de una web propia del proyecto.

\end{itemize}

\bigskip
Para la realización de los objetivos se pondrán en practica los conocimientos alcanzados en:

\begin{itemize}
  \item \textbf{Ingeniería del software} para el análisis y diseño del proyecto, así como modelar el sistema.
  \item \textbf{Programación orientada a objetos} para la estructura y la organización del código \textbf{Java}.
  \item \textbf{Programación de sistemas multimedia} para poder implementar las interfaces de usuario en \textbf{Java Swing}, así como visualizar y tratar las imagenes.
  \item \textbf{Infraestructura virtual} para poder gestionar los sistemas, teniendo habilidad para realizar instalaciones y aprovisionamiento del servidor.
  \item \textbf{Transmisión de datos y redes de computadores} para comprender el comportamiento del protocolo \textbf{INDI} y configurar correctamente las redes para las pruebas.
  \item \textbf{Algorítmica} para optimizar las rutinas de tratamiento de imagenes.
  \item \textbf{Calculo matemático} para modelar los objetos celestes como una función gaussiana que representa la luminosidad, así como el calculo del FWHM.
  \item \textbf{Estructura de datos} dado que contamos con representaciones de las imagenes, que debemos conocer y saber manejar.


\end{itemize}

\bigskip
Por otro lado, han sido necesarios alcanzar conocimientos en otras áreas:

\begin{itemize}

  \item \textbf{Astronomía básica y equipos astronómicos} para entender a los usuarios potenciales y poder acomodar la aplicación a sus necesidades.
  \item \textbf{Raspberry Pi}\footnote{Ordenador de placa reducida y única de bajo coste.} para montar un servidor permanente de pruebas o acceso público para probar la aplicación
  \item \textbf{Latex}\footnote{Sistema de composición de textos.} para la realización del presente documento y la ampliación de conocimientos para futuros textos científicos.
  \item \textbf{Git} para la gestión de versiones y la publicación de código abierto que permita a otros desarrolladores participar.
\end{itemize}

\chapter{Planificación}

\section{Fases}

Dado que el proyecto cuenta con módulos que pueden considerarse subsistemas de un sistema de magnitud mayor. 
Una vez identificados los módulos que conforman el sistema, podemos hacer una separación en fases que nos ayude a evaluar las tareas y
hacer estimaciones. Adicionalmente debemos contemplar un coste extra de integración de los módulos.

\begin{itemize}
	
	\item \textbf{ Módulo Hardware}
	\begin{itemize}
		\item \textbf{Fase 0:} Planteamiento del problema.
		\item \textbf{Fase 1:} Investigar tecnologías implicadas.
		\item \textbf{Fase 2:} Análisis y diseño.
		\item \textbf{Fase 3:} Implementación.
		\item \textbf{Fase 4:} Pruebas.
	\end{itemize}
	
	\item \textbf{Módulo Firmware}
	\begin{itemize}
		\item \textbf{Fase 0:} Planteamiento del problema.
		\item \textbf{Fase 1:} Investigar librerías y lenguaje programación.
		\item \textbf{Fase 2:} Análisis y diseño.
		\item \textbf{Fase 3:} Implementación.
		\item \textbf{Fase 4:} Pruebas.
	\end{itemize}
	
	\item \textbf{Módulo Driver Indi}
	\begin{itemize}
		\item \textbf{Fase 0:} Planteamiento del problema.
		\item \textbf{Fase 1:} Familiarizarme con arquitectura INDI.
		\item \textbf{Fase 2:} Análisis y diseño.
		\item \textbf{Fase 3:} Implementación.
		\item \textbf{Fase 4:} Pruebas.
	\end{itemize}
	
	\item \textbf{Módulo Software} 
	\begin{itemize}
		\item \textbf{Fase 0:} Planteamiento del problema.
		\item \textbf{Fase 1:} Investigar entorno de procesamiento de imágenes.
		\item \textbf{Fase 2:} Análisis y diseño.
		\item \textbf{Fase 3:} Implementación.
		\item \textbf{Fase 4:} Pruebas.
	\end{itemize}
		\item \textbf{Integración} 
		\item \textbf{Pruebas de integración}
		\item \textbf{Documentación}
\end{itemize}



\begin{figure}[h]
\centering
\includegraphics[width=0.7\linewidth]{../images/Fases}
\caption{Desarrollo de los módulos}
\label{fig:Fases}
\end{figure}



\section{Estimación de tiempos}

Dicho desglose lo realizo en el capitulo destinado a cada uno de los módulos, entrado en las tareas realizadas en cada una de las fases.


\section{Recursos humanos}

Dado que el objetivo del proyecto es formarme tanto en la gestión, como en el desarrollo de un proyecto, he sido el único integrante del equipo técnico, a excepción de tareas que requerían manejo de maquinaria avanzada, como cortadora laser, CNC o torno. 

\section{Presupuesto}

Para el presente proyecto se tendrán en cuenta los siguientes costes:

\begin{itemize}
	
	\item \textbf{Costes componentes y hardware necesario}, hacemos un desglose pormenorizado en el capitulo hardware.
	
	\item \textbf{Costes por hora de equipo humano}, a 25\euro la hora.
	
	\item \textbf{Costes asociados a licencias necesarias para publicar o desarrollar
	el software} dado que trabajamos con GNU3 \cite{GNU3}, no tenemos que invertir dinero en este aspecto.

	\item \textbf{Máquina de desarrollo}, portátil valorado en 400\euro con sistema operativo Ubuntu 15.10
	
	\item \textbf{Servidor INDI}, Raspberry Pi, B+ que tiene un coste asociado de 35\euro. Para realizar una instalación de Servidor Indi con el Driver del Ardufocuser.
 
\end{itemize}

\section{Metodología}

\bigskip
En este punto explicamos la metodologia seguida para conseguir  solucionar el problema propuesto, cumpliendo cada uno de los objetivos. 
Podemos diferenciar grandes bloques o subsistemas, que dada la diferente naturaleza debemos analizar, diseñar, implementar y testear por separado, para posteriormente dedicarnos a la integración de unos módulos con los otros.

\bigskip
Es importante mencionar que dado la novedad del proyecto y la gran labor de investigación que necesitaba, tanto a nivel conceptual como a nivel técnico, no se ha tratado con un diseño inicial totalmente estanco y cerrado, se deja abierto en muchos aspectos, y a medida que han surgido nuevas ideas o se han descubierto tecnologías, no se tenga limitación en incorporarlas, siempre con el objetivo de  optimizar la solución mediante refinamiento.

\bigskip
La metodología seguida en el proyecto podemos enmarcarla como \textit{"Iterativa basada en prototipos"}, para cada módulo se crean versiones 
cada vez más optimas y aproximadas a la solución optima.

Esto permite:

\begin{itemize}
	\item Una rápida retroalimentación por parte del cliente y validación de los requisitos. 
	\item Optimizacion de la solución según se tiene un conocimiento más fuerte del dominio, la tecnología manejada.
	\item Los prototipos son rápidos y economicos de construir.
	\item Los prototipos garantizan que el producto se adapta a las necesidades del cliente, dado que el cliente es participe del dasarrollo mediante revisiones continuas. 
	\item Los prototipos constituyen un medio de comunicación mejor que los modelos formales, ya sean diagramas o listados de requisitos. 
	\item Permiten refinar la estimación de tiemposde desarrollo, observando de forma más realista las dificultades que se van dando.  
	
\end{itemize}

\begin{figure}[h]
	\centering
	\includegraphics[width=0.7\linewidth]{../images/analisis1}
	\caption[Iterativa basada en prototipos modular]{Iterativa basada en prototipos modular}
	\label{fig:}
\end{figure}
\newpage

\chapter{Módulo hardware}
<<<<<<< HEAD
\label{cap:hard}


En esta parte nos centraremos en profundizar en las parte más tangible del proyecto, introduciendo la electrónica y 
mecánica del mismo.

Es importante destacar que gracias a las  diferentes plataformas de hardware libre he podido reducir en gran parte el alcance del proyecto, pudiéndome  centrar en implementar las funcionalidades y no tanto en montar todo el marco de trabajo necesario para llegar a ello. 

Tenemos presentes las herramientas de última generación, que están teniendo un gran auge y aceptación 
en la comunidad DIY, como lo son las \textbf{impresoras en 3D} o las \textbf{cortadoras láser}, que agilizan la mecanización y fabricación de las partes mecánicas.


\section{Análisis}

Después de las reuniones con el cliente y el estudio del dominio puedo llegar a extraer los requisitos que debe satisfacer el dispositivo para el objetivo propuesto. Dichos requisitos se desglosan a continuación.

\subsection{Requisitos funcionales}

\begin{itemize}
	\item \textbf{RF-1.}: Controlar motor paso a paso:
	\begin{itemize}
		\item \textbf{RF-1.1.}: Controlar la dirección.
		\item \textbf{RF-1.2.}: Controlar velocidad.
		\item \textbf{RF-1.3.}: Controlar microstepping, para conseguir la máxima resolución en los pasos. \cite{micros}
	\end{itemize} 	
		\item \textbf{RF-2.}: Mostrar información del sistema mediante una pantalla LCD.
		\item \textbf{RF-3.}: Modificar parámetros de forma manual.
	\begin{itemize}
		\item \textbf{RF-3.1.}: Modificar velocidad.
		\item \textbf{RF-3.2.}: Modificar mover \texttt{n} pasos por pulsación.
	\end{itemize}	
	\item \textbf{RF-4.}:Ejecutar funciones de movimiento del motor desde una botonera o control manual.
		\begin{itemize}
			\item \textbf{RF-4.1.}: Ejecutar función de movimiento continuo en la dirección deseada.
			\item \textbf{RF-4.2.}: Ejecutar función de movimiento (\texttt{n} pasos) en la dirección deseada. 
		\end{itemize}
	\item \textbf{RF-5.}: Monitorizar temperatura y mostrar aviso cuando se detecte un cambio.
	\item \textbf{RF-6.}: Controlar límites físicos en el desplazamiento de las partes mecánicas del enfocador.  	
\end{itemize}

\subsection{Requisitos no funcionales}

\begin{itemize}
	\item \textbf{RNF1.}: El movimiento del motor debe tener una precisión alta y una suavidad adecuada. 
	\item \textbf{RNF2.}: Los controles deben ser intuitivos y responder de forma interactiva. 
	\item \textbf{RNF3.}: La pantalla LCD no debe tener excesiva iluminación para conservar la oscuridad.
	\item \textbf{RNF4.}: Fácil instalación y trasporte.
	\item \textbf{RNF5.}: Dispositivo robusto y tolerante a condiciones adversas, sesiones de trabajo prolongadas, así como posibles golpes y movimientos.
	\item \textbf{RNF6.}: Bajo consumo, dado que puede ser alimentado por baterías.
\end{itemize}

\section{Planificación}

\begin{itemize}
	\item \textbf{Fase 0:} \textbf{Planteamiento del problema}.
	\begin{itemize}
		\item \textbf{Reunión inicial con el cliente}: Donde se propone la idea y los puntos clave del proyecto. 
		\item \textbf{Investigar la temática}: Buscar enfocadores comerciales, ver su funcionalidad y posibles nuevas funcionalidades.
		\item \textbf{Segunda reunión con cliente}: Se proponen las funcionalidades más interesantes y se compara con los dispositivos comerciales estudiados, el cliente corrobora que las funciones resultan más o menos interesantes desde su punto de vista como astrónomo amateur.
	\end{itemize}
	\item \textbf{Fase 1:} \textbf{Investigar tecnologías implicadas}.
	\begin{itemize}
		\item \textbf{Estudiar tecnologías que se emplearán}, para ello se estudian las diferentes plataformas de hardware libre, comparando a grandes rasgos sus puntos fuertes, limitaciones, así como otros aspectos como precio, soporte por la comunidad, componentes y repuestos.
		
		\item \textbf{Buscar componentes} a utilizar y comparar precios con los diferentes proveedores.
	\end{itemize}
	\item \textbf{Fase 2:} \textbf{Análisis y diseño}.
	\begin{itemize}
		\item Redactar formalmente los \textbf{requisitos} del sistema. 
		\item Realizar \textbf{esquemáticos} de la electrónica. 
		\item Realizar \textbf{diseños} para el corte de la caja exterior.
	\end{itemize}
	\item \textbf{Fase 3:} \textbf{Implementación}.
	\begin{itemize}
		\item \textbf{Montaje versión 0} del dispositivo usando placas de prototipado y cables.		
		\item \textbf{Corte de piezas} para construir la carcasa del dispositivo. 
		\item \textbf{Soldadura de la PCB}, con todos los componentes, 
		\item \textbf{Montaje prototipo 1}, versión 1, haciendo uso de la caja y la placa PCB. 
	\end{itemize}
	\item \textbf{Fase 4:} \textbf{Pruebas}.
	\begin{itemize}
		\item \textbf{Pruebas de los componentes}, asegurando que no tienen defectos de fábrica.
		\item \textbf{Pruebas prototipo versión 0}.
		\item \textbf{Pruebas prototipo versión 1}.
	\end{itemize} 
\end{itemize}


\subsection{Mockup}


Un primer boceto que se realiza del sistema con los requisitos anteriores es el mostrado en la figura~\ref{fig:diagrama_ardufocuser}.

\begin{figure}[h]
\centering
\includegraphics[width=1\linewidth]{../images/diagrama_ardufocuser}
\caption[Mockup del dispositivo hardware]{Diseño parte externa del dispositivo, donde podemos ver los diferentes controles.}
=======

\bigskip
En esta parte me dedicaré a profundizar en las parte más tangible del proyecto, introduciendo así en la electrónica, 
mecánica.

\bigskip
Es importante destacar que gracias alas  diferentes plataformas de hardware libre he podido reducir en gran parte el alcance del proyecto, pudiendo así centrarme en implementar las funcionalidades y no tanto en montar todo el marco de trabajo necesario para llegar a ello. 

\bigskip
Teniendo presentes las  herramientas de última generación, que están teniendo un gran auge y aceptación 
en la comunidad DIY \cite{DIY} "hágalo usted mismo", como lo son las \textbf{impresoras en 3D} o las \textbf{cortadoras láser}, que agilizan la mecanización y fabricación de las partes mecánicas.

\section{Fases}

\begin{itemize}
	\item \textbf{Fase 0:} Planteamiento del problema.
	\item \textbf{Fase 1:} Investigar tecnologías implicadas.
	\item \textbf{Fase 2:} Análisis y diseño.
	\item \textbf{Fase 3:} Implementación.
	\item \textbf{Fase 4:} Pruebas.
\end{itemize}


\section{Estimación de tiempos}



\section{Requisitos sistema hardware}

EL requisito principal del sistema hardware es actuar sobre la \textbf{ruleta de ajuste del foco} que incorpora el \textbf{ocular del telescopio}. 

Este movimiento tiene que ser perfectamente controlado, con una precisión alta y una suavidad adecuada. 

El movimiento debe responder a las ordenes, que proporcione directamente el usuario por medio de un \textbf{control manual} o bien las ordenes en forma de comandos  que comunique la rutina de enfoque automático que se ejecute.

El control manual permite girar el enfocador en ambos sentidos de dos modos diferentes:
    1. Movimiento continuo, hasta llegar al límite físico.
    2. Movimiento de n pasos.

Modificar la velocidad dentro de un rango válido.
Modificar el número n de pasos por pulsación. 
Establecer posición actual, como posición de foco válido dada la temperatura actual. 
 
Mediante comandos podemos acceder y modificar todos los estados del sistema, respondiendo con un mensaje que informa del cambio producido. 

Para emular la característica de compensación térmica, incorpora un sensor térmico, que permita avisar al usuario de cambios considerables en la temperatura de la óptica, siendo una alarma para revisar el valor del enfoque.  

Debe existir una pantalla que proporcione la máxima información del sistema, (posición de enfoque, velocidad seleccionada, temperatura).


Un primer boceto que se realiza del sistema con los requisitos anteriores es el siguiente:

\begin{figure}[h]
\centering
\includegraphics[width=0.7\linewidth]{../images/diagrama_ardufocuser}
\caption{}
>>>>>>> c9f08dfe66521d4f0dba18e652f93a6a37a333aa
\label{fig:diagrama_ardufocuser}
\end{figure}


<<<<<<< HEAD
\section{Diseño sistema hardware y electrónica: arquiectura del sistema}

Para la elaboración y diseño del sistema hardware hemos seleccionado distintos componentes que hemos agrupado en los siguientes módulos:

\begin{itemize}
 \item Núcleo: Arduino
 \item Módulo de motores
 \item Módulo de pantalla
 \item Módulo de sensores
 \item Módulo de control manual
\end{itemize}

Estos módulos se describen en las siguientes secciones.

En la figura~\ref{fig:diagramaHardware} mostramos un esquemático en el que se integran los distintos componentes hardware.

\begin{figure}[h]
	\centering
	\includegraphics[width=1\linewidth]{../images/diagramaHardware}
	\caption[Diagrama general, sistema hardware]{\textbf{Diagrama general, sistema hardware.} En el diagrama podemos ver los elementos que intervienen y de forma simplificada la relación que existe entre ellos.}
	\label{fig:diagramaHardware}
\end{figure}



\subsection{Núcleo: Arduino}
Para el diseño de la electrónica, partimos como base de una placa de prototipo \textbf{Arduino} \cite{ARDUINO}.

El hardware consiste en una placa de circuito impreso con un microcontrolador, usualmente \textbf{Atmel AVR} \cite{Atmel}, junto con  puertos digitales y analógicos para entrada/salida.

Las características principales de las placas Arduino, que pueden servirnos para decidirnos entre un tipo de placa u otro son las siguientes:
=======
\section{Diseño sistema hardware y electrónica}

Las entrada al sistema hardware de enfoque, proviene de cuatro fuentes.

\begin{enumerate}
\item Los sensores inherentes al dispositivo.
\item Variables de configuración o estados de la sesión anterior. 
\item Controles manuales, ya sean botoneras o potenciómetros.
\item Controles remotos provenientes, que llegan al dispositivo en forma de comandos, por el puerto serie.
\end{enumerate}

Por tanto el sistema hardware, se descompone en los siguiente módulos.

\begin{itemize}

\item Módulo para el control de motores, encargado de actuar sobre el motor, de forma precisa.

Controlar variables y estados tales como, posición actual, velocidad, aceleración, sentido de giro y respetando los módulos de sensores y control. 

\item Módulo de sensores. Se ocupa de manejar la información proveniente de sensores externos, tales como sensores finales de recorrido, sensor de temperatura.

Controlando estados, presencia de un tope que limite el movimiento en uno de los sentidos, temperatura de último enfoque y actual.

\item Módulo control manual, se encarga de leer los diferentes botones y potenciómetros, con los que el usuario puede manejar directamente el dispositivo.
Tenemos estados para cada uno de los botones y los comunica al módulo de motores, cambia la configuración o lo comunica por puerto serie y/o módulo de visualización LCD.

\item Módulo de control remoto, su trabajo es estar a la escucha de los posibles comandos que puedan llegas por puerto serie y comunicarlo al modulo que corresponda. 

\item Modulo de configuración y almacenamiento de sesión, se encarga de manejar todas las variables de configuración, así como almacenarlos en EEPROM, para que se mantengan de forma persistente para la próxima sesión de trabajo

\end{itemize}




\section{Implementación dispositivo hardware}

\subsection{Framework Arduino}
Para el diseño de la electrónica, partimos como base de una placa de prototipo Arduino.

\bigskip
El hardware consiste en una placa de circuito impreso con un microcontrolador, usualmente \textbf{Atmel AVR} \cite{Atmel}, junto con  puertos digitales y analógicos para entrada/salida.

\bigskip
Por otro lado, el software consiste en un entorno de desarrollo (IDE) basado en \textbf{Processing} \cite{Process}  y lenguaje de programación basado en \textbf{Wiring}, así como en el cargador de arranque (bootloader) que es ejecutado en la placa.

Características principales de las placas Arduino, que pueden servirnos para decidirnos entre un tipo de placa u otro.
>>>>>>> c9f08dfe66521d4f0dba18e652f93a6a37a333aa

\begin{itemize}
	\item \textbf{Microcontrolador}, por su arquitectura y su frecuencia de reloj.
	\item \textbf {Tamaño de las memorias},
	\begin{itemize}
<<<<<<< HEAD
		\item{\textbf{Flash}} (espacio del programa) es donde Arduino almacena el \textbf{sketch}. Esta memoria es no volátil y su tamaño oscila entre los 16 kilobytes.
		
		\item{\textbf{SRAM}} Memoria estática de acceso aleatorio, de tipo volátil. Es el espacio donde los sketches (programas) almacenan y manipulan variables al ejecutarse. La información guardada en esta memoria será eliminada cuando Arduino pierda la alimentación. Esta memoria es de uso exclusivo para el programa en ejecución. Su tamaño oscila en torno a los 1024 bytes.
		
		\item{\textbf{EEPROM}} es un espacio de memoria que puede ser utilizado por los programadores para almacenar información a largo plazo. Este tipo de memoria es no volátil, por lo que los datos guardados en ella permanecerán aunque Arduino pierda la alimentación. Permite almacenar configuraciones de la sesión. Es importante resaltar quee tiene un número de escrituras muy limitado por lo cual no se puede usar de forma extensiva. Su tamaño oscila en torno a los 512 bytes.
=======
		
		\item{Flash} (espacio del programa) es donde Arduino almacena el \textbf{sketch}\footnote{Un sketch es el nombre que usa Arduino para un programa.}. Esta memoria es no volátil y su tamaño oscila entre los 16 kilobytes.
		
		\item{SRAM} Static Random Access Memory ó memoria estática de acceso aleatorio,  es de tipo volátil, es el espacio donde los sketches (programas) almacenan y manipulan variables al ejecutarse. La información guardada en esta memoria será eliminada cuando Arduino pierda la alimentación. Esta memoria es de uso exclusivo para el programa en ejecución. Su tamaño oscila entre los 1024 bytes.
		
		\item{EEPROM} es un espacio de memoria que puede ser utilizado por los programadores para almacenar información a largo plazo. Este tipo de memoria es no volátil, por lo que los datos guardados en ella permanecerán aunque Arduino pierda la alimentación. Permite almacenar configuraciones de la sesión, he de decir que tiene un número de escrituras muy limitado por lo cual no se puede usar de forma extensiva. Su tamaño oscila entre los 512 bytes.
>>>>>>> c9f08dfe66521d4f0dba18e652f93a6a37a333aa
		
	\end{itemize}
	\item \textbf {Puertos de entrada y salida I/O}. Son los puertos que disponemos para comunicarnos con los periféricos, ya sean sensores o actuadores. 
	\begin{itemize}
<<<<<<< HEAD
		\item \textbf{Digitales}: Estos \textbf{pines}, tal como podemos intuir trabajan con señales digitales de 5V, (estados LOW y HIGH), existiendo un tipo de especial de ellos denominados \textbf{PWM}, que permiten ``emular" señales analógicas. Otro punto a tener en cuenta es el número de pines que permiten ejecutar \textbf{interrupciones hardware}. 
		
		\item \textbf{Analógicos}: Permiten lectura de valores analógicos que se encuentren el la escala de 0V a 5V. Las entradas analógicas disponen de 10 bits de resolución, lo que proporciona 1024 niveles digitales, lo que a 5V supone una precisión de la medición de +-2,44mV. 
=======
		\item{Digitales} Estos \textbf{pines}\footnote{patilla metálica de un conector multipolar.}, tal como podemos intuir trabajan con señales digitales de 5V, (estados LOW y HIGH), existiendo un tipo de especial de ellos denominados \textbf{PWM}\footnote{señal de modulación por ancho de pulso}, que permite ``emular", señales analógicas. \newline
		Otro punto a tener en cuenta es el número de pines que permiten ejecutar  \textbf{interrupciones hardware}. 
		
		\item{Analógicos}, permiten lectura de valores analógicos que se encuentren el la escala de 0V a 5V, las entradas analógicas disponen de 10 bits de resolución, lo que proporciona 1024 niveles digitales, lo que a 5V supone una precisión de la medición de +-2,44mV. 
		
>>>>>>> c9f08dfe66521d4f0dba18e652f93a6a37a333aa
	\end{itemize}
	
	\item \textbf{Otros factores}
	\begin{itemize}
<<<<<<< HEAD
		\item \textbf{Factores de forma}: Características como las dimensiones, el peso, la forma.
		\item \textbf{Conectores auxiliares}: Tamaño y forma de los conectores USB, ya sea de Tipo A, Tipo B o mini-USB. 
		\item \textbf{Consumo}: Muy importante si alimentamos la placa con  baterías. 
=======
		\item \textbf{Factores de forma}, características como las dimensiones, el peso, la forma.
		\item \textbf{Conectores auxiliares}, Tamaño y forma de los conectores USB, ya sea de Tipo A, Tipo B o mini-USB. 
		\item \textbf{Consumo}, muy importante si alimentamos la placa con  baterías. 
		
>>>>>>> c9f08dfe66521d4f0dba18e652f93a6a37a333aa
	\end{itemize}	
\end{itemize}


\begin{figure}[h]
<<<<<<< HEAD
	\centering
	\includegraphics[width=1\linewidth]{../images/arduino_data}
	\caption[Diagrama de pines de Arduino]{\textbf{Diagrama de pines de Arduino}, muestra los diferentes pines de entrada/salida de la placa Arduino UNO, así como sus conexiones con el microcontrolador. También podemos ver los pines de alimentación a 5V y 3V3. \textbf{Fuente: }\cite{ARDUINO}}
	\label{fig:arduino_data}
\end{figure}

Para este proyecto hemos seleccionado la placa \textbf{Arduino UNO} (figura~\ref{fig:arduino_data}) en base a los siguientes criterios:

\begin{itemize}
	\setlength\itemsep{0em}
	\item \textbf{Precio reducido}, que oscila en torno a los los 20 \euro.
	\item \textbf{Potencia del procesador y memoria}, suficiente para soportar el firmware a implementar.
	\item \textbf{Conector USB tipo B}, al ser de buen tamaño, hace la conexión robusta (importante en telescopios donde existe movimiento o posibles tensiones en los cables).
	\item \textbf{Número de pines}, suficiente para conectar todos los periféricos previstos.
	\item \textbf{Características especiales}, (PWM y interrupciones hardware). 
	\item \textbf{Comunicación I2C}. 
\end{itemize}

\subsection{Módulo de motores}

Para permitir el control del motor de un paso a paso, usaremos un Pololu A4988 \cite{pololu} (figura~\ref{fig:pololu}). Las características más reseñables de tal controlador y que los hacen destacar frente a sus competidores son las siguientes:

\begin{itemize}
	\item{Interfaz simple} para controlar pasos y dirección.
	\item{Cinco resoluciones de paso diferentes}, paso completo, $1/2$ paso, $1/4$ de paso, $1/8$ de paso y $1/16$ de paso. 
	\item{Control de corriente} de paso ajustable.
	\item{Protecciones} contra sobretensiones y sobrecalentamientos.
=======
\centering
\includegraphics[width=0.8\linewidth]{../images/arduino_data}
\caption{Arduino Pinout Diagram}
\label{fig:arduino_data}
\end{figure}


\newpage

Me decido por Arduino UNO en base a los siguientes criterios:

\begin{itemize}
	\item{Precio reducido}, que oscila entre los 20 \euro
	\item{Potencia del procesador y menoría}, suficiente para soportar el firmware.
	\item{Conector USB tipo B}, al ser de buen tamaño, hace la conexión robusta.
	\item{Número de pines} suficiente para conectar todos los periféricos.
	\item{Características especiales}, (PWM y interrupciones hardware). 
	\item{I2C} compatible. 
\end{itemize}


\subsection{Módulo de motores}

Para permitir el control del motor de un paso a paso,  Pololu A4988 \cite{pololu},
las características más reseñables de tal controlador y que los hacen destacar frente a sus competidores.

\begin{itemize}
	\item{} Interfaz simple para controlar pasos y dirección.
	\item{} Cinco resoluciones de paso diferentes, paso completo, medio paso, cuarto de paso, octavo de paso, dieciseisava parte de paso. 
	\item{} Control de corriente de paso ajustable.
	\item{} Protecciones contra sobretensiones y sobrecalentamientos.
>>>>>>> c9f08dfe66521d4f0dba18e652f93a6a37a333aa
\end{itemize}


\begin{figure}[h]
<<<<<<< HEAD
	\centering
	\includegraphics[width=0.8\linewidth]{../images/pinout_pololu}
	\caption[Diagrama Pololu A4988]{\textbf{Pololu A4988} - Vista del controlador de motor paso a paso utilizado. Podemos observar las diferentes conexiones, puertos de control (STEP y DIR), alimentación del controlador (VDD), alimentación del motor (VMOT), salidas para el motor (1A, 1B, 2A, 2B) y pines para ajuste de la resolución de pasos (MS1, MS2, MS3) \textbf{Fuente:} \cite{pololu} }
	\label{fig:pololu}
\end{figure}

El motor empleado para nuestro sistema es un motor paso a paso \textbf{NEMA 17} / 3.2Kg/cm, de uso muy habitual, en el montaje de impresoras 3D. Sus características principales son las siguientes:

\begin{itemize}
	
	\item \textbf{Tamaño:} 42.3x48mm (sin incluir el eje) (NEMA 17)
	\item \textbf{Peso:} 350 gramos (13 oz)
	\item \textbf{Diámetro del eje:} 5 mm "D"
	\item \textbf{Longitud del eje:} 25 mm
	\item \textbf{Pasos por vuelta:} 200 (1.8\degree/paso)
	\item \textbf{Corriente:}  1.2 Amperios por bobinado
	\item \textbf{Tensión:} 12 V
	\item \textbf{Resistencia:} 3.3 Ohm por bobina
	\item \textbf{Torque:} 3.2 kg/cm (44 oz-in)
	\item \textbf{Inductancia:} 2.8 mH por bobina 
=======
\centering
\includegraphics[width=0.3\linewidth]{../images/pololu}
\caption{Pololu A4988 Stepper Motor Driver Carrier}
\label{fig:pololu}
\end{figure}

\newpage

El motor empleado es un Motor paso a paso NEMA 17 / 3.2Kg/cm, de uso muy habitual, 
en el montaje de impresoras 3D,  

\bigskip
Sus características son las siguientes.

\begin{itemize}
	\item{Tamaño:} 42.3x48mm, sin incluir el eje (NEMA 17)
	\item{Peso:} 350 gramos (13 oz)
	\item{Diámetro del eje:} 5 mm "D"
	\item{Longitud del eje:} 25 mm
	\item{Pasos por vuelta:} 200 (1,8º/paso)
	\item{Corriente:}  1.2 Amperios por bobinado
	\item{Tensión:} 4 V
	\item{Resistencia:} 3.3 Ohm por bobina
	\item{Torque:} 3.2 kg/cm (44 oz-in)
	\item{Inductancia:} 2.8 mH por bobina 
>>>>>>> c9f08dfe66521d4f0dba18e652f93a6a37a333aa
\end{itemize}

\subsection{Módulo de pantalla}

<<<<<<< HEAD
Para visualizar los estados del sistema fácilmente, se incorporará una pequeña pantalla LCD. Uno de los displays LCD más corrientes son los alfanuméricos de 16x2 o 16x4 caracteres.

Para su utilización en un principio se estima controlarla mediante los pines digitales usando 4 señales digitales D7, D6, D5, y D4, junto con dos pines adicionales de control tal y como se puede ver ver en las figuras~\ref{circuito1} y \ref{circuito2} del apéndice~\ref{ap:diagramas}.


En una fase de rediseño optimizamos el circuito incorporando un \textbf{interfaz I2C}, reduciendo las señales a solo dos (figura~\ref{fig:lcd_i2c}). 

\begin{figure}[h]
	\centering
	\includegraphics[width=0.7\linewidth]{../images/lcd_i2c}
	\caption[Diagrama conexión pantalla LCD con interfaz I2C]{Diagrama conexión pantalla LCD con interfaz I2C}
	\label{fig:lcd_i2c}
\end{figure}


\subsection{Módulo de sensores}
=======
Para visualizar los estados del sistema fácilmente, el sistema incorpora una pequeña pantalla LCD. 

Los más corrientes son los display LCD, de 16x2 o 16x4.

Para su utilización en un principio se estima controlarla mediante los pines digitales usando 4 señales digitales D7, D6, D5, y D4, junto con dos pines adicionales de control. Como podemos ver en los primeros diagramas.

En una fase de rediseño optimizamos el circuito incorporando una interfaz I2C \cite{I2C}, reduciendo las señales a solo dos. 


\subsection{Módulo sensores}
>>>>>>> c9f08dfe66521d4f0dba18e652f93a6a37a333aa

El dispositivo incorpora dos tipos de sensores:

\begin{itemize}
<<<<<<< HEAD
	\item \textbf{Sensor térmico:} Hacemos uso de un sensor LM35 \cite{LM35}. Es un sensor de temperatura con una precisión calibrada de 1 \grad C. Su rango de medición abarca desde -55\grad C hasta 150 \grad C. La salida es lineal y cada grado Celsius equivale a 10 mV.
	
	
	En una fase de rediseño se reemplaza el sensor por uno del tipo DHT11 \cite{dth11}, que mide temperatura en el rango -10 \grad C y 50\grad C. Adicionalmente tambien mide humedad, en un rango del 20\% al 95\% (figura~\ref{fig:dht11}). Usa un protocolo de un solo hilo digital, (1-wire). 
	
	
	\begin{figure}[h]
		\centering
		\includegraphics[width=0.5\linewidth]{../images/dht11}
		\caption[Diagrama sensor de temperatura y humedad DHT11]{Diagrama de conexión del sensor de temperatura y humedad DHT11}
		\label{fig:dht11}
	\end{figure}
	
	\item \textbf{Endstop sensor}, o contactos fines de carrera. Son interruptores del tipo todo o nada, que se activan cuando se llega a un límite físico. 
	
\end{itemize}


\subsection{Módulo de control manual}

Para permitir el control manual del dispositivo usamos los siguientes componentes:

\begin{itemize}
	\item \textbf{Potenciómetros}: para regular algunos valores, en concreto usamos 10 kOhms.
	\item \textbf{Botones}: en los primeros prototipos una botonera para activar las distintas funciones.
	\item \textbf{Wii Nunchuck}: se utiliza un mando de la consola Wii, para de una forma ergonómica (y barata) activar los comandos, mediante el botón zeta y algunos pulsadores (figura~\ref{fig:nunchuck}). Este mando totalmente compatible con Arduino dado que funciona bajo el protocolo I2C.
		
	\begin{figure}
		\centering
		\includegraphics[width=1\linewidth]{../images/nunchuck}
		\caption[Diagrama de conexiones Wii Nunchuck]
		{Diagrama de conexiones Wii Nunchuck: observamos que contamos con las señales (SDA y SCL) y se alimenta a 3.3V  \textbf{Fuente:} \cite{nunchuck_diagram}}
		\label{fig:nunchuck}
	\end{figure}
	 
\end{itemize}


\section{Presupuesto}

En todo momento se ha tratado de optimizar el precio del desarrollo buscando componentes de reducido precio comparando entre los diferentes proveedores. Sin embargo es importante destacar que en componentes críticos (núcleo, controladora de motor) no es adecuado arriesgarse a introducir elementos de proveedores dudosos. Es por ello que se usa placa de Arduino UNO Oficial así como una controladora de motor Pololu. El desglose costes para cada componente hardware se muestra en la tabla~\ref{tabla_costes_hardware} (algunos precios pueden variar ligeramente).

\begin{table}[h!]
	\centering
	
	\begin{tabular}{|l|r|}
		\hline
		\textbf{Componente}                  				& \textbf{Precio (\euro)} \\ \hline\hline
		Arduino UNO        									&                      22 \euro \\ \hline
		Motor PaP Nema 17         							&                      10 \euro \\ \hline
		Controladora Motor (Pololu o similar) 				&                      7  \euro \\ \hline
		Pantalla LCD + controladora I2C           			&                      2.5 \euro \\ \hline
		Nunchaku Wii (compatible)        					&                      4  \euro \\ \hline
	    Sensor Temperatura, resistencias, condensador, led  & 					   1  \euro \\ \hline
		Cables, soldadura, pines, placas, etc.               &                      4  \euro \\ \hline
		Potenciómetros            							&                      2  \euro \\ \hline 
		Caja            									&                      5  \euro \\ \hline 
		Clavijas aviador            						&                      3  \euro \\ \hline 
		Fines de carrera            						&                      2  \euro \\ \hline \hline
		\textbf{Total}                  					&            \textbf{62.5 \euro} \\ \hline
	\end{tabular} 
	\caption[Lista de componentes y costes]{Lista de componentes y desglose de precios}
	\label{tabla_costes_hardware}	
\end{table}


\section{Implementación hardware}

La implementación del dispositivo hardware se puede definir como un proceso totalmente iterativo, en el marco del prototipado. Comenaremos siempre con pruebas de cada unos de los módulos por separado y trabajando con placas que permiten quitar y cambiar la distribución de los componentes.

Para realizar el diseño del circuito se utiliza la herramienta \textbf{Frizing} \cite{frizing} (figura~\ref{fig:fritzing}), de la que llama la atención su fácil uso dado que esta orientada al prototipado. Cuenta un repertorio de componentes genéricos de Arduino.

\begin{figure}[h]
	\centering
	\includegraphics[width=1\linewidth]{../images/frizing}
	\caption[Diseño circuito con Frizing]{Diseño de la circuitería haciendo uso de la herramienta \textbf{Frizing}}
	\label{fig:fritzing}
\end{figure}

Una vez definido el esquemático pasamos a la fase de soldadura. Usaremos la tecnología de agujeros pasantes o \textbf{THT}, donde el circuito se monta aprovechando los orificios que trae una placa, y mediante pequeños puentes se van uniendo los agujeros creando pistas. En el diseño del circuito nos tenemos que asegurar que las pistas no se crucen (figura~\ref{fig:prototipoAgujeros}).  

\begin{figure}
	\centering
	\includegraphics[width=1\linewidth]{../images/circuito01}
	\caption[Fase de soldadura de la placa]{\textbf{Fase de soldadura de la placa.} Haciendo uso de un soldador de estaño y un polimetro para comprobar continuidad, procedo a soldar un prototipo de PCB en una placa de baquelita perforada}
	\label{fig:prototipoAgujeros}
\end{figure}



\subsection{Prototipos}

Durante el proceso iterativo se han implementado cuatro versiones progresivamente más avanzadas. Detallo las características más importantes de cada uno de ellos.

\begin{itemize}
	\item \textbf{Versión 0:} Usando una placa de prototipado que permite poner y quitar los componentes sin realizar soldaduras. Se usa \textbf{LCD Keypad Shield} \cite{lcd_keypad} (figura~\ref{fig:foto_prototipo}) para reducir la dificultad inicial. LCD Keypad Shield permite utilizar una pantalla LCD y una botonera acoplándolo dicho módulo sobre Arduino MEGA. Podemos ver el diagrama seguido en la figura~\ref{circuito1} del apéndice~\ref{ap:diagramas}. 
	
	\begin{figure}
		\centering
		\includegraphics[width=1\linewidth]{../images/proto_recorte}
		\caption[Foto del primer prototipo]{Foto del primer prototipo, montada en placa de prototipado. Carece de carcasa y todas las conexiones se hacen usando cables.}
		\label{fig:foto_prototipo}
	\end{figure}
	
	
	\item \textbf{Versión 1:} Se realiza una completa reestructuración de los componentes, se modifica  \textbf{LCD Keypad Shield} por una pantalla LCD (interfaz de conexión clásico), una \textbf{botonera analógica} (figura~\ref{fig:botonera}) y conectores RS232 \cite{rs232}. Se puede ver el diagrama correspondiente en la figura~\ref{circuito2} del apéndice~\ref{ap:diagramas}. En la figura~\ref{fig:carcasaAluminio} se muestra la primera carcasa de aluminio realizada para el prototipo (mecanizada manualmente). Pese al buen acabado de la misma, hacen falta muchas herramientas y tiempo para construir dicha carcasa.
	
		\begin{figure}[h]
			\centering
			\includegraphics[width=0.75\linewidth]{../images/botonera_manual}
			\caption[Foto de la botonera manual]{Foto de la botonera manual analógica}
			\label{fig:botonera}
		\end{figure}
		
		
	\item \textbf{Versión 2:} Se rediseña modificando interfaz de la pantalla a I2C. Se intercambia la botonera manual por un mando \textbf{Wii Nunchuck} reciclado (tambien por I2C). Esto permite que se reduzca la cantidad de cables usados.  Podemos ver el circuito completo en en la figura~\ref{circuito3} del apéndice~\ref{ap:diagramas}. 
	
	\item \textbf{Versión 3:} Modificamos sensor de temperatura analógico LM35 por DHT11 (digital) (figura~\ref{circuito3} del apéndice~\ref{ap:diagramas}).
	
	\begin{figure}[h]
		\centering
		\includegraphics[width=0.9\linewidth]{../images/ardufocuser}
		\caption[Prototipo en caja de aluminio]{\textbf{Prototipo en caja de aluminio}, con un acabado muy sólido.}
		\label{fig:carcasaAluminio}
	\end{figure}
	
\end{itemize}









\section{Diseño de la carcasa y piezas mecánicas}
=======
	\item Sensor térmico: Hacemos usos de  LM35 \cite{LM35},es un sensor de temperatura con una precisión calibrada de 1 \grad C. Su rango de medición abarca desde -55 \grad C hasta 150 \grad C. La salida es lineal y cada grado Celsius equivale a 10 mV.
	\item \textbf{Bumper sensor}, o contactos fines de carrera, son interruptores todo o nada, que se activan cuando se llega a límite físico. 
\end{itemize}


\subsection{Módulo control manual}

\begin{itemize}
	\item Potenciómetros, para regular algunos valores, en concreto usamos 10 kOhms.
	
	\item Botones, para unos primeros prototipos una botonera, para activar las funciones que me interesan.
	
	\item \textbf{Wii Nunchuck}, reutilizo un mando de la consola Wii, para de una forma ergonómica activar los comandos, mediante la zeta y algunos pulsadores, este mando es totalmente compatible dado que funciona bajo el protocolo I2C.
	 
\end{itemize}

\section{Presupuesto}

\begin{itemize}
	\item Arduino UNO: 22\euro
	\item Motor PaP Nema 17: 10\euro
	\item Controladora Motor (Pololu o similar): 7\euro
	\item Pantalla LCD + controladora I2C: 2.5\euro
	\item Nunchaku Wii (compatible): 4\euro
	\item Sensor Temperatura, resistencias, condensador, led: 1\euro
	\item Cablecillos, soldadura, pines, placas, etc: 4\euro
	\item Potenciómetros: 2\euro
	\item Caja: 5\euro
	\item Clavijas aviador: 3\euro
	\item Fines de carrera: 2\euro
	
	
\end{itemize}
\textbf{Total: 62.5\euro}

\subsection{Integración de módulos}
\bigskip
Tras trabajar en cada uno de los módulos de forma independiente una visualización más o menos precisa es la siguiente:

\begin{figure}[h]
	\centering
	\includegraphics[width=0.9\linewidth]{../images/diagramaHardware}
	\caption{}
	\label{fig:diagramaHardware}
\end{figure}

Aún no tengo claro detalles como la alimentación de cada uno de los dispositivos,  exactamente las conexiones que va a requerir cada periférico ni su ubicación.

\bigskip
Es importante contabilizar número de pines que necesita cada periférico, si son digitales o analógico, pwd o hacen uso de interrupciones hardware. 

\bigskip
Con ello nos podemos hacer una idea aproximada del sistema y de los recursos que necesitamos.


\newpage
\section{Implementación hardware}

La implementación del dispositivo hardware, se puede definir como un proceso totalmente iterativo, en el marco del prototipado, comenzando por las pruebas de cada unos de los módulos, y trabajando con placas que permiten remover y cambiar la distribución de los componentes.



\begin{figure}[h]
\centering
\includegraphics[width=0.7\linewidth]{../images/proto_recorte}
\caption{}
\label{fig:prototipoArdufocuser}
\end{figure}

\bigskip
Tras muchas pruebas y una vez claras las partes fundamentales de las que consta el dispositivo, así como su integración, se pasa a formalizar el diseño, buscando la elegancia, la máxima limpieza en la distribución de los componentes, así como la economía de espacio.

\begin{figure}[h]
	\centering
	\includegraphics[width=1\linewidth]{../images/circuito}
	\caption{}
	\label{fig:prototipoArdufocuser}
\end{figure}


Una vez perfectamente definido el esquemático pasamos a la fase de soldadura.

\newpage


\section{Diseño carcasa y piezas mecánicas}
>>>>>>> c9f08dfe66521d4f0dba18e652f93a6a37a333aa

Para dar soporte a toda la electrónica, así como hacer el dispositivo robusto y fácilmente transportable,
se decide diseñar una carcasa que permita cubrir toda la electrónica y concentrar todos los conectores y botones. 

<<<<<<< HEAD
Para ello debemos hacer una buena estimación del espacio y la distribución de los elementos. Otro punto a tener en cuenta es que debemos permitir la fácil apertura y manipulación del los elementos internos, con el fin de poder realizar mejoras o simplemente permitir a los interesados abrir el dispositivo y conocer sus interioridades. 

En un primer momento se optó por adquirir cajas prefabricadas a un proveedor externo y se estudiaron diferentes diseños.


El acabado era excelente, pero se encontraron varios inconvenientes:

\begin{itemize}
	\item \textbf{Dependencia de un proveedor}, que facilite siempre el mismo modelo de caja.
	\item \textbf{Diseño poco personalizable}, debemos asegurarnos de que el dispositivo se adapte perfectamente a las dimensiones.
	\item  \textbf{Difícil mecanización}, dado el material metálico, realizar agujeros pasantes, soldaduras, grabados, roscas etc, se convierte en tareas tediosas. 
\end{itemize}

Por estos motivos decidimos plantear la alternativa de fabricar una carcasa propia. Dado que contamos con una \textbf{cortadora láser} controlada numéricamente decidimos crear una carcasa totalmente propia a partir de laminas de \textbf{metacrilato} de 3mm de espesor (figura~\ref{fig:cajaMetacrilato}).

\begin{figure}
	\centering
	\includegraphics[width=1\linewidth]{../images/cajas}
	\caption[Foto de la carcasa en metacrilato]{\textbf{Resultado después del corte de la carcasa en metacrilato}. Previamente se realizan ensayos en materiales más baratos como \textbf{DM} \cite{dm}.}
	\label{fig:cajaMetacrilato}
\end{figure}
=======
Para ello debemos hacer una buena estimación del espacio y la distribución de los elementos, otro punto a tener en cuenta es que debemos permitir la fácil apertura y manipulación del los elementos internos, con el fin de poder realizar mejoras o simplemente permitir a los interesados abrir el dispositivo y conocer sus interioridades. 

En un primer momento se optó por adquirir cajas prefabricadas a un proveedor externo y estudié diferentes diseños.

\begin{figure}
\centering
\includegraphics[width=0.7\linewidth]{../images/ardufocuser}
\caption{}
\label{fig:ardufocuser}
\end{figure}


El acabado era excelente, pero me encontraba varios inconvenientes, por los cueles decidí plantearme la alternativa de fabricar mi propia carcasa "Ardufocuser". 

\begin{itemize}
	\item Dependencia de contar con un proveedor que facilite siempre el mismo modelo de caja.
	\item Diseño poco personalizable, debo asegurarme que el dispositivo se adapte perfectamente a las dimensiones.
	\item  Difícil mecanización, dado el material metálico, realizar agujeros pasantes, soldaduras, grabados, roscas etc, se convertían en tareas tediosas. 
\end{itemize}

Dado que contamos con una cortadora láser de última generación decidimos, crear una carcasa totalmente propia, a partir de laminas de "metacrilato".
>>>>>>> c9f08dfe66521d4f0dba18e652f93a6a37a333aa

Ventajas de esta opción:

\begin{itemize}
	\item Bajo coste de los materiales. 
	\item Totalmente personalizada y a medida.
<<<<<<< HEAD
	\item Replicable al 100\%.
\end{itemize}

En el apéndice~\ref{ap:caja} se muestran algunas imágenes del diseño de la caja con un programa de modelado 3D así como los el fichero SVG con los planos de las piezas utilizado para el corte de la misma.


\section{Pruebas sobre el dispositivo hardware}

La sección de pruebas las dividimos en varias partes, pasando pruebas de concepto de cada uno de los componentes y módulos individuales, pruebas de integración, pruebas de rendimiento. 

\begin{itemize}
	\item \textbf{Pruebas de concepto y de componentes:} Es necesario revisar que los componentes que hemos adquirido no tienen ningún defecto que de ser detectados más adelante pueden ser costosos de cambiar. En las tablas~\ref{tab:prueba1},~\ref{tab:prueba2} y ~\ref{tab:prueba3} se muestran algunos de los casos de prueba realizados. 
	
	\begin{table}[h]
		\centering

		\begin{tabular}{|l|l|}
			\hline
			ID caso de prueba             &  1 \\ \hline
			Nombre prueba                 &  Prueba lectura potenciómetros \\ \hline
			Autor de la prueba            &  José Miguel López \\ \hline
			Responsable diseño            &  José Miguel López \\ \hline
			Pasos y condiciones ejecución &  \begin{tabular}[c]{@{}l@{}}
													- Compilar y cargar sketch (\ref{lst:potenciometro_test_code})\\
													- Conectar placa a la alimentación \\
													- Variar resistencia en los potenciómetros.							
											 \end{tabular} \\ \hline
			Resultado deseado             & \begin{tabular}[c]{@{}l@{}}
											Se debe pintar por pantalla valores \\
											de 0 a 100 proporcionales a la resistencia. \\
										\end{tabular} \\ \hline
		
			Resultado obtenido            &  \begin{tabular}[c]{@{}l@{}}
												Se pintan por pantalla valores \\
												de 0 a 100 proporcionales a la resistencia. \\
										     \end{tabular} \\ \hline
			
			Estado caso de prueba         &  Éxito\\ \hline
			Errores asociados             &  Ninguno\\ \hline
			Comentario                    &  \\ \hline
		\end{tabular}
				\caption{Caso de prueba, lectura de potenciómetros}
		\label{tab:prueba1}
	\end{table} 
	
	
	\begin{table}[h]
		\centering

		\begin{tabular}{|l|l|}
			\hline
			ID caso de prueba             &  2 \\ \hline
			Nombre prueba                 &  Prueba mover motor velocidad constante \\ \hline
			Autor de la prueba            &  José Miguel López \\ \hline
			Responsable diseño            &  José Miguel López \\ \hline
			Pasos y condiciones ejecución &  \begin{tabular}[c]{@{}l@{}}
				- Compilar y cargar sketch (\ref{lst:motor_test_code})\\
				- Conectar placa a la alimentación \\						
			\end{tabular} \\ \hline
			Resultado deseado             & \begin{tabular}[c]{@{}l@{}}
				El motor debe girar a velocidad constante \\ y de forma uniforme.
			\end{tabular} \\ \hline
			
			Resultado obtenido            &  \begin{tabular}[c]{@{}l@{}}
				El motor gira a velocidad constante y de \\ forma uniforme.
			\end{tabular} \\ \hline
			Estado caso de prueba         &  Éxito\\ \hline
			Errores asociados             &  Ninguno\\ \hline
			Comentario                    &  \\ \hline
		\end{tabular}
				\caption{Caso de prueba, mover motor a velocidad constante}
		\label{tab:prueba2}
	\end{table} 
	
	
	\begin{table}[h]
		\centering

		\begin{tabular}{|l|l|}
			\hline
			ID caso de prueba             &  3 \\ \hline
			Nombre prueba                 &  Prueba mando Nunchuck \\ \hline
			Autor de la prueba            &  José Miguel López \\ \hline
			Responsable diseño            &  José Miguel López \\ \hline
			Pasos y condiciones ejecución &  \begin{tabular}[c]{@{}l@{}}
												- Compilar y cargar sketch (\ref{lst:nunchuck_test_code})\\
												- Conectar placa a la alimentación \\
												- Pulsar sobre los diferentes botones del mando.						
											\end{tabular} \\ \hline
			Resultado deseado             & \begin{tabular}[c]{@{}l@{}}
												Se debe pintar en la pantalla LCD: \\
												- LEFT: Cuando movemos la palanca \\hacia la izquierda.\\
												- RIGHT: Cuando movemos la palanca \\hacia la derecha.\\
												- Z: Al pulsar el botón Z.\\
												- C: Al pulsar el botón C.\\
											\end{tabular} \\ \hline
											
			Resultado obtenido             & \begin{tabular}[c]{@{}l@{}}
				Se pinta en la pantalla LCD: \\
				- LEFT: Cuando movemos la palanca \\ hacia la izquierda.\\
				- RIGHT: Cuando movemos la palanca \\ hacia la derecha.\\
				- Z: Al pulsar el botón Z.\\
				- C: Al pulsar el botón C.\\
			\end{tabular} \\ \hline
			Estado caso de prueba         &  Éxito\\ \hline
			Errores asociados             &  Ninguno\\ \hline
			Comentario                    &  \\ \hline
		\end{tabular}
				\caption{Caso de prueba, detectar pulsaciones en Wii Nunchuck}
		\label{tab:prueba3}
	\end{table} 
	
		   
	\item \textbf{Pruebas de rendimiento:} Comprobar que el sistema es tolerante a fatigas, no se sobrecalienta ningún componente y continua siendo estable  después de un tiempo prolongado. En la tabla~\ref{tab:prueba4} se muestran uno de los casos de prueba realizados. 
	
	
		\begin{table}[h]
			\centering

			\begin{tabular}{|l|l|}
				\hline
				ID caso de prueba             &  4 \\ \hline
				Nombre prueba                 &  Prueba temperatura motor \\ \hline
				Autor de la prueba            &  José Miguel López \\ \hline
				Responsable diseño            &  José Miguel López \\ \hline
				Pasos y condiciones ejecución &  \begin{tabular}[c]{@{}l@{}}
					- Compilar y cargar sketch (\ref{lst:motor_test_code})\\
					- Conectar placa a la alimentación \\
					- Dejar el circuito conectado durante 120 minutos.					
				\end{tabular} \\ \hline
				Resultado deseado             & \begin{tabular}[c]{@{}l@{}}
					La temperatura del motor debe ser normal \\
					no debe existir sobrecalentamiento.
				\end{tabular} \\ \hline
				
				Resultado obtenido            &  \begin{tabular}[c]{@{}l@{}}
					El motor presenta una temperatura \\ por encima de la normal \\
				\end{tabular} \\ \hline
				Estado caso de prueba         &  Error \\ \hline
				Errores asociados             &  Se puede quemar el motor con un uso más prolongado. \\ \hline
				Comentario                    &   \begin{tabular}[c]{@{}l@{}}
													Se diagnostica posible causa, \\
													 mal reglaje en la alimentación del motor. 
											\end{tabular} \\ \hline
			\end{tabular}
						\caption{Caso de prueba, temperatura de motor tras un tiempo de funcionamiento prolongado}
			\label{tab:prueba4}
		\end{table} 
\end{itemize}





=======
	\item  Replicable 100\textdiscount
\end{itemize}


% Meter foto caja metacrilato.


\section{Pruebas sobre el dispositivo hardware.}


La sección de pruebas las dividimos en varias partes, pruebas de concepto de cada uno de los componentes y módulos individuales, pruebas de integración, pruebas de rendimiento. 




\begin{figure}[h]
\centering
\includegraphics[width=1\linewidth]{../images/travis_ci}
\caption{}
\label{fig:travis_ci}
\end{figure}

\begin{figure}[h]
	\centering
	\includegraphics[width=0.7\linewidth]{../images/diagramaGeneral}
	\caption{}
	\label{fig:diagramaHardware}
\end{figure}


\begin{figure}[h]
	\centering
	\includegraphics[width=0.7\linewidth]{../images/circuito}
	\caption{}
	\label{fig:diagramaHardware}
\end{figure}
>>>>>>> c9f08dfe66521d4f0dba18e652f93a6a37a333aa



\section{Implementación firmware}

En este capítulo se expone el trabajo relacionado con la programación a bajo nivel realizada sobre la electrónica propuesta en el capítulo Módulo Hardware.
Dado que la electrónica se basa en la plataforma Arduino, tal programación se realiza usando su propio lenguaje,  implementado en C/C++.

Este lenguaje se llama \textbf{Wiring} y se define a sí mismo como:

\textit{Entorno de programación de entradas/salidas de código abierto para explorar las artes electrónicas, los medios materiales, la enseñanza y el aprendizaje de la programación informática y creación de prototipos con electrónica.
}


\subsection{Diseño y Análisis}

Pasamos a determinar los requisitos que debe satisfacer nuestro firmware.
El requisito principales es claro, y consiste en ser capaz de controlar y manejar toda la información todos los periféricos  los datos que arrojan, ello se puede desglosar:

\begin{itemize}
	\item \textbf{Motor}, controla el movimiento, la velocidad, y la posición. 
	\item \textbf{Botones y potenciometos}, controla las pulsaciones y el cambio en el valor de los potenciometros, para controlar el dispositivo de forma manual. 
	\item \textbf{Pantalla LCD}, mostrar toda la información por pantalla.
	\item \textbf{Sensores}, recabar información del estado del dispositivo, temperatura, fines de recorrido.
	\item \textbf{Sesión persistente}, almacenar información de trabajo a otra.
	\item \textbf{API Serie}, interpretar y manejar comando vía mediante comunicación puerto serie, para permitir controlar el dispositivo desde un host.
\end{itemize}

Todo ello debe implementarse con una buena calidad permitiendo añadir más módulos y que se pueda cambiar fácilmente el repertorio de instrucciones de la API.

\subsection{Arquitectura firmware}


Para la programación del firmware sigo un diseño donde diferencio 4 bloques.


\begin{itemize}
	\item \textbf{Initializer}: Método \textbf{init} de la api de Arduino, que se ejecuta en primera instancia, prepara todo el entorno de ejecución, declarando las entradas, salidas,  reservando memoria, iniciar interrupciones, así como ejecutar rutinas como cargar sesión anterior o crear instancias a modo de singleton para manejador los diferentes periféricos. 
	\item \textbf{Controlador principal}: Bucle principal que se ejecuta periódicamente (el periodo no se puede modificar y lo marca la velocidad del reloj del micro), por lo tanto no es adecuado ejecutar rutinas con restricciones temporales fuertes, se corresponde con el método \textbf{loop}, se ocupa de manejar la mayoria de las entradas y salidas.
	\item \textbf{Interrupciones software}, se ejecutan en un periodo marcado por un timer, similar al loop pero con periodos fijo, se usa para rutinas con restricciones temporales fuertes.
	item \textbf{Interrupciones hardware}, se ejecutan rutinas ligadas directamente a eventos de entrada y salida en alguno de los pines habilitados para tal fin. 
\end{itemize}

Podemos ver un diagrama simplificado en la siguiente imagen.

\begin{figure}
\centering
\includegraphics[width=0.7\linewidth]{"../images/Arquitectura firmware"}
\caption{}
\label{fig:Arquitecturafirmware}
\end{figure}

\subsection{Módulo motores}

Para el control del motor, dado que debemos hacerlo con una gran precisión, hacemos uso de la libreria 
\textbf{AccelStepper} \cite{accelstp} 

\subsection{Módulo control remoto}

Para realizar el control remoto, hago uso de la comunicación serie que incorpora la placa Arduino.


El puerto serie envía la información mediante una secuencia de bits. Para ello se necesitan al menos dos conectores para realizar la comunicación de datos, RX (recepción) y TX (transmisión). 

En ocasiones veréis referirse a los puertos de serie como UART. La UART (universally asynchronous receiver/transmitter) es una unidad que incorporan ciertos procesadores, encargada de realiza la conversión de los datos a una secuencia de bits y transmitirlos o recibirlos a una velocidad determinada.

Los puertos serie están físicamente unidos a distintos pines, en Arduino UNO y Mini Pro los pines empleados son 0 (RX) y 1 (TX).


\begin{lstlisting}[language=C, caption={Ejemplo lectura y escritura puerto serie},label={lst:write_read_serial_port_sample}]

void setup(){
Serial.begin(9600);
}

void loop(){

}

void witeCharapter(char c){
Serial.println(c);
}

void readCharapter(){
if (Serial.available()>0){
input=Serial.read();
Serial.println(input);
}


\end{lstlisting}


Dado que tengo una cantidad de funciones a ejecutar mediante comandos, necesito formalizar un protocolo basado en mensajes preestablecidos. \\
Por seguir una nomenclatura, todos los comandos tienen el siguiente formato: \\


$ COMANDO?ARGUMENTO1?ARGUMENTO2  $

Y a continuación una lista de los comandos utilizados.


\begin{figure}[h]
	\centering
	\includegraphics[width=1.1\linewidth]{../images/comando_ardufocuser}
	\caption{}
	\label{fig:comando_ardufocuser}
\end{figure}

Por regla general, todo comando tiene un mensaje de respuesta, bien indicando el nuevo estado del sistema que se ha cambiado (comandos escritura), bien información sobre el estado sobre el que solicitamos información (comandos lectura) o un ECHO del comando en algún otro caso.

\newpage


Con la experiencia y el trabajo, voy familiarizándome con la plataforma, conozco nuevas bibliotecas que me ayudan, nuevos periféricos,  así como mediante sesiones de refactorización, se consigue hacer código mantenible y modular. 

Es importante mencionar el uso de la herramienta \textbf{PlatformIO}, que se define a sí misma como un ecosistema abierto de desarrollo orientado a hardware. 


Tres son las características más sustanciales. 


- PlatformIO IDE: Un IDE orientado a la programación hardware, con funciones para facilitar la depuración, así como avisos para mejorar la calidad, totalmente configurado para compilar y cargar el programa directamente en una placa física, o en una placa virtual.

\begin{figure}
	\centering
	\includegraphics[width=0.7\linewidth]{../images/ide_arduino}
	\caption{}
	\label{fig:ide_arduino}
\end{figure}

- Terminal para realizar compilación en el cloud  para diferentes placas, así como permite compilación desde plataformas de integración continua como Travis CI.

- Gestor de bibliotecas y dependencias, tiene un repositorio con las bibliotecas más usadas, pudiendo buscar y actualizar rápidamente. 



El código fuente del firmware queda distribuido en los siguientes ficheros.

\begin{itemize}
	
\item $Ardufocuser-config.h$: Encapsula gran parte de la configuración, así como el mapeo de pines y demás definiciones. 
\item $Ardufocuser-init.h$: Inicializa el sistema, creando las instancias de los objetos utilizados, variables globales, contadores y demás estados.
\item $Ardufocuser-cmd.h$: Se definen los comandos remotos, junto con sus funciones callback asociadas.
\item $Ardufocuser-utils.h$: Incorpora algunas funciones de propósito general útiles en algunos de los módulos. 
\item $Ardufocuser.ino$: Es el script principal, incluye los ficheros anteriores, contiene las funciones setup(),  loop() y callback de las interrupciones hardware y software.

\item $library.json$: Hace referencia a las librerías de terceros, catalogadas, y con referencias al sitio web y al desarrollador o equipo de desarrollo. Por seguridad también las incorporo al repositorio en el directorio libs.  


\end{itemize}
\newpage
\begin{lstlisting}[language=C, caption={Núcleo implementación firmware  ardufocuser},label={lst:nucleo_firmware_ardufocuser }]

	void setup()
	{
		//Inicia comunicación serie.
		Serial.begin(9600);
		
		// Inicia pantalla LCD.
		lcd.begin();
		lcd.backlight();
		
		// Saludo inicial.
		welcome("   ARDUFOCUSER  ");
		
		// Velocidad y Aceletación inicial del motor.
		motor.setMaxSpeed(200);
		motor.setAcceleration(1000);
		
		//Iniciamos control Nunckuck
		chuck.begin();
		chuck.update();
		
		// Actualizamos con datos guardados en sesion anterior
		load_session();
		
		// Iniciamos interrupciones Software.
		// Gestiona movimiento del motor.
		Timer1.initialize(50);
		Timer1.attachInterrupt(timerFunction);
		
		// Inicia interrupciones hardware a la escucha.
		attachInterrupt ( 0, finA,RISING);
		attachInterrupt ( 1, finB,RISING);
		
		// Registramos y iniciamos comandos serie.
		registerCommand();
	}
	
	void loop()
	{
		// Leemos nuevo comando serie.
		serial_cmd.readSerial();
		
		// Leemos control manual y sensores auxiliares.
		read_manual_controller();
		
		// Actualizamos LCD.
		update_lcd_display();
		
		// Leemos controles nunchuck.
		nunckuck_controller();
		
		// Almacenamos estados de forma persistente 
		// para otra sesión.
		save_current_session();
	}



\end{lstlisting}


\section{Pruebas}

Además para asegurar que el firmware implementado, permite ser compilado en las distintas placas hacemos uso del framework   \textbf{PlatformIO} \cite{patform}.


- Buscador centralizado de bibliotecas, en los proyectos Arduino es uno de los problemas, dado que cada una de ellas puede provenir de una fuente totalmente diferente, con esta herramienta contamos con un solo lugar.

\newpage
\begin{lstlisting}[language=python, caption={Script travis para realizar integración continua},label={lst:write_read_serial_port_sample}]

language: python
python:
- "2.7"

install:
- python -c "$(curl -fsSL https://raw.githubusercontent.com/platformio/platformio/master/scripts/get-platformio.py)"
- wget https://github.com/josemlp91/ardufocuser_firmware/raw/master/Ardufocuser/libs/TimerOne.zip
- unzip TimerOne.zip
- wget https://github.com/josemlp91/ardufocuser_firmware/raw/master/Ardufocuser/libs/i2clcd.zip
- unzip i2clcd.zip
- wget https://github.com/josemlp91/ardufocuser_firmware/raw/master/Ardufocuser/libs/AccelStepper-1.49.zip
- unzip AccelStepper-1.49.zip
- wget https://github.com/josemlp91/ardufocuser_firmware/raw/master/Ardufocuser/libs/nunchuck.zip
- unzip nunchuck.zip

script:
- platformio ci Ardufocuser --lib="TimerOne" --lib="i2clcd" --lib="nunchuck" --lib="AccelStepper" --board=uno

\end{lstlisting}


<<<<<<< HEAD
\chapter{Módulo Driver INDI}


En esta parte del proyecto nos centraremos en exponer como se ha creado el driver INDI específico para el dispositivo desarrollado bajo la tecnología Arduino en los capítulos anteriores de este mismo proyecto.

Este driver dotará a este dispositivo de la funcionalidad necesaria para poder funcionar en remoto, bajo el protocolo de control INDI, destinado a dispositivos astronómicos. 

Para ello se trabajará con la versión implementada en Java de este protocolo, \textbf{INDI for Java} \cite{INDIFJ}, por lo cual el lenguaje de programación que usaremos es Java.


\section{Breve introducción a INDI}

INDI consiste a su nivel más básico en un protocolo que permite el control, automatización, obtención de datos e intercambio de los mismos entre distintos dispositivos hardware y programas cliente. 

La idea subyacente en el protocolo INDI es desacoplar aspectos específicos del hardware, haciendo que cambios en el hardware no impliquen necesariamente cambios en el software.

\begin{figure}[h]
	\begin{center}
		\includegraphics[width=0.5\textwidth]{../images/indi.png}
		\caption[INDI Logo]{Logo de INDI  Fuente: \cite{indi}}
		\label{fig:indi}
	\end{center}
\end{figure}

Para conseguir un desacople efectivo entre los clientes y el hardware se define un protocolo basado en \textbf{XML} que permite abstraer los dispositivos hardware como conjuntos de \textbf{propiedades} que pueden ser leídas, y modificadas por los clientes (siempre estableciendo las restricciones oportunas).


\subsection{Drivers, Servidores y Clientes INDI}

Pese a que nivel más básico INDI es ``simplemente'' una especificación de un protocolo basado en XML, a un nivel superior se distinguen tres entidades diferentes que interaccionan entre sí para tener un sistema de control plenamente funcional (figuras~\ref{fig:DriverServerIndi} y \ref{fig:indi-tophology}):


\begin{itemize}
	\item \textbf{Drivers:} Son programas que se ejecutarán (normalmente) en la máquina donde se conectan los dispositivos hardware. Son los encargados de la comunicación directa con los dispositivos y su abstracción a propiedades INDI.
		
	\item \textbf{Servidor:} Es un programa cuya función principal es ejecutar los drivers y permitir la conexión a los mismos por parte de los clientes (funciona de un modo similar a un proxy). Normalmente reside en la máquina donde están conectados los dispositivos, aunque en principio se pueden crear estructuras de red tipo árbol de servidores. El intercambio de información entre el servidor y los drivers se realiza utilizando el protocolo INDI.
	
	\item \textbf{Cliente:} Es un programa que permite conectar con uno o más servidores y su función principal e hacer de \textbf{interfaz} con el usuario. Para ello conecta (usualmente a través de la red) con el servidor e intercambia información sobre los dispositivos utilizando el protocolo INDI. Es interesante recalcar que los clientes pueden ser de cualquier estilo: desde programas con interfaz de usuario avanzadas, hasta programas simples en línea de comandos scripts completamente automáticos que controlen o monitoricen los dispositivos.
\end{itemize}


\begin{figure}[h]
	\centering
	\includegraphics[width=1\linewidth]{../images/INDIPipeline}
	\caption[Pipeline INDI]{\textbf{Pipeline INDI}, flujo de información y conversión en diferentes entidades. (String Comando Serie - Propiedad Indi - Documento XML )}
	\label{fig:DriverServerIndi}
\end{figure}

\begin{figure}
	\centering
	\includegraphics[width=1\linewidth]{../images/indi-tophology}
	\caption[Arquitectura Servidor - Driver INDI]{Arquitectura Servidor - Driver INDI}
	\label{fig:indi-tophology}
\end{figure}


La biblioteca \textbf{INDI} original está escrita en lenguaje \textbf{C}, pero existe una implementación completa realizada en \textbf{Java} y que se encuentra en constante mejora. En la página oficial de \href{http://indilib.org/develop/indiforjava.html}{INDI} podemos encontrar toda la información sobre nuevas versiones y la documentación para poder utilizarla. La principal ventaja de poder usar \textbf{Java} es que podemos implementar drivers y clientes con la potencia de un lenguaje orientado a objetos y combinarlo con otras tecnologías como los dispositivos móviles basados en la plataforma \textbf{Android}  \href{https://play.google.com/store/apps/details?id=com.jtbenavente.jaime.indiandroidui&hl=es}{(\textbf{enlace descarga de Remote Observatory (App Android)})} 





\subsection{Propiedades INDI}

El protocolo INDI define un conjunto de 5 propiedades que envían los diferentes drivers para formar la interfaz del cliente. Las propiedades se relacionan con el tipo de dato que maneja el dispositivo para controlar una característica concreta:

\begin{itemize}
	\item \textbf{Textos:} Son cadenas de caracteres ordenados arbitrariamente.
	
	\item \textbf{Números:} Son cantidades numéricas. Además de la cantidad numérica se envían otros parámetros que sirven para el formato de su visualización y configuración (mínimo, máximo, paso, etc.).
	
	\item \textbf{Switches:} Son propiedades binarias con estado encendido o apagado. Sus agrupaciones pueden ser de tres tipos diferentes:
	
	\begin{itemize}
		\item \textbf{Una de muchas:} para todas las opciones se tiene que seleccionar
		obligatoriamente una.
		
		\item \textbf{Como máximo una:} de todas las opciones se puede seleccionar
		como máximo una.
		
		\item \textbf{Cualquiera de muchas:} se podrán seleccionar todas las que se
		deseen.
		
	\end{itemize}	
	
	\item \textbf{Lights:} Son propiedades de solo lectura, que pueden estar en cada uno de los cuatro
	estados que se definen.
	
	\begin{itemize}
		\item \textbf{Inactivo:} Luz de color gris.
		\item \textbf{Alerta:} Luz de color rojo.
		\item \textbf{Ocupado:} Luz de color amarillo.
		\item \textbf{Ok:} Luz de color verde.
	\end{itemize}
		
	\item \textbf{BLOBs:} Datos binarios cualesquiera.
\end{itemize}


\section{Diseño del Driver INDI}

A continuación se desglosa información sobre el proceso de diseño y planificación de este módulo.

\subsection{Requisitos funcionales}

\begin{itemize}
	\item \textbf{RF-1.}: Crear conexión con dispositivo enfocador Ardufocuser.
	\item \textbf{RF-2.}: Controlar posición del enfocador Ardufocuser, permitir moverlo a la posición deseada.
	\item \textbf{RF-3.}: Modificar velocidad del enfocador.
	\item \textbf{RF-4.}: Modificar pasos por pulso que se mueve.
	\item \textbf{RF-5.}: Modificar posición relativa del enfocador.
	\item \textbf{RF-5.}: Mostrar temperatura del enfocador.
	\item \textbf{RF-5.}: Establece límite software del enfocador.
	\item \textbf{RF-6.}: Mostrar aviso si se alcanza un límite software.
	\item \textbf{RF-7.}: Mostrar aviso si se alcanza un límite hardware.
	\item \textbf{RF-8.}: Encender y apagar la iluminación de la pantalla lcd.		
\end{itemize}

\subsection{Requisitos no funcionales}
\begin{itemize}
	\item \textbf{RNF-1.}: El intercambio de mensajes por el puerto serie tiene que ser eficiente y tolerante a perdida de mensajes.
	\item \textbf{RNF-2.}: La información ha de presentarse al usuario de una forma clara e intuitiva.
	\item \textbf{RNF-3.}: Debe ser compatible con los clientes INDI que existen actualmente.
\end{itemize}

\subsection{Arquitectura INDI for Java}

Debemos conocer la estructura que tiene un Driver INDI. Para ello se recurrimos a la referencia oficial   \href{http://www.indilib.org/develop/indiforjava/i4j-indi-driver.html}{indiforjava}, donde podemos encontrar una buena guía para implementar un Driver, junto un ejemplo simple. 

\textbf{INDI for Java}, ya cuenta con una clase abstracta que define muchos de los atributos y métodos mínimos con los que debe contar un enfocador ``\texttt{Focuser}'' genérico. Debemos heredar de dicha clase para definir nuestro \texttt{ArduFocuserDriver} particular, añadiendo nuevos atributos e implementando los métodos abstractos existentes. 


\begin{figure}
\centering
\includegraphics[width=1\linewidth]{../images/indi_classes}
\caption{Diagrama de clases Driver INDI}
\label{fig:indi_classes}
\end{figure}



\paragraph{Propiedades a implementar}

Para ello lo primero que tenemos que hacer es definir exactamente las propiedades que tiene nuestro dispositivo, el tipo y si son lectura, escritura o ambas. 

Dichas propiedades (tabla~\ref{fig:propiedades_indi}) serán comunicadas al cliente INDI para que este puede hacer una representación en la interfaz gráfica.

\begin{table}[h]
	\centering

	\begin{tabular}{|l|l|l|l|l|}
		\hline
		Descripción                        & RW/OW/OR  & Tipo Propiedad &          \begin{tabular}[c]{@{}l@{}}   
			                                                                          Mininimo \\ 
			                                                                          Máximo   \\ 
			                                                                          Step     \\
			                                                                     \end{tabular} \\ \hline\hline
			                                                                     
			                                                                     
		Posición Actual                    & RW        & Number&            \begin{tabular}[c]{@{}l@{}}
																						MinimumAbsPos       \\
																						MaximumAbsPos    \\
																						1 \\
																		        \end{tabular} \\ \hline
																				    
		Velocidad                          & RW        & Number&             \begin{tabular}[c]{@{}l@{}}
			                                                                         1              \\
			                                                                         MaximumSpeed	\\
			                                                                         1			    \\
			                                                                    \end{tabular} \\ \hline
			                                                                    
		Movimiento enfocador           & OR       & SwitchOneOrNone&    --           \\ \hline
		Pasos por pulso                    & RW       & Number&            \begin{tabular}[c]{@{}l@{}}
																					1  \\
																					99 \\
																					1  \\ 
																			   \end{tabular} \\ \hline

		Temperatura                        & OR       & Number&             --\\ \hline
		Límite Hardware                     & OR       & Switch&             --\\ \hline
		Límite Software                     & OR       & Switch&             --\\ \hline	
		Posición Relativa                  & RW       & Number&            \begin{tabular}[c]{@{}l@{}}
																					MinimumAbsPos  \\
																					MaximumAbsPos \\
																					1  \\ 
																				\end{tabular} \\ \hline	
		Iluminación LCD                       & RW       & Switch&             --\\  \hline
	                  
	\end{tabular}
		\caption{Propiedades del Driver INDI del Ardufocuser}
	\label{fig:propiedades_indi}
\end{table}


\section{Implementación}


\paragraph{Gestionar conexión serie}

La comunicación directa con \textbf{Ardufocuser} se basa en el protocolo serie. Por tanto el primer paso es hacer que nuestra rutina Java se conecte al puerto serie de Arduino. Para ello se hace uso de la biblioteca \textbf{RXTX library}, junto con un conector específico para Arduino, \textbf{JavaDuino} \cite{JavaDuino}. En el fragmento de código~\ref{lst:ejemplo_libreria_serial_commandd} podemos ver un ejemplo de su uso.


\begin{lstlisting}[language=javascript, caption={Ejemplo biblioteca \texttt{SerialCommand}},label={lst:ejemplo_libreria_serial_commandd}]
 import gnu.io.SerialPortEvent;
 import gnu.io.SerialPortEventListener;
 public class Main {
 
	 public static void main(String[] args) { 
		 // Abrir la conexión Arduino.
		 ArduinoConnection ac = new ArduinoConnection();
		 boolean connected = ac.connectToBoard();
		 
		 if (connected) {
			 System.out.println("Conectado!");
		 } else {
			 System.out.println("No se puede conectar con Arduino :-(");
			 return;
		 }
		 
		 // Añadimos escuchador, que responde a un evento série.
		 ac.addListener(new SampleListener(ac));
		 // Enviar mensajes por puerto serie.
		 ac.sendString("Arduino!");
		 // Cerrar la conexión.
		 ac.close();
	 }
	 
	 private static class SampleListener implements SerialPortEventListener {	 
		 private ArduinoConnection ac;	 
			 public SampleListener(ArduinoConnection ac) {
			 this.ac = ac;
		 }
		 		 
		 @Override
		 public void serialEvent(SerialPortEvent serialPortEvent) {
			 // Callback que se ejecuta con un evento serie.
			 if (serialPortEvent.getEventType() == SerialPortEvent.DATA_AVAILABLE) {
				 String inLine = ac.readLine();
				 System.out.println("GOT: " + inLine);
			 }
		 }
	 }
 }
\end{lstlisting}


\texttt{ArduinoConnection} cuenta con los siguiente métodos:

\begin{itemize}
	\item \texttt{ArduinoConnection()} : Constructor principal.
	\item \texttt{connectToBoard()} : Se conecta a la placa Arduino.
	\item \texttt{addListener(SerialPortEventListener)} : Añade un escuchador, permite ejecutar una función callback dado un evento serie, sin bloquear el flujo de la rutina main.
	\item \texttt{sendString(String)} : Envía una cadena e texto a Arduino.
\end{itemize}

Otro aspecto importante en la implementación del driver INDI es la creación e inicialización de las propiedades que hemos comentado.

Para ello se debe crear una instancia \texttt{INDIProperty}, que es el objeto que representa a la propiedad en concreto, indicando el grupo de propiedades al que pertenece, el estado inicial así como si es de lectura, escritura o lectura escritura.

Una propiedad puede tener varios elementos: de manera informal podemos decir que los elementos son las diferente variables que componen una propiedad. Cuando instanciamos un elemento debemos pasar como parámetro la propiedad a la que pertenece. En el fragmento de códgo~\ref{lst:ejemplo_iniciar_propiedad_indi} podemos ver un ejemplo de inicialización de una propiedad. Entre otras se muestra la propiedad definida en el ejemplo anterior.





\begin{lstlisting}[language=javascript, caption={Ejemplo iniciar una propieda INDI Numérica},label={lst:ejemplo_iniciar_propiedad_indi}]
// Propiedad numérica para informar de la temperatura.
private INDINumberProperty temperatureP;

// Elemento numérico para informar de la temperatura.
private INDINumberElement temperatureE;

// Inicializador propiedad para manejar temperatura.
private void inizializeTempretureProperty() {
	if (temperatureP == null) {
		// Se crea la propiedad, con el estado inicial, indicando el conjunto al que pertenece y si es escritura/lectura.
		temperatureP = new INDINumberProperty(this, "temperature", "Temperature", "Control", PropertyStates.IDLE, PropertyPermissions.RO);
		temperatureE = temperatureP.getElement("temperature_value");
		if (temperatureE == null) {
			// La propiedad 
			temperatureE = new INDINumberElement(temperatureP, "temperature", "Temperature", "1", "1", "99", "1", "%f");
		}
	}
}
\end{lstlisting}










\section{Instalación servidor}

Para llevar a cabo todas las pruebas necesarias durante el desarrollo el servidor INDI se ha instalado en un mini-pc Raspberry-Pi, haciendo uso de la distribución Raspbian. El propósito de esta máquina es doble, su uso principal como \textbf{servidor astronómico INDI} (figura~\ref{fig:raspberry}) y además al contar con todo el entorno de Arduino, facilita la tarea de \textbf{actualización del firmware}. En la figura~\ref{fig:diagramaGeneral} encontramos un diagrama completo del sistema Ardufocuser contando con el servidor INDI en la Raspberry Pi.

\begin{figure}[h]
\centering
\includegraphics[width=1\linewidth]{../images/rasp_server}
\caption[Servidor INDI funcionando en Raspberry Pi]{Servidor INDI funcionando en Raspberry Pi}
\label{fig:raspberry}
\end{figure}

\begin{figure}
\centering
\includegraphics[width=1\linewidth]{../images/diagramaGeneral}
\caption[Diagrama completo del sistema Ardufocuser]{Diagrama completo del sistema Ardufocuser}
\label{fig:diagramaGeneral}
\end{figure}


Para ello se han instalado las siguientes herramientas:
\begin{itemize}
	\item \textbf{Java}, entorno de ejecución Java.
	\item \textbf{Netbeans} Entorno de desarrollo Java.
	\item \textbf{Arduino IDE}, permite cargar el firmware en la placa.
	\item \textbf{KStars}, cliente INDI.
	\item \textbf{Servidor INDI + Driver INDI}.
\end{itemize}

La biblioteca necesaria para que Raspberry y Arduino se puedan comunicar mediante puerto serie es \textbf{RXTX} en su versión para Java. 

\begin{center}
\texttt{> sudo apt-get install librxtx-java}
\end{center}

En los fragmentos de código~\ref{lst:script_inicio_indi} y \ref{lst:script_inicio_2} se muestran los scripts que se han creado / modificado en la distribución Raspbian para ejecutar automáticamente el servidor INDI.

\begin{lstlisting}[language=bash, caption={Script de inicio del servidor INDI},label={lst:script_inicio_indi}][h!]
#! /bin/sh
# /etc/init.d/indiserver-init

### BEGIN INIT INFO
# Provides: 		indiserver-init
# Required-Start: 	$all
# Required-Stop:	$remote_fs $syslog
# Default-Start:	2 3 4 5
# Default-Stop:		0 1 6
# Short-Description:	Run Indi Server.
# Description:		Run Indi Server with Driver for Ardufocuser.
### END INIT INFO

case "$1" in
start)
echo "IndiServer Runing"
/home/pi/Desktop/ardufocuser_indidriver/RunDriver2IndiServerRPBeep.sh
;;
stop)
echo "IndiServer Stoping"
;;
*)
echo "Modo de uso: /etc/init.d/indiserver-init {start|stop}"
exit 1
;;
esac

exit 0
\end{lstlisting}


\begin{lstlisting}[language=bash, caption={Script de inicio del servidor INDI (2)},label={lst:script_inicio_2}]
#!/bin/bash

# Emite un beep por un zumbador.
python /home/pi/Desktop/ardufocuser_indidriver/beepOK.py

# Ejecuta Servidor INDI con el correspondiente Driver como módulo.
java  -Djava.library.path=/usr/lib/jni -cp /usr/share/java/RXTXcomm.jar  -jar /home/pi/Desktop/ardufocuser_indidriver/I4JServer/dist/I4JServer.jar -add=/home/pi/Desktop/ardufocuser_indidriver/I4JArdufocuserDriver/dist/I4JArdufocuserDriver.jar
\end{lstlisting}





\subsection{Crear imagen Rasbindi}

Para facilitar la instalación del sistema en la Raspberry se ha creado una imagen de todo el entorno. La imagen la podemos descargar directamente del siguiente enlace:


\href{https://drive.google.com/open?id=0Bz7iXJ4BvZ9SbnJPZWkweVhUVjQ}{Enlace para desscargar \textbf{Raspbindi}, imagen personalizada con un servidor INDI configurado.}






\section{Pruebas}

Para este módulo se han realizado pruebas de caja negra, es decir, sin conocer detalles internos del dispositivo tratamos de reproducir posibles casos de uso, y se comprueba si las salidas corresponden con lo esperado. 


Para las pruebas se han usado dos clientes INDI diferentes, anteriormente comentados, KStars (Linux)~\ref{fig:kstar_driver1} y  Observatorio Remoto (Android)~\ref{fig:remote_obs}.


\begin{figure}
	\centering
	\includegraphics[width=0.8\linewidth]{../images/kstar_driver_full}
	\caption[Pruebas desde KStars]{Se realizan pruebas desde KStars (Cliente de Escritorio)}
	\label{fig:kstar_driver1}
\end{figure}

 \begin{figure}[h]
	\centering
	\includegraphics[width=1\linewidth]{../images/obs_remoto_driver_full}
	\caption[Pruebas desde Observatorio Remoto]{Se realizan pruebas desde Observatorio Remoto (Cliente Android)}
	\label{fig:remote_obs}
\end{figure}

En las tablas~\ref{tab:cas1}, \ref{tab:cas2}, \ref{tab:cas3} y \ref{tab:cas4} se pueden ver ejemplos de dichos casos de prueba. Además, en las figuras~\ref{fig:test_1} y \ref{fig:test_2} podemos ver los resultados visuales de la prueba de conexión (con el cliente kstars) al driver y encendido del mismo.

\begin{table}[h]
	\centering

	\begin{tabular}{|l|l|}
		\hline
		ID caso de prueba             &  1 \\ \hline
		Nombre prueba                 &  Conectar a servidor INDI con driver Ardufocuser \\ \hline
		Autor de la prueba            &  José Miguel López \\ \hline
		Responsable diseño            &  José Miguel López \\ \hline
		Pasos y condiciones ejecución &  \begin{tabular}[c]{@{}l@{}}
			- Ejecutar cliente INDI. \\
			- Ir a Device Manager en KStars y pulsar en Add. \\
			- Introducir IP y puerto del servidor INDI. \\
			- Pulsar botón Connect. \\
		\end{tabular} \\ \hline
		Resultado deseado             & \begin{tabular}[c]{@{}l@{}}
			Debe aparecer una nueva ventana \\
			donde se muestran las propiedades \\
			y dispositivos INDI, organizados en pestañas.\\
			
		\end{tabular} \\ \hline
		
		Resultado obtenido            &  \begin{tabular}[c]{@{}l@{}}
			Aparece una nueva ventana \\
			donde se muestran las propiedades\\ de los dispositivos INDI.\\
		\end{tabular} \\ \hline
		Estado caso de prueba         &  Éxito\\ \hline
		Errores asociados             &  Ninguno\\ \hline
		Comentario                    &  \\ \hline
	\end{tabular}
		\caption{Caso de prueba, conectar con servidor INDI}
	\label{tab:cas1}
\end{table} 


\begin{figure}
	\centering
	\includegraphics[width=1\linewidth]{../images/test_1}
	\caption{Conectando el servidor INDI con el driver Ardufocuser}
	\label{fig:test_1}
\end{figure}

 

\begin{table}[h]
	\centering

	\begin{tabular}{|l|l|}
		\hline
		ID caso de prueba             &  2 \\ \hline
		Nombre prueba                 &  Conectar Ardufocuser \\ \hline
		Autor de la prueba            &  José Miguel López \\ \hline
		Responsable diseño            &  José Miguel López \\ \hline
		Pasos y condiciones ejecución &  \begin{tabular}[c]{@{}l@{}}
			- Partimos del resultado obtenido \\
			en el caso de uso anterior.\\
			- Ir a la pestaña \texttt{Ardufocuser}\\
			- Ir a la pestaña \texttt{Main Control}\\
			- Pulsar \texttt{Connect}
		\end{tabular} \\ \hline
		Resultado deseado             & \begin{tabular}[c]{@{}l@{}}
			Debe cambiar la luz de la propiedad a verde. \\
			Se activan dos pestañas adicionales\\\texttt{Control} y \texttt{Configuration}. \\
		\end{tabular} \\ \hline
		
		Resultado obtenido            &  \begin{tabular}[c]{@{}l@{}}
			Cambiar la luz de la propiedad a verde. \\
			Se activan dos pestañas adicionales\\\texttt{Control} y \texttt{Configuration}. \\
		\end{tabular} \\ \hline
		Estado caso de prueba         &  Éxito\\ \hline
		Errores asociados             &  Ninguno\\ \hline
		Comentario                    &  \\ \hline
	\end{tabular}
		\caption{Caso de prueba, conectar Ardufocuser}
	\label{tab:cas2}
\end{table} 


\begin{figure}
	\centering
	\includegraphics[width=1\linewidth]{../images/01_connection}
	\caption{Conectando el Ardufocuser a través de Kstars}
	\label{fig:test_2}
\end{figure}


\begin{table}[h]
	\centering

	\begin{tabular}{|l|l|}
		\hline
		ID caso de prueba             &  3 \\ \hline
		Nombre prueba                 &  Modificar velocidad del motor \\ \hline
		Autor de la prueba            &  José Miguel López \\ \hline
		Responsable diseño            &  José Miguel López \\ \hline
		Pasos y condiciones ejecución &  \begin{tabular}[c]{@{}l@{}}
			- Partimos del resultado obtenido \\
			en el caso de uso anterior.\\
			- Ir a la pestaña \texttt{Ardufocuser}.\\
			- Ir a la pestaña \texttt{Configuration}.\\
			- Modificar valor \texttt{Focus Speed} y pulsar en \texttt{Set}.
		\end{tabular} \\ \hline
		Resultado deseado             & \begin{tabular}[c]{@{}l@{}}
		    Debe cambiar la luz de la propiedad a verde. \\
		    Debe cambiar el valor SP en la pantalla LCD \\
		\end{tabular} \\ \hline
		
		Resultado obtenido            &  \begin{tabular}[c]{@{}l@{}}
			Cambiar la luz de la propiedad a verde. \\
			Cambiar el valor SP en la pantalla LCD \\
		\end{tabular} \\ \hline
		Estado caso de prueba         &  Éxito\\ \hline
		Errores asociados             &  Ninguno\\ \hline
		Comentario                    &  \\ \hline
	\end{tabular}
		\caption{Caso de prueba, modificar velocidad}
	\label{tab:cas3}
\end{table} 


\begin{table}[h]
	\centering

	\begin{tabular}{|l|l|}
		\hline
		ID caso de prueba             &  4 \\ \hline
		Nombre prueba                 &  Establecer posición del enfocador. \\ \hline
		Autor de la prueba            &  José Miguel López \\ \hline
		Responsable diseño            &  José Miguel López \\ \hline
		Pasos y condiciones ejecución &  \begin{tabular}[c]{@{}l@{}}
			- Partimos del resultado obtenido \\
			en el caso de uso anterior.\\
			- Ir a la pestaña \texttt{Ardufocuser}.\\
			- Ir a la pestaña \texttt{Control}.\\
			- Modificar valor \texttt{Absolute} y pulsar en \texttt{Set}.
		\end{tabular} \\ \hline
		Resultado deseado             & \begin{tabular}[c]{@{}l@{}}
			Debe cambiar la luz a rojo. \\
			El motor del enfocador comienza a girar \\
			Se debe ver el avance del motor en tiempo real \\
			Al pararse la luz vuelve a color verde. \\
		\end{tabular} \\ \hline
		
		Resultado obtenido            &  \begin{tabular}[c]{@{}l@{}}
			Cambiar la luz a rojo. \\
			El motor del enfocador comienza a girar \\
			Se ve el avance del motor en tiempo real \\
			Al pararse la luz vuelve a color verde. \\
		\end{tabular} \\ \hline
		Estado caso de prueba         &  Éxito\\ \hline
		Errores asociados             &  Ninguno\\ \hline
		Comentario                    &  \\ \hline
	\end{tabular}
		\caption{Caso de prueba, establecer posición del enfocador}
	\label{tab:cas4}
\end{table} 







\input{chapters/08_Modulo_Software}
\chapter{Documentación y Difusión}

Para favorecer el interés del proyecto entre los aficionados a la astronomía (y al hardware libre en general) se han emprendido diversas actividades divulgativas, como son:


\begin{itemize}
	\item Creación de una página web del proyecto.
	\item Creación de canales en redes sociales.
	\item Asistencia y presentación del proyecto en congresos astroómicos.
	\item Mantenimiento del repositorio \textbf{GitHub} con toda la información relevante del proyecto (documentación, código, diagramas...)
\end{itemize}

En las siguientes secciones se comenta más en detalle cada una de dichas actividades.


\section{Página web del proyecto}

Se diseña una página web a modo de \textbf{``landing page''} que resuma toda la información del proyecto de una forma agradable al visitante. Las caracerísticas más reseñables de dicha página web son:

\begin{itemize}
  \item Mostrar toda la información del proyecto bien catalogada y organizada en las diferentes secciones. 
 Se incluyen fotos y resumenes de algunos pasos del proyecto, lista de materiales y enlaces de descarga.

  \item Seguir diseño visual atractivo y favorecer una buena experiencia de usuario, cuidando el esquema de colores, la disposición de los elementos y su tamaño, permitiendo la navegación por el sitio de una forma cómoda e intuitiva.

  \item Diseño \textit{responsive}, asegurando que se ven todos los elementos de forma correcta en ordenadores, móviles y tabletas. 

  \item Seguir los últimos estándares HTML5, validando una correcta semántica, utilizar \textbf{metadescripciones} \cite{meta}. Con ello favorecemos  una correcta indexación por los motores de búsqueda, haciendo llegar la página al público objetivo de una forma más rápida y selectiva (uno de los principios básico de SEO \cite{seo}).
\end{itemize}

Dicha página web puede verse en \url{http://ardufocuser.info}.

\section{Canales en redes sociales}

Al comienzo del desarrollo se creó una cuenta en twitter \textbf{@ardufocusindi} con la finalidad de crear una comunidad de gente interesada en el proyecto y en otros temas relacionados.

La idea de este canal era mantener informada a la comunidad de los últimos avances del proyecto y nuevas actualizaciones así como compartir también material interesante para estar al día de los últimos avances.

Sin embargo, hay que reconocer que este intento de difusión no ha funcionado bien, probablemente por la falta de actualización así como la dificultad de publicitar algo tan específico que prácticamente solo interesa a un público objetivo muy concreto (astrónomos) sin ser el propio desarrollador un miembro de esta comunidad.


\section{Congresos Astronómicos}

Durante el proyecto he tenido la oportunidad de asistir a eventos y congresos astronómicos donde se ha promocionado el proyecto. 


\begin{itemize}
	\item \textbf{Astro-Encuentro La Sagra 2015}: Celebrado del 16 al 18 de Octubre en la localidad de la Puebla de Don Fabrique, dos días diversas actividades en el marco de la astronomía, donde tuve oportunidad de visitar el ``Observatorio Astronómico de la Sagra'' \cite{lasagra} operado, entre otros, por el Instituto de Astrofísica de Andalucía (IAA).
	
	Entre todas las actividades, por su relación con el presente proyecto, cabe destacar la ponencia de D. Nicolás Morales (investigador del Instituto de Astrofísica de Andalucía) donde realiza una demostración de una observación con telescopios manejados en remoto en tiempo real demostrando como realizan estudios de impactos en la Luna.
	
	\item \textbf{Astro-Alcala Alcalá la Real 2016}: Celebrado del 8 al 10 de Abril en la localidad de Alcalá la Real. En dicho encuentro se presentó un póster mostrando un resumen del proyecto Ardufocuser (figura~\ref{fig:posterAstroAlcala}).
	
\begin{figure}[h]
	\centering
	\includegraphics[width=1\linewidth]{../images/poster_Ardufocuser_AstroAlcala}
	\caption[Poster presentando el proyecto en AstroAlcalá 2016]{Poster presentando el proyecto en AstroAlcalá 2016.}
	\label{fig:posterAstroAlcala}
\end{figure}
	
	\item Presentación del proyecto \textbf{Ardufocuser} ante la SAG (Sociedad Astronómica Granadina) \cite{sag} en Febrero de 2016: En dicha presentación se mencionan a grandes rasgos las ventajas que ofrece en comparación a los productos comerciales y se comparten detalles técnicos del mismo. Varios de los asistentes mostraron mucho interés en el proyecto.
\end{itemize}
 
 
\section{Mantenimiento del repositorio GitHub}

Para el desarrollo del proyecto se ha contado con varios repositorios en Github, por comodidad uno para cada parte del proyecto. 


\begin{itemize}
	\item \textbf{Ardufocuser-firmware:} Contiene todo el código fuente del firmware que ejecuta la placa de Arduino, junto con las bibliotecas correspondientes y los scripts de pruebas. 
	\item \textbf{Ardufocuser-indidriver:} Contiene el módulo driver INDI específico para Ardufocuser, implementado en Java a partir de las especificaciones \textbf{INDIforJava}.
	\item \textbf{Ardufocuser-image-processing:} Contiene el código fuente del software de análisis de imágenes, detección de objetos y calculo de FWHM. Además cuenta con la biblioteca de manejo de imágenes FITS implementada para facilitar el trabajo.
	\item \textbf{Ardufocuser-documentations:} Contene los fuentes  \LaTeX de este mismo documento junto con las imágenes y recursos utilizados.
\end{itemize}

Utilizar GitHub me ha proporcionado grandes ventajas.

\begin{itemize}
	\item Contar con un sistema de control de versiones distribuido, con toda la potencia de \textbf{Git}.
	\item Publicar los cambios directamente y ser accesibles desde la web (se puede consultar desde un navegador web).
	\item Conectar servicio de integración continua Travis CI.
\end{itemize}

En todo momento se ha intentado cuidar el estilo y la calidad del código, añadiendo los pertinentes comentarios y utilizando nombres adecuados para nombrar variables, clases, métodos o funciones con el propósito que cualquier interesado pueda acceder al código, comprender la lógica y realizar cambios de forma fácil.  

Cualquier interesado en colaborar queda invitado a realizar \textbf{Merge Request}, indicando un resumen de la mejora o el bug que soluciona. En lo más breve se evaluará el cambio, se incorporará a la base de código y aparecerá como colaborador del proyecto.


 

\chapter{Conclusiones y trabajos futuros}

Lo primero que me gustaría resaltar es que creo que el resultado de este trabajo ha sido muy satisfactorio tanto por el resultado como por el proceso de desarrollo. Con el desarrollo del proyecto he tenido oportunidad de familiarizarme con multitud de tecnologías y disciplinas para solucionar problemas de diferente naturaleza. 

He conseguido implementar una solución de enfoque perfectamente válida, que cumple con gran parte de los objetivos propuestos (no solo los obligatorios, sino también algunos secundarios). No obstante es importante remarcar que \textbf{Ardufocuser} aun permanece en fase \textbf{beta}.

Otro de los grandes alicientes que ha tenido el presente TFG, ha sido poder participar en jornadas de observación, agradeciendo experiencia al grupo de personas que conforman \textbf{La Azotea} \cite{laazotea} ya que con ellos he aprendido concepto básicos de astronomía, así como hacer pruebas del sistema y comprender mejor algunos de los retos que suponen algo tan aparentemente sencillo como es enfocar un telescopio.

También he podido participar en varios congresos astronómicos \textbf{AstroAlcalá 2016} (Alcalá la Real) y \textbf{AstroEncuentro La Sagra 2015} (Puebla Don Fabrique), eventos muy enriquecedores donde he tenido posibilidad de ver el funcionamiento de observatorios profesionales, las soluciones de control que aplican y discutir con profesionales las posibles limitaciones, inconvenientes de los sistemas actuales y como se podrían hacer una implantación de sistemas de control basados en la tecnologías que hemos presentado en el proyecto, especialmente \textbf{INDIforJava} y \textbf{Hardware Libre}.

En el presente proyecto nos hemos centrado en el módulo de enfoque, por ser uno de los problemas más interesantes, pero con la base tecnológica implementada podemos afrontar el control de muchos otros elementos de un observatorio con facilidad. 

Es también reseñable y digno de destacar que es posible crear soluciones avanzadas con un presupuesto bajo y obteniendo unos resultados más que aceptables. Todo ello gracias a tecnologías libres como Arduino y Raspberry Pi, que además tienen una amplísima comunidad de personas detrás. 

El proyecto ha servido para hacerme evolucionar como ingeniero en gran medida y son muchas las lecciones aprendidas:

\begin{itemize}
	\item Dar la importancia necesaria a las reuniones con los clientes, aprendiendo a escuchar y dirigir la sesión para conseguir profundizar en la necesidad reales.
	\item Destacar la importancia de las fases de diseño y análisis, pues en base a ello se puede acortar el proceso de implementación de forma considerable, reduciendo el número de iteraciones. 
	\item Aprender a ser más realista realizando estimaciones, especialmente cuando tenemos un alto índice de incertidumbre en muchos puntos. Este punto es de vital importancia de cara a proyectos profesionales donde los proyectos estén limitado en tiempo (existan \textit{deadlines}) y limitados economicamente mediante presupuestos fijos.
	\item Tener una introducción a las metodológicas ágiles, en este caso la iterativa basada en prototipos.  
	\item Ser consciente de las capacidades adquiridas durante mi formación y como con el suficiente trabajo soy capaz de trabajar y aprender en muchas otras disciplinas relacionadas.
	\item Además, el hecho de tener que aprender y aplicar diversos conocimientos de áreas afines (como eletrónica o programación a bajo nivel) me ha hecho percatarme de como ciertos desarrollos probablemente pueden hacerse mejor si el equipo de desarrollo no se limita a una única persona sino a arios miembros con experiencia y conocimientos en áreas distintas. Por ejemplo, el presente proyecto podría haberse abordado de manera más sencilla y probablemente más eficiente si en el equipo hubieramos contado con:
	 
	 \begin{itemize}
	 	\item Un \textbf{Astrofísico}  con conocimiento en los elementos del observatorio y problemas que pueden darse en las observaciones de los fenómenos.
	 	\item Un \textbf{Ingeniero Electrónico} con conocimiento de electrónica y componentes a bajo nivel.
	 	\item Un \textbf{Ingeniero Informático} con conocimientos de ingeniería del software y procesamiento de imágenes. 
	 \end{itemize}
	 
\end{itemize}


También quiero reseñar que la aceptación del proyecto (aún en fases tempranas) ha sido grande y ya hay 
simpatizantes de diferente formación (astrónomos aficionados, ingenieros informáticos, mecánicos) interesados en participar en el proyecto, aportando también nuevas ideas y habilidades: Dado que es un proyecto abierto y con licencia libre puede participar cualquier interesado. Así mismo me consta que varias personas están esperando a la publicación del proyecto en su actual forma para directamente construir y utilizar Ardufocuser en sus observatorios particulares.



\section{Trabajos futuros}

Ardufocuser se encuentra en fase beta, lo que implica que es probable que queden detalles por afinar y ajustes que realizar. Además, existen algunas ideas que aún no se han completado o implementado o simplemente que quedan como posibles mejoras a probar. A continuación se enumeran algunas de estas ideas que podrán incorporarse o desarrollarse en el futuro:

\begin{itemize}
	\item Reducir el tamaño (y coste) de la electrónica y simplificar las conexiones, utilizando un modelo de Arduino Mini o Micro.
	
	\item Incorporar el repertorio de instrucciones \textbf{Robofocus} al protocolo serie, permitiendo compatibilidad con cualquier software de enfoque compatible con Robofocus (ASCOM).
\end{itemize}


Por otra parte a nivel software tenemos los siguientes puntos a completar en el futuro:

\begin{itemize}
	\item Implementar los algoritmos autofocus de estrellas que se han diseñado y realizar las pertinentes pruebas en un entorno real.
	\item Corrección automática de foco por compensación por cambio de temperatura.
	\item Implementar autofocus superficies planetarias, punto muy interesante y que aún no esta resuelto de manera satisfactoria por las soluciones comerciales. Para ello se deben estudiar nuevas funciones de evaluación adecuada para superficies planetarias. Esto no es una tarea trivial pues implica conocimientos de procesamiento de imágenes relativamente avanzadas. Además las imágenes planetarias gozan de muy poco contraste (en muchas ocasiones ordenes de magnitud por debajo que fotografías ``convencionales'') y sufren de distorsiones generadas por la atmósfera lo que no hace sencilla la adaptación de algoritmos en enfoque ``clásicos'' como los que pueden incorporar las cámaras de fotos. 
	\item Implementar test automáticos para las rutinas de evaluación y detección.
	\item Plantear INDI como un protocolo / plataforma que pueda ser utilizado para otros ámbitos distintos a la astronomía, como podrían ser la \textbf{domotización doméstica} o la \textbf{sensorización agrícola}.
\end{itemize}

De hecho, me consta que gracias al buen resultado obtenido en el proyecto pronto otras personas (tanto alumnos como aficionados de la astronomía) tienen pensado continuar con el diseño e implementación de otros módulos de control de instrumental astronómico como pueden ser ruedas portafiltros, cúpulas, techos corredizos, controladoras de relés, etc.








%\newpage


\chapter*{Glosario}



{
	\section*{A}
}
\noindent 
\textbf{API}. \label{API} \ \ Application Programming Interface
\newline
\textbf{AWT}. \ \ Abstract Window Toolkit
\newline
\textbf{ASCOM}. \ \label{ASCOM} AStronomy Common Object Model



{
	\section*{C}
}
\noindent 
\textbf{CSS}. \ \ Cascading Style Sheets
\newline
\textbf{CU}. \ \ Caso de uso
\newline
\textbf{CCD}. \ \ \label{CCD} Charge Coupled Device

 {
 	\section*{D}
 }
 
 \noindent 
 \textbf{DIY}.  \label{DIY} \ \ Do It Yourself - Hágalo usted mismo

{
	\section*{E}
}
\noindent 
\textbf{EEG}. \ \ Electroencefalografía
\newline
\textbf{EDF}. \ \ European Data Format

{
	\section*{F}
}
\noindent 
\textbf{FITS}. \ \ \label{FITS} Flexible Image Transport System
\newline
\textbf{FWHM}. \ \ \label{FWHM} Full Width at Half Maximum

{
	\section*{G}
}
\noindent 
\textbf{GPS}. \ \ \label{GPS} Global Positioning System
\newline
\textbf{GSM}. \ \ \label{GSM} Global System for Mobile 
\newline
\textbf{GPRS}. \ \ \label{GPRS} General Packet Radio Service 




{
	\section*{H}
}
\noindent 
\textbf{Hz}. \ \ Hertzios

{
	\section*{I}
}
\noindent 
\textbf{IDE}. \ \ \label{IDE} Integrated Development Environment
\newline
\textbf{I2C}. \ \ \label{I2C} Inter-Integrated Circuit
\newline
\textbf{INDI}. \ \ \label{INDI} Instrument Neutral Distributed Interface




{
	\section*{L}
}

\noindent 
\textbf{LCD}. \label{LCD} \ \ Liquid Crystal Display

{
	\section*{M}
}
\noindent 
\textbf{MVC}. \ \ Modelo-Vista-Controlador


{
	\section*{P}
}
\noindent 
\textbf{PC}. \ \ Personal Computer
\newline
\textbf{PWM}. \ \ \label{PWM} Pulse Width Modulation - Modulación por ancho de pulsos.
\newline
\textbf{PCB}. \ \ \label{PCB} Printed Circuit Board - Placa de circuito impreso.
\newline
\textbf{PHP}. \ \ \label{PHP} Pre Hypertext - Processor.





{
	\section*{R}
}
\noindent 
\textbf{RF}. \ \ \label{RF} Requisito Funcional
\newline
\textbf{RNF}. \ \ \label{RNF} Requisito no Funcional

{
	\section*{S}
}
\noindent 
\textbf{SDK}. \ \ Software Development Kit
\newline
\textbf{SPI}. \ \ \label{SPI} Serial Peripheral Interface
\newline
\textbf{SIM}. \ \ \label{SIM}  Subscriber identity module


{
	\section*{T}
}

\noindent 
\textbf{THT}. \ \ 	\label{THT} Through Hole Technology

{
	\section*{U}
}

\noindent 
\textbf{UBS}. \ \ 	\label{USB} Universal Serial Bus


{
	\section*{X}
}
\noindent 
\textbf{XML}. \label{XML} \ \ eXtensible Markup Language


 


\appendix
\chapter{Diagramas circuito}
\label{ap:diagramas}

\begin{figure}[h]
	\centering
	\includegraphics[width=1.0\textwidth]{../images/circuito1}
	\caption[Versión 0 del dispositivo]{Versión inicial del dispositivo, montado sobre una placa de prototipado, así como un módulo LCD con botonera y utilizando el controlador Arduino Mega.}
	\label{circuito1}
\end{figure}



\begin{figure}[h]
	\centering
	\includegraphics[width=0.65\textwidth]{../images/circuito2}
	\caption[Versión 1 del dispositivo]{Versión con pantalla LCD interfaz clásica, botonera analógica y Arduino UNO.}
	\label{circuito2}
\end{figure}


\begin{figure}[h]
	\centering
	\includegraphics[width=0.9\textwidth]{../images/circuito3}
	\caption[Versión 2 del dispositivo]{Versión con pantalla LCD con interfaz I2C, mando Nunchuck, sensor de temperatura LM35 (interfaz analógico) y con controlador Arduino UNO.}
	\label{circuito3}
\end{figure}


\begin{figure}[h]
	\centering
	\includegraphics[width=0.9\textwidth]{../images/circuito4}
	\caption[Versión 3 del dispositivo]{Versión con pantalla LCD con interfaz I2C, mando Nunchuck, sensor de temperatura DHT11 (interfaz digital) ycon controlador Arduino UNO.}
	\label{circuito4}
\end{figure}







\chapter{Diagramas de Gantt}
\label{gantt}



\begin{figure}[h]
\centering
\includegraphics[width=0.41\linewidth]{../images/gantt_inicial}
\caption{Diagrama de Gantt Teórico}
\label{gantt_inicial}
\end{figure}




\begin{figure}[h]
\centering
\includegraphics[width=0.41\linewidth]{../images/gantt_inicial}
\caption{Diagrama de Gantt Real}
\label{gantt_final}
\end{figure}
\chapter{Diseño Carcasa}
\label{ap:caja}

\begin{figure}[h]
	\centering
	\includegraphics[width=0.65\textwidth]{../images/ardufocuser_Box_3mm}
	\caption[Plano para cortar la carcasa con láser CNC]{Plano para cortar la carcasa con láser CNC.}
	\label{fig:plano_carcasa}
\end{figure}


\begin{figure}[h]
	\centering
	\includegraphics[width=0.9\textwidth]{../images/render_caja}
	\caption[Vista tridimensional de la carcasa]{Vista tridimensional del ensamblado de la carcasa. Visualización creada con Blender.}
	\label{fig:render_caja}
\end{figure}
\chapter{Pruebas}


\begin{lstlisting}[language=cpp, 
				   caption={Pruebas lectura de valores potenciómetros},
				   label={lst:potenciometro_test_code}]
#include <Wire.h>
#include <LiquidCrystal_I2C.h>

#define MIN 0
#define MAX 100

LiquidCrystal_I2C lcd(0x27, 16, 2);
unsigned long  lastTimeUpdate=100;
int lastPulse;

void PotController(){
	int result, result2;
	int pot = analogRead(A0);
	int pot2 = analogRead(A1);
	result = map(pot, 0, 1024, MIN, MAX );
	result2 = map(pot2, 0, 1024, MIN, MAX );	
	lcd.clear();
	lcd.print(result); lcd.print("  "); lcd.print(result2);
}

void setup() {
	Serial.begin(9600);
	lcd.begin();
	lcd.backlight();
	lcd.print("RUN!");
}

void loop() {
	if (millis() > lastTimeUpdate) {
	PotController();
	lastTimeUpdate = millis() + 100;
	}
}
\end{lstlisting}




\begin{figure}[h]
\centering
\includegraphics[width=0.9\linewidth]{../images/test_potenciometros}
\caption{Circuito para las pruebas lectura de valores potenciómetros}
\label{fig:test_potenciometros}
\end{figure}


\begin{lstlisting}[language=cpp,
				   caption={Prueba mover motor paso a paso},
				   label={lst:motor_test_code}]
#include <AccelStepper.h>
#define PINDIR 3
#define PINSTEP 2

AccelStepper motor(1, PINSTEP, PINDIR);

void setup() {
	Serial.begin(9600);
	motor.setMaxSpeed(500);
	motor.setAcceleration(100);
}

void loop() {
	motor.run();
	motor.moveTo(-3000);
}


\end{lstlisting}

\newpage

\begin{figure}[h]
	\centering
	\includegraphics[width=0.9\linewidth]{../images/test_motor}
	\caption{Circuito para la prueba mover motor paso a paso}
	\label{fig:test_motor_circuit}
\end{figure}


\newpage
\begin{lstlisting}[language=cpp,
				   caption={Prueba Wii nunchuck},
				   label={lst:nunchuck_test_code}]
#include <Wire.h>
#include <math.h>
#include <nunchuck.h>
#include <LiquidCrystal_I2C.h>

LiquidCrystal_I2C lcd(0x27, 16, 2);
WiiChuck chuck = WiiChuck();
const int btnRIGHT=0, btnUP=1, btnDOWN=2, btnLEFT=3;
const int btnC=4, btnZ=5, btnNONE=6;
int lastTimeUpdate=1000;
int lastPulse;

void nunckuckController(){
	chuck.update();
	if(chuck.cPressed() && lastPulse!=btnC ){
		lcd.clear();
		lcd.print("C");
		lastPulse=btnC;
	}if(chuck.zPressed() && lastPulse!=btnZ){
		lcd.clear();
		lcd.print("Z");
		lastPulse=btnZ;
	}if(chuck.rightJoy() && lastPulse!=btnRIGHT){
		lcd.clear();
		lcd.print("RIGHT");
		lastPulse=btnRIGHT;
	}if(chuck.leftJoy()&& lastPulse!=btnLEFT){
		lcd.clear();
		lcd.print("LEFT");
		lastPulse=btnLEFT;
	}else  {
		lastPulse=btnNONE;
	}
}

void setup() {
	Serial.begin(9600);
	lastPulse=btnNONE;
	lcd.begin();
	lcd.backlight();
	chuck.begin();
	chuck.update();
}

void loop() {
	if (millis() > lastTimeUpdate) {
		nunckuckController();
		lastTimeUpdate = millis();
	}
}
\end{lstlisting}

\begin{figure}[h]
	\centering
	\includegraphics[width=0.9\linewidth]{../images/test_nunchuck}
	\caption{Circuito para la prueba Wii nunchuck}
	\label{fig:test_nunchuck_circuit}
\end{figure}










\bibliographystyle{bibliografia/splncs}
\bibliography{bibliografia/bibliografia}


=======
>>>>>>> c9f08dfe66521d4f0dba18e652f93a6a37a333aa


\thispagestyle{empty}

<<<<<<< HEAD
=======
\newpage
%\bibliography{BIBLIO}
%\bibliographystyle{apalike}
\bibliography{bibliografia}\addcontentsline{toc}{chapter}{Bibliografía}
%\bibliographystyle{miunsrturl}
\bibliographystyle{ieeetr}
>>>>>>> c9f08dfe66521d4f0dba18e652f93a6a37a333aa


\end{document}
